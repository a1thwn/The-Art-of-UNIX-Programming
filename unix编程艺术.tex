% !Mode:: "TeX:UTF-8"%確保文檔utf-8編碼
%新加入的命令如下:addchtoc addsectoc reduline printendnotes hlabel
%新加入的环境如下:common-format  fig scalefig 

\documentclass[12pt,oneside]{book}
\newlength{\textpt}
\setlength{\textpt}{12pt} 
\newif\ifphone
\phonefalse

\usepackage{myconfig}
\usepackage{mytitle}


\begin{document}
\frontmatter   

\titlea{unix编程}
\titleb{艺术}
\titlec{OCR识别+精确校对版}
\author{Eric S. Raymond}
\authorinfo{作者:埃里克·斯蒂芬·雷蒙(Eric Steven Raymond,著名黑客。)}
\editor{德山书生}
\email{a358003542@gmail.com}
\editorinfo{\href{http://www.catb.org/~esr/writings/taoup/html/}{官方英文网站},}
\version{0.01}
\titleLC


\addchtoc{目录}
\setcounter{tocdepth}{2}    
\tableofcontents

\begin{common-format}
\mainmatter 

\chapter{哲学}

\begin{flushright}
\begin{notecard}[red!30]
不懂Unix的人注定最终还要重复发明一个蹩脚的Unix。

Usenet 签名,1987年11月

{\hfill —Henry Spencer}
\end{notecard}
\end{flushright}


\section{文化?什么文化}
这是一本讲Unix编程的书,然而在这本书里,我们将反复提到“文化”、“艺术”以及“哲学”这些字眼。如果你不是程序员,或者对Unix涉水未深,这可能让你感觉很奇怪。但是Unix确实有它自己的文化;有独特的编程艺术;有一套影响深远的设计哲学。理解这些传统,会使你写出更好地软件,即使你是在非Unix平台开发。

工程和设计的每个分支都有自己的技术文化。在大多数工程领域中,就一个专业人员的素养组成来说,有些不成文的行业素质具有与标准手册及教科书同等重要的地位(并且随着专业人员经验的日积月累,这些经验常常会比书本更重要)。资深工程师们在工作中会积累大量的隐形只是,他们用类似禅宗“教外别传”[译注\footnote{禅宗用语,不依文字、语言、直悟佛陀所悟之境界,即称为教外别传。}]的方式,通过言传身教传授给后辈。

软件工程师算是此规则的一个例外:技术变革如此之快,软件环境日新月异,软件技术文化暂如朝露。然而,例外之中也有例外。确有极少数软件技术被证明经久耐用,足以演进为强势的技术文化、有鲜明特色的艺术和世代相传的设计哲学。

Unix文化便是其一。互联网文化又是其一——或者,这两者在21世纪无可争议地合二为一。其实,从1980年代早期开始,Unix和互联网便越来越难以分割,本书也无意强求区分。


\section{Unix的生命力}
Unix诞生于1969年,此后便一直应用于生产领域。按照计算机工业的标准,那已经是好几个地质纪元前的事了——比PC机、工作站、微处理器甚至视频显示终端都要早,与第一块半导体存储器是同一个时代的古物。在现今所有分时系统中,也只有IBM的VM/CMS敢说它比Unix资格更老,但是Unix机器的服务时间却是VM/CMS的几十万倍;事实上,在Unix平台上完成的计算量可能比所有其他分时系统加起来的总和还要多。

Unix比其它任何操作系统都更广泛地应用在各种机型上。从超级计算机到手持计算机到嵌入式网络设备,从工作站到服务器到PC机到微型计算机。Unix所能支持的机器架构和奇特硬件可能比你随便抓取任何其他三种操作系统所能支持的总和还要多。

Unix应用范围之广简直令人难以置信。没有哪一种操作系统能像Unix那样,能同时在作为研究工具、定制技术应用的友好宿主机、商用成品软件平台和互联网技术的重要部分等各个领域都大放异彩。

从Unix诞生之日起,各种信誓旦旦的预言就伴随着它,说Unix必将衰败,或者被其他操作系统挤出市场。可是在今天,化身为Linux、BSD、Solaris、MacOS X以及好几种其它变种的Unix,却显得前所未有的强大。
\begin{quote}[Ken Thompson]
Robert Metcalf[以太网络的发明者]曾说过:如果将来有什么技术来取代以太网,那么这个取代物的名字还会叫“以太网”。因此以太网是永远不会消亡的\footnote{事实上,以太网已经两次被不同的技术所取代,只是名字没有变。第一次是双绞线取代了同轴电缆,第二次是千兆以太网的出现。}。Unix也多次经历了类似的转变。
\end{quote}

至少,Unix的一个核心技术——C语言——已经在其他系统中植根。事实上,如果没有无处不在的C语言这个通用语言,还如何奢谈系统级软件工程。Unix还引入了如今广泛采用的带目录节点的树形文件名字空间已经用于程序间通信的管道机制。

Unix的生命力和适应力委实令人惊奇。尽管其它技术如蜉蝣般生生灭灭,计算机性能成千倍增长,语言历经嬗变,业界规范几次变革——然而Unix依然巍然屹立,仍在运行,仍在创造价值,仍然能够赢得这个星球上无数最优秀、最聪明的软件技术人员的忠诚。

性能—时间的指数曲线对软件开发过程所引发的结果,就是每过18个月,就有一半的知识会过时。Unix并不承诺让你免遭此劫,只是让你的知识投资更趋稳定。因为不变的东西很多:语言、系统调用、工具用法——它们积年不变,甚至可以用上数十载。而在其他操作系统中则无法预判什么东西会持久不变,有时候甚至整个操作系统都会被淘汰。在Unix中,持久性知识和短期性知识有着明显的区别,人们在一开始学习的时候,就能提前判断(命中率约有九成)要学的知识属于那一类。这些便是Unix有众多忠实用拥趸的原因。

Unix的稳定和成功在很多程度上归功于它与生俱来的内在优势,归功于Ken Thompson, Dennis Ritchie, Brain Kernighan, Doug McIroy, Rob Pike和其他早期Unix开发者一开始作出的设计决策。这些决策,连同设计哲学、编程艺术、技术文化一起,从Unix的婴儿期到今天的成长路程中,已经被反复证明是健康可靠的,而Unix才得以有今天的成功。


\section{反对学习Unix文化的理由}
Unix的耐用性及其技术文化对于喜爱Unix的人们、以及技术史家来说肯定颇为有趣。但是,Unix的本源用途——作为大中型计算机的通用分时系统,由于受到个人工作站的围剿,正迅速地退出舞台,隐入历史的迷雾之中。因而Unix究竟能否在目前被Microsoft主宰的主流商务桌面市场上取得成功,人们自然也存在着一定的疑问。

外行常常把Unix当作是教学用的玩具或者是黑客的沙盒而不屑一顾。有一本著名的抨击Unix的书——《Unix反对者手册》(Unix Hater's Handbook)[Garfinkel],几乎从Unix诞生时就一直奉行反对路线,将Unix的 追随者描写成一群信奉邪教的怪人和失败者。AT\&{}T、Sun、Novell,以及其他一些大型商业销售商和标准联盟在Unix定位和市场推广方面不断铸下的大错也已经成为经典笑柄。

即使在Unix世界里,Unix的通用性也一直受到怀疑,摇摆在危崖边。在持怀疑态度的外行人眼中,Unix很有用,不会消亡,只是等不了大雅之堂:注定只能是个小众的操作系统。

挫败这些怀疑者的不是别的,正是Linux和其他开源Unix(如现代BSD各个变种)的崛起。Unix文化是如此的有生命力,即使十几年的管理不善也丝毫未箝制\footnote{同抑制,而钳制有胁迫之意。}它的勃勃生机。现在Unix社区自身已经重新控制了技术和市场,正快速而有效地解决着Unix的问题(第20章将有详述)。


\section{Unix之失}
对于一个始于1969年的设计来说,在Unix设计中居然很难找到硬伤,这着实令人称奇。其他的选择不是没有,但是每一个这样的选择同样面临争议,无论是Unix爱好者,还是操作系统设计社群的人们。

Unix文件在字节层次以上再无结构可言。文件删除了就没法恢复。Unix的安全模型公认地太过原始。作业控制有欠精致。命名方式非常混乱。或许拥有文件系统本身就是一个错误。我们将在第20章讨论这些技术问题。

但是也许Unix最持久的异议恰恰来自Unix哲学的一个特性,这一条特性是X window设计者首先明确提出的。X致力于提供一套“机制,而不是策略”,以支持一套极端通用的图形操作,从而把工具箱和界面的“观感”(策略)推后到应用层。Unix其他系统级的服务也有类似的倾向:行为的最终逻辑被尽可能推后到使用端。Unix用户可以在多种shell中进行选择。而Unix应用程序通常会提供很多的行为选项和令人眼花缭乱的定制功能。

这种倾向也反映出Unix的遗风:原本是技术人员设计的操作系统;同时也表明设计的信念;最终用户永远比操作系统设计人员更清楚他们究竟需要什么。

\begin{quote}[Doug Mcllroy]
贝尔实验室的Dick Hamming在1950年代便树立了此信条:尽管计算机稀缺昂贵,但是开放式的计算模式,即客户可以为系统写出自己的应用程序,这一点势在必行,因为“用错误的方式解决正确的问题总比用正确的方法解决错误的问题好”。
\end{quote}

然而这种选择机制而不是策略的代价是:当用户“\textbf{可以}”自己设置策略时,他们其实是“必须”自己设置策略。非技术型的终端用户常常会被Unix丰富的选项和接口风格搞得晕头转向,于是转而选择那些伪称能够给他们提供简洁性的操作系统。

只看眼前的话,Unix的这种自由放纵主义风格会让它失去很多非技术性用户,但从长远考虑,最终你会发觉这个“错误”换来至关重要的优势:策略相对短寿,而机制才会长存。现今流行的界面观感常常会变成明日进化的死胡同(去问问那些使用已经过时的X工具包的用户,他们会有一肚子苦水倒给你!)。说来说去,只提供机制不提供方针的哲学能使Unix长久保鲜;而那些被束缚在一套方针或界面风格的操作系统,也许早就从人们的视线中消失了。


\section{Unix之得}
最近Linux爆炸式的发展和Intemet技术重要性的渐增,都给我们充足的理由来否定怀疑者的论断。其实,退一步说,就算怀疑者的断言正确,Unix文化也同样值得研习,因为在有些方面,Unix及其外围文化明显比任何竞争对手都出色。

\subsection{开源软件}
尽管“开源”这个术语和开源定义( the Open Source Definition)直到1998年才出现,但是自由共享源码的同僚严格复审的开发方式打从Unix诞生起就是其文化最具特色的部分。

最初十年中的AT\&{}T原始Unix,及其后来的主要变种Berkeley Unix,通常都随源代码一起发布。下文要提到的Unix的优势,大多数也由此而来。

\subsection{跨平台可移植性和开放标准}
Unix仍是唯一一个在不同种类的计算机、众多厂商、各种专用硬件上提供了一个一致的、文档齐全的应用程序接口(API)的操作系统。Unix也是唯一一个从嵌入式芯片、手持设备到桌面机,从服务器到专门用于数值计算的怪兽级计算机以及数据库后端都腾挪有余的操作系统。

Unix API几乎就可以作为编写真正可移植软件的硬件无关标准。难怪最初IEEE称之为“可移植操作系统标准”(Portabje Operating System Standard)的POS很快就被大家加了后缀变成了“POSIX”[译注:缩写为POSIX是为了读音更像Unix]。  确实,只有称之为Unix API的等价物才能算是这种标准比较可信的模型。

其它操作系统只提供二进制代码的应用程序,并随其诞生环境的消亡而消亡,而Unix源码却是永生的。至少,永生在数十年不断维护翻修它们的Unix技术文化之中。

\subsection{Internet和万维网}
美国国防部将第一版TCP/IP协议栈的开发合同交给一个Unix研发组就是因为考虑到Unix大部分是开放源码。除了TCP/IP之外,Unix也已成为互联网服务提供商(Intemet Service Provider)行业不可或缺的核心技术之一。甚至在1980年代中期TOPS系列操作系统消亡之际,大部分互联网服务器(实际上PC以上所有级别的机器)都依赖于Unix。

在Intemet市场上,Unix甚至面对Microsoft可怕的行销大锤也毫发无伤。  虽然成型于TOPS-10的TCP/IP标准(互联网的基础)在理论上可以与Unix分开,但当应用在其它操作系统上时,  一直都饱受兼容性差、不稳定、bug太多等问题的困扰。实际上,理论和规格说明人人都可以获取,但是只有Unix世界中你才见得到这些稳固可靠的现实成果。

互联网技术文化和Unix文化在1980年代早期开始汇合,现在已经共生共存,难以分割。万维网的设计——也就是互联网的现代面孔,从其祖先ARPANET所得到的,不比从Unix得到的更多。实际上,统一资源定位符URL(Uniform Resource Locator)作为Web的核心概念,也是Unix中无处不在的统一文件名字空间概念的泛化。要作为一个有效的网络专家,对Unix及其文化的理解绝对是必不可少的。

\subsection{开源社区}
伴随早期Unix源码发布而形成的社群从未消亡——在1990年代早期互联网技术的爆炸式发展之后,这个社群新造就了整整一代的使用家用机的狂热黑客。

今天,Unix社区是各种软件开发的强大支持组。  高质量的开源开发工具在Unix世界极为丰富(在本书中我们会讲到很多)。开源的Unix应用程序已经达到、或者超越它们专属同侪的高度[Fuzz]。整个Unix操作系统连同完整的工具包、基本的应用套件,都可以在互联网上免费获取。既然能够改编、重用、再造,节省自己90\%的工作量,为什么还要从零开始编码呢?

通过协作开发与代码复用路上艰辛的探索,才耕耘出代码共享的传统。不是在理论上,而是通过大量工程实践,才有了这些并非显而易见的设计规则:程序得以形成严丝合缝的工具套装,而不是应景的解决对策。本书的一个主要目的就是阐明这些原则。

今天,方兴未艾的开源运动给Unix传统注入了新的血液、新的技术方法,同时也带来了新一代年轻而有才华的程序员。包括Linux操作系统以及共生的应用程序如Apache、Mozilla等开源项目已经使Unix传统在主流世界空前亮眼与成功。如今,在争相对未来计算基础设施进行定义的这场竞争中,开源运动似乎已经站在了胜利的边缘——新架构的核心正是运行在互联网上的Unix机器。

\subsection{从头到脚的灵活性}
许多操作系统自诩比起Unix来有多么的“现代”,用户界面又是多么的“友好”。它们漂亮外表的背后,却是以貌似精巧实则脆弱狭隘难用的编程接口,把用户和开发者禁锢在单一的界面方针下。在这样的操作系统中,完成设计者(指操作系统)预见的任务很容易,但如果要完成设计者没有预料到的任务,用户不是无计可施就是痛苦不堪。

相反,Unix具有非常彻底的灵活性。Unix提供众多的程序粘合手段,这意味着Unix基本工具箱的各种组件连纵开合后,将收到单个工具设计者无法想象的功效。

Unix支持多种风格的程序界面(通常也因为给终端用户增加了明显的系统复杂度而被视为Unix的一个缺点),从而增加了它的灵活性:只管简单数据处理的程序而无需背上精巧图形界面的担子。

Unix传统将重点放在尽力使各个程序接口相对小巧、简洁和正交——这也是另一个提高灵活性的方面。整个Unix系统,容易的事还是那么容易,困难的事呢,至少是有可能做到的。

\subsection{Unix Hack之趣}
那些夸夸其谈Unix技术优越性的家伙一般不会提到Unix的终极法宝、它赖以成功的原因:Unix Hack的趣味。

一些Unix的玩家有时羞于认同这一点,似乎这会破坏他们的正统形象。但是,确实如此,同Unix打交道,搞开发就是好玩:现在是,且一向如是。

并没有多少操作系统会被人们用“好玩”来描述。实际上,在其它操作系统下搞开发的摩擦和艰辛,就像是有人比喻的“把一头搁浅的死鲸推下海”一样费力不讨好;或者,最客气的也就是“尚可容忍”、“不是太痛苦”之类形容词。与之成鲜明对比的是,在Unix世界里,操作系统以成就感而不是挫折感来回报人们的努力。Unix下的程序员通常会把Unix当作一个积极有效的帮手,而不是把操作系统当作一个对手还非得用蛮力逼迫它干活。

这一点有着实实在在的重要经济意义。趣味性在Unix早期的历史中开启了一个良性循环。正是因为人们喜爱Unix,所以编制了更多的程序让它用起来更好,  而如今,连编制一个完整商用产品级的开源Unix操作系统都成了一项爱好。如果想知道这是多么惊人的伟绩,想想看你什么时候听说过谁为了好玩来临摹OS/360或者VAX VMS或者Microsoft Windows就行了。

从设计角度来说,趣味性也绝非无足轻重。对于程序员和开发人员来说,如果完成某项任务所需要付出的努力对他们是个挑战却又恰好还在力所能及的范围内,他们就会觉得很有乐趣。因此,趣味性是一个峰值效率的标志。充满痛苦的开发环境只会浪费劳动力和创造力;这样的环境会在无形之中耗费大量时间、资金,还有机会。

就算Unix在其它各个方面都一无足处,Unix的工程文化仍然值得学习,它使得开发过程充满乐趣。乐趣是一个符号,意味着效能、效率和高产。

\subsection{Unix的经验别处也可适用}
在探索开发那些我们如今已经觉得理所当然的操作系统特性的过程中,Unix程序员已经积累了几十年的经验。哪怕是非Unix的程序员也能够从这些经验中获益。好的设计原则和开发方法在Unix上实施相对容易,所以Unix是一个学习这些原则和方法的良好平台。

在其它操作系统下,要做到良好实践通常要相对困难一些,但是尽管如此,Unix文化中的有益经验仍然可以借鉴。多数Unix代码(包括所有的过滤器、主要脚本语言和大多数代码生成器)都可以直接移植到任何只要支持ANSI C的操作系统中(原因在于C语言本身就是Unix的一项发明,而ANSI C程序库表述了相当大一部分的Unix服务)。


\section{Unix哲学基础}
Unix哲学起源于Ken Thompson早期关于如何设计一个服务接口简洁、小巧精干的操作系统的思考,随着Unix文化在学习如何尽可能发掘Thompson设计思想的过程中不断成长,同时一路上还从其它许多地方博采众长。

Unix哲学说来不算是一种正规设计方法。它并不打算从计算机科学的理论高度来产生理论上完美的软件。那些毫无动力、松松垮垮而且薪水微薄的程序员们,能在短短期限内,如同神灵附体般造出稳定而新颖的软件——这只不过是经理人永远的梦呓罢了。
    
Unix哲学(同其它工程领域的民间传统一样)是自下而上的,而不是自上而下的。Unix哲学注重实效,立足于丰富的经验。你不会在正规方法学和标准中找到它,它更接近于隐性的半本能的知识,即Unix文化所传播的专业经验。它鼓励那种分清轻重缓急的感觉,以及怀疑一切的态度,并鼓励你以幽默达观的态度对待这些。
   
Unix管道的发明人、Unix传统的奠基人之一Doug Mcllroy在[Mcllroy78]中曾经说过:
\begin{enumerate}
\renewcommand{\labelenumi}{(\roman{enumi})}
\item 让每个程序就做好一件事。如果有新任务,就重新开始,不要往原程序中加入新功能而搞得复杂。
\item 假定每个程序的输出都会成为另一个程序的输入,哪怕那个程序还是未知的。输出中不要有无关的信息干扰。避免使用严格的分栏格式和二进制格式输入。不要坚持使用交互式输入。
\item 尽可能早地将设计和编译的软件投入试用,哪怕是操作系统也不例外,理想情况下,应该是在几星期内。对拙劣的代码别犹豫,扔掉重写。
\item 优先使用工具而不是拙劣的帮助来减轻编程任务的负担。工欲善其事,必先利其器。
\end{enumerate}

后来他这样总结道( 引自《Unix的四分之一世纪》  (A Quarter Century of Unix [Salus]  ):

Unix哲学是这样的:一个程序只做一件事,并做好。程序要能协作。程序要能处理文本流,因为这是最通用的接口。
    
Rob Pike,最伟大的C语言大师之一,在《Notes on C  Programming》中从另一个稍微不同的角度表述了Unix的哲学[Pike]:

原则1:你无法断定程序会在什么地方耗费运行时间。瓶颈经常出现在想不到的地方,所以别急于胡乱找个地方改代码,除非你已经证实那儿就是瓶颈所在。

原则2:估量。在你没对代码进行估量,特别是没找到最耗时的那部分之前,别去优化速度。

原则3:花哨的算法在n很小时通常很慢,而n通常很小。花哨算法的常数复杂度很大。除非你确定n总是很大,否则不要用花哨算法(即使n很大,也优先考虑原则2)。

原则4:花哨的算法比简单算法更容易出bug、更难实现。尽量使用简单的算法配合简单的数据结构。

原则5:数据压倒一切。如果已经选择了正确的数据结构并且把一切都组织得井井有条,正确的算法也就不言自明。编程的核心是数据结构,而不是算法\footnote{引用是来自于The Mythical Man - Month【Brooks】早期的版本;引语为“给我看流程图而不让我看(数据)表,我仍会茫然不解;如果给我看(数据)表,通常就不需要流程图了;数据表是够说明问题了。”}。

原则6:没有原则6。

Ken Thompson——Unix最初版本的设计者和实现者,禅宗偈语般地对Pike的原则4作了强调:

拿不准就穷举。
    
Unix哲学中更多的内容不是这些先哲们口头表述出来的,而是由他们所作的一切和Unix本身所作出的榜样体现出来的。从整体上来说,可以概括为以下几点:
\begin{enumerate}
\item 模块原则:使用简洁的接口拼合简单的部件。
\item 清晰原则:清晰胜于机巧。
\item 组合原则:设计时考虑拼接组合。
\item 分离原则:策略同机制分离,接口同引擎分离。
\item 简洁原则:设计要简洁,复杂度能低则低。
\item 吝啬原则:除非确无它法,不要编写庞大的程序。
\item 透明性原则:设计要可见,以便审查和调试。
\item 健壮原则:健壮源于透明与简洁。
\item 表示原则:把知识叠入数据以求逻辑质朴而健壮。
\item 通俗原则:接口设计避免标新立异。
\item 缄默原则:如果一个程序没什么好说的,就沉默。
\item 补救原则:出现异常时,马上退出并给出足够错误信息。
\item 经济原则:宁花机器一分,不花程序员一秒。
\item 生成原则:避免手工hack,尽量编写程序去生成程序。
\item 优化原则:雕琢前先要有原型,跑之前先学会走。
\item 多样原则:决不相信所谓“不二法门”的断言。
\item 扩展原则:设计着眼未来,未来总比预想来得快。
\end{enumerate}

如果刚开始接触Unix,这些原则值得好好体味一番。谈软件工程的文章常常会推荐大部分的这些原则,但是大多数其它操作系统缺乏恰当的工具和传统将这些准则付诸实践,所以,多数的程序员还不能自始至终地贯彻这些原则。蹩脚的工具、糟糕的设计、过度的劳作和臃肿的代码对他们已经是家常便饭了;他们奇怪,Unix的玩家有什么好烦的呢。


\subsection{模块原则:使用简洁的接口拼合简单的部件}
正如Brian Kernighan曾经说过的:“计算机编程的本质就是控制复杂度”[Kernighan-Plauger]。排错占用了大部分的开发时间,弄出一个拿得出手的可用系统,通常与其说出自才华横溢的设计成果,还不如说是跌跌撞撞的结果。

汇编语言、编译语言、流程图、过程化编程、结构化编程、所谓的人工智能、第四代编程语言、面向对象、以及软件开发的方法论,不计其数的解决之道被抛售者吹得神乎其神。但实际上这些都用处不大,原因恰恰在于它们“成功’’地将程序的复杂度提升到了人脑几乎不能处理的地步。就像Fred Brooks的一句名言[Brooks]:没有万能药。

要编制复杂软件而又不至于一败涂地的唯一方法就是降低其整体复杂度——用清晰的接口把若干简单的模块组合成一个复杂软件。如此一来,多数问题只会局限于某个局部,那么就还有希望对局部进行改进而不至牵动全身。


\subsection{清晰原则:清晰胜于机巧}
维护如此重要而成本如此高昂;在写程序时,要想到你不是写给执行代码的计算机看的,而是给人——将来阅读维护源码的人,包括你自己——看的。

在Unix传统中,这个建议不仅意味着代码注释。良好的Unix实践同样信奉在选择算法和实现时就应该考虑到将来的可扩展性。而为了取得程序一丁点的性能提升就大幅度增加技术的复杂性和晦涩性,这个买卖做不得——这不仅仅是因为复杂的代码容易滋生bug,也因为它会使日后的阅读和维护工作更加艰难。

相反,优雅而清晰的代码不仅不容易崩溃——而且更易于让后来的修改者立刻理解。这点非常重要,尤其是说不定若干年后回过头来修改这些代码的人可能恰恰就是你自己。

\begin{quote}[Henry Spencer]
永远不要去吃力地解读一段晦涩的代码三次。第一次也许侥幸成功,但如果发现必须重新解读一遍——离第一次太久了,具体细节无从回想——那么你该注释代码了,这样第三次就相对不会那么痛苦了。
\end{quote}


\subsection{组合原则:设计时考虑拼接组合}
如果程序彼此之间不能有效通信,那么软件就难免会陷入复杂度的泥淖。

在输入输出方面,Unix传统极力提倡采用简单、文本化、面向流、设备无关的格式。在经典的Unix下,多数程序都尽可能采用简单过滤器的形式,即将一个输入的简单文本流处理为一个简单的文本流输出。

抛开世俗眼光,Unix程序员偏爱这种做法并不是因为他们仇视图形用户界面,而是因为如果程序不采用简单的文本输入输出流,它们就极难衔接。

Unix中,文本流之于工具,就如同在面向对象环境中的消息之于对象。文本流界面的简洁性加强了工具的封装性。而许多精致的进程间通讯方法,比如远程过程调用,都存在牵扯过多各程序间内部状态的倾向。

要想让程序具有组合性,就要使程序彼此独立。在文本流这一端的程序应该尽可能不要考虑文本流另一端的程序。将一端的程序替换为另一个截然不同的程序,而完全不惊扰另一端应该很容易做到。

GUI可以是个好东西。有时竭尽所能也不可避免复杂的二进制数据格式。但是,在做一个GUI前,最好还是应该想想可不可以把复杂的交互程序跟干粗活的算法程序分离开,每个部分单独成为一块,然后用一个简单的命令流或者是应用协议将其组合在一起。在构思精巧的数据传输格式前,有必要实地考察一下,是否能利用简单的文本数据格式;以一点点格式解析的代价,换得可以使用通用工具来构造或解读数据流的好处是值得的。

当程序无法自然地使用序列化、协议形式的接口时,正确的Unix设计至少是,把尽可能多的编程元素组织为一套定义良好的API。这样,至少你可以通过链接调用应用程序,或者可以根据不同任务的需求粘合使用不同的接口。

(我们将在第7章详细讨论这些问题。)


\subsection{分离原则:策略同机制分离,接口同引擎分离}
在Unix之失的讨论中,我们谈到过X系统的设计者在设计中的基本抉择是实行“机制,而不是策略”这种做法——使X成为一个通用图形引擎,而将用户界面风格留给工具包或者系统的其它层次来决定。这一点得以证明是正确的,因为策略和机制是按照不同的时间尺度变化的,策略的变化要远远快于机制。GUI工具包的观感时尚来去匆匆,而光栅操作和组合却是永恒的。

所以,把策略同机制揉成一团有两个负面影响:一来会使策略变得死板,难以适应用户需求的改变,二来也意味着任何策略的改变都极有可能动摇机制。

相反,将两者剥离,就有可能在探索新策略的时候不足以打破机制。另外,我们也可以更容易为机制写出较好的测试(因为策略太短命,不值得花太多精力在这上面)。

这条设计准则在GUI环境之外也被广泛应用。总而言之,这条准则告诉我们应该设法将接口和引擎剥离开来。

实现这种剥离的一个方法是,比如,将应用按照一个库来编写,这个库包含许多由内嵌脚本语言驱动的C服务程序,而至于整个应用的控制流程则用脚本来撰写而不是用C语言。这种模式的经典例子就是Emacs编辑器,它使用内嵌的脚本语言Lisp解释器来控制用C编写的编辑原语操作。我们会在第11章讨论这种设计风格。

另一个方法是将应用程序分成可以协作的前端和后端进程,通过套接字上层的专用应用协议进行通讯:我们会在第5章和第7章讨论这种设计。前端实现策略,后端实现机制。比起仅用单个进程的整体实现方式来说,这种双端设计方式大大降低了整体复杂度,bug有望减少,从而降低程序的寿命周期成本。


\subsection{简洁原则:设计要简洁,复杂度能低则低}
来自多方面的压力常常会让程序变得复杂(由此代价更高,bug更多),其中一种压力就是来自技术上的虚荣心理。程序员们都很聪明,常常以能玩转复杂东西和耍弄抽象概念的能力为傲,这一点也无可厚非。但正因如此,他们常常会与同行们比试,看看谁能够鼓捣出最错综复杂的美妙事物。正如我们经常所见,他们的设计能力大大超出他们的实现和排错能力,结果便是代价高昂的废品。

\begin{quote}[Doug Mcllroy]
“错综复杂的美妙事物”听来自相矛盾。Unix程序员相互比的是谁能够做到“简洁而漂亮”并以此为荣,这一点虽然只是隐含在这些规则之中,但还是很值得公开提出来强调一下。
\end{quote}    

更为常见的是(至少在商业软件领域里),过度的复杂性往往来自于项目的要求,而这些要求常常基于当月的推销热点,而不是基于顾客的需求和软件实际能够提供的功能。许多优秀的设计被市场推销所需要的大堆大堆“特性清单”扼杀——实际上,这些特性功能几乎从未用过。然后,恶性循环开始了:比别人花哨的方法就是把自己变得更花哨。很快,庞大臃肿变成了业界标准,每个人都在使用臃肿不堪、bug极多的软件,连软件开发人员也不敢敝帚自珍。

无论以上哪种方式,最后每个人都是失败者。

要避免这些陷阱,唯一的方法就是鼓励另一种软件文化,以简洁为美,人人对庞大复杂的东西群起而攻之——这是一个非常看重简单解决方案的工程传统,总是设法将程序系统分解为几个能够协作的小部分,并本能地抵制任何用过多噱头来粉饰程序的企图。
    
这就有点Unix文化的意味了。


\subsection{吝啬原则:除非确无它法,不要编写庞大的程序}
“大”有两重含义:体积大,复杂程度高。程序大了,维护起来就困难。由于人们对花费了大量精力才做出来的东西难以割舍,结果导致在庞大的程序中把投资浪费在注定要失败或者并非最佳的方案上。

(我们会在第13章就软件的最佳大小进行更多的详细讨论。)

\subsection{透明性原则:设计要可见,以便审查和调试}
因为调试通常会占用四分之三甚至更多的开发时间,所以一开始就多做点工作以减少日后调试的工作量会很划算。一个特别有效的减少调试工作量的方法就是设计时充分考虑透明性和显见性。

软件系统的透明性是指你一眼就能够看出软件是在做什么以及怎样做的。显见性指程序带有监视和显示内部状态的功能,这样程序不仅能够运行良好,而且还可以看得出它以何种方式运行。

设计时如果充分考虑到这些要求会给整个项目全过程都带来好处。至少,调试选项的设置应该尽量不要在事后,而应该在设计之初便考虑进去。这是考虑到程序不但应该能够展示其正确性,也应该能够把原开发者解决问题的思维模型告诉后来者。

程序如果要展示其正确性,应该使用足够简单的输入输出格式,这样才能保证很容易地检验有效输入和正确输出之间的关系是否正确。

出于充分考虑透明性和显见性的目的,还应该提倡接口简洁,以方便其它程序对其进行操作——尤其是测试监视工具和调试脚本。


\subsection{健壮原则:健壮源于透明与简洁}
软件的健壮性指软件不仅能在正常情况下运行良好,而且在超出设计者设想的意外条件下也能够运行良好。

大多数软件禁不起磕碰,毛病很多,就是因为过于复杂,很难通盘考虑。如果不能够正确理解一个程序的逻辑,就不能确信其是否正确,也就不能在出错的时候修复它。

这也就带来了让程序健壮的方法,就是让程序的内部逻辑更易于理解。要做到这一点主要有两种方法:透明化和简洁化。
\begin{quote}[Henry Spencer]
就健壮性而言,设计时要考虑到能承受极端大量的输入,这一点也很重要。这时牢记组合原则会很有益处;经不起其它一些程序产生的输入(例如,原始的Unix C编译器据说需要一些小小的升级才能处理好Yacc的输出)。当然,这其中涉及的一些形式对人类来说往往看起来没什么实际用处。比如,接受空的列表/字符串等等,即使在人们很少或者根本就不提供空字符串的地方也得如此,这可以避免在用机器生成输入时需要对这种情况进行特殊处理。
\end{quote}
    
在有异常输入的情况下,保证软件健壮性的一个相当重要的策略就是避免在代码中出现特例。bug通常隐藏在处理特例的代码以及处理不同特殊情况的交互操作部分的代码中。

上面我们曾说过,软件的透明性就是指一眼就能够看出来是怎么回事。如果“怎么回事”不算复杂,即人们不需要绞尽脑汁就能够推断出所有可能的情况,那么这个程序就是简洁的。程序越简洁,越透明,也就越健壮。

模块性(代码简朴,接口简洁)是组织程序以达到更简洁目的的一个方法。另外也有其它的方法可以得到简洁。接下来就是另一个。

\subsection{表示原则:把知识叠入数据以求逻辑质朴而健壮}
即使最简单的程序逻辑让人类来验证也很困难,但是就算是很复杂的数据,对人类来说,还是相对容易地就能够推导和建模的。不信可以试试比较一下,是五十个节点的指针树,还是五十行代码的流程图更清楚明了;或者,比较一下究竟用一个数组初始化器来表示转换表,还是用switch语句更清楚明了呢?可以看出,不同的方式在透明性和清晰性方面具有非常显著的差别。参见Rob Pike的原则5。

数据要比编程逻辑更容易驾驭。所以接下来,如果要在复杂数据和复杂代码中选择一个,宁愿选择前者。更进一步:在设计中,你应该主动将代码的复杂度转移到数据之中去。

此种考量并非Unix社区的原创,但是许多Unix代码都显示受其影响。特别是C语言对指针使用控制的功能,促进了在内核以上各个编码层面上对动态修改引用结构。在结构中用非常简单的指针操作就能够完成的任务,在其它语言中,往往不得不用更复杂的过程才能完成。

(我们将在第9章再讨论这些技术。)

\subsection{通俗原则:接口设计避免标新立异}
(也就是众所周知的“最少惊奇原则”。)

最易用的程序就是用户需要学习新东西最少的程序——或者,换句话说,最易用的程序就是最切合用户已有知识的程序。

因此,接口设计应该避免毫无来由的标新立异和自作聪明。如果你编制一个计算器程序,‘+’应该永远表示加法。而设计接口的时候,尽量按照用户最可能熟悉的同样功能接口和相似应用程序来进行建模。

关注目标受众。他们也许是最终用户,也许是其他程序员,也许是系统管理员。对于这些不同的人群,最少惊奇的意义也不同。

关注传统惯例。Unix世界形成了一套系统的惯例,比如配置和运行控制文件的格式,命令行开关等等。这些惯例的存在有个极好的理由:缓和学习曲线。应该学会并使用这些惯例。

(我们将在第5章和第10章讨论这些传统惯例。)

\begin{quote}[Henry Spencer]
最小立异原则的另一面是避免表象相似而实际却略有不同。这会极端危险,因为表象相似往往导致人们产生错误的假定。所以最好让不同事物有明显区别,而不要看起来几乎一模一样。
\end{quote}


\subsection{缄默原则:如果一个程序没什么好说的,就保持沉默}
Unix最古老最持久的设计原则之一就是:若程序没有什么特别之处可讲,就保持沉默。行为良好的程序应该默默工作,决不唠唠叨叨,碍手碍脚。沉默是金。

“沉默是金”这个原则的起始是源于Unix诞生时还没有视频显示器。在1969年的缓慢的打印终端,每一行多余的输出都会严重消耗用户的宝贵时间。现在,这种情况已不复存在,一切从简的这个优良传统流传至今。

\begin{quote}[Ken Arnold]
我认为简洁是Unix程序的核心风格。一旦程序的输出成为另一个程序的输入,就很容易把需要的数据挑出来。站在人的角度上来说——重要信息不应该混杂在冗长的程序内部行为信息中。如果显示的信息都是重要的,那就不用找了。
\end{quote}

设计良好的程序将用户的注意力视为有限的宝贵资源,只有在必要时才要求使用。

(我们将在第1 1章末尾进一步讨论缄默原则及其理由。)


\subsection{补救原则:出现异常时,马上退出并给出足量错误信息}
软件在发生错误的时候也应该与在正常操作的情况下一样,有透明的逻辑。最理想的情况当然是软件能够适应和应付非正常操作;而如果补救措施明明没有成功,却悄无声息地埋下崩溃的隐患,直到很久以后才显现出来,这就是最坏的一种情况。

因此,软件要尽可能从容地应付各种错误输入和自身的运行错误。但是,如果做不到这一点,就让程序尽可能以一种容易诊断错误的方式终止。

同时也请注意Postel的规定\footnote{Jonathan Postel是第一个互联网RFC系列标准的编纂者,也是互联网的主要架构者之一。网上有一个由Postel实验网络中心(Postel Center for Experimental Networking)维护的纪念网页:\href{http://www.postel.org/postel.html}{http://www.postel.org/postel.html}。}:“宽容地收,谨慎地发”。Postel谈的是网络服务程序,但是其含义可以广为适用。就算输入的数据很不规范,一个设计良好的程序也会尽量领会其中的意义,以尽量与别的程序协作:然后,要么响亮地倒塌,要么为工作链下一环的程序输出一个严谨干净正确的数据。

然而,也请注意这条警告:

\begin{quote}[Doug Mcllroy]
最初HTML文档推荐“宽容地接受数据”,结果因为每一种浏览器都只接受规范中一个不同的超集,使我们一直倍感无奈。要宽容的应该是规范而不是它们的解释工具。
\end{quote}

Mcllroy要求我们在设计时要考虑宽容性,而不是用过分纵容的实现来补救标准的不足。否则,正如他所指出的一样,一不留神你会死得很难看。

\subsection{经济原则:宁花机器一分,不花程序员一秒}
在Unix.早期的小型机时代,这一条观点还是相当激进的(那时机器要比现在慢得多也贵得多)。如今,随着技术的发展,开发公司和大多数用户(那些需要对核爆炸进行建模或处理三维电影动画的除外)都能够得到廉价的机器,所以这一准则的合理性就显然不用多说啦!

但不知何故,实践似乎还没完全跟上现实的步伐。如果我们在整个软件开发中很严格的遵循这条原则的话,大多数的应用场合都应该使用高一级的语言,如Perl、Tcl、Python、Java、Lisp,甚至shell——这些语言可以将程序员从自行管理内存的负担中解放出来(参见[Ravenbrook])。

这种做法在Unix世界中已经开始施行,尽管Unix之外的大多数软件商仍坚持采用旧Unix学派的C(或C++)编码方法。本书会在后面详细讨论这个策略及其利弊权衡。

另一个可以显著节约程序员时间的方法是:教会机器如何做更多低层次的编程工作,这就引出了……

\subsection{生成原则:邂免手工hack,尽量编写程序去生成程序}
众所周知,人类很不善于干辛苦的细节工作。因此,程序中的任何手工hacking都是滋生错误和延误的温床。程序规格越简单越抽象,设计者就越容易做对。由程序生成代码几乎(在各个层次)总是比手写代码廉价并且更值得信赖。

我们都知道确实如此(毕竟这就是为什么会有编译器、解释器的原因),但我们却常常不去考虑其潜在的含义。对于代码生成器来说,需要手写的重复而麻木的高级语言代码,与机器码一样是可以批量生产的。当代码生成器能够提升抽象度时——即当生成器的说明性语句要比生成码简单时,使用代码生成器会很合算,而生成代码后就根本无需再费力地去手工处理了。

在Unix传统中,人们大量使用代码生成器使易于出错的细节工作自动化。Parser/Lexer生成器就是其中的经典例子,而makefile生成器和GUI界面式的构建器(interface builder)则是新一代的例子。

(我们会在第9章讨论这些技术。)


\subsection{优化原则:雕琢前先得有原型,跑之前先学会走}
原型设计最基本的原则最初来自于Kernighan和Plauger所说的“90\%的功能现在能实现,比100\%的功能永远实现不了强”。做好原型设计可以帮助你避免为蝇头小利而投入过多的时间。

由于略微不同的一些原因,Donald Knuth(程序设计领域中屈指可数的经典著作之一《计算机程序设计艺术》的作者)广为传播普及了这样的观点:“过早优化是万恶之源”\footnote{完整的句子是这样的:“97\%的时间里,我们不应考虑蝇头小利的效率提升:过早优化是万恶之源”。Knuth自称这一观点来C. A. R. Hoare。}。他是对的。

还不知道瓶颈所在就匆忙进行优化,这可能是唯一一个比乱加功能更损害设计的错误。从畸形的代码到杂乱无章的数据布局,牺牲透明性和简洁性而片面追求速度、内存或者磁盘使用的后果随处可见。滋生无数bug,耗费以百万计的人时——这点芝麻大的好处,远不能抵消后续排错所付出的代价。

经常令人不安的是,过早的局部优化实际上会妨碍全局优化(从而降低整体性能)。在整体设计中可以带来更多效益的修改常常会受到一个过早局部优化的干扰,结果,出来的产品既性能低劣又代码过于复杂。

在Unix世界里,有一个非常明确的悠久传统(例证之一是Rob Pike以上的评论,另一个是Ken Thompson关于穷举法的格言):先制作原型,再精雕细琢。优化之前先确保能用。或者:先能走,再学跑。“极限编程”宗师Kent Beck从另一种不同的文化将这一点有效地扩展为:先求运行,再求正确,最后求快。

所有这些话的实质其实是一个意思:先给你的设计做个未优化的、运行缓慢、很耗内存但是正确的实现,然后进行系统地调整,寻找那些可以通过牺牲最小的局部简洁性而获得较大性能提升的地方。

\begin{quote}[Mike Lesk]
制作原型对于系统设计和优化同样重要——比起阅读一个冗长的规格说明,判断一个原型究竟是不是符合设想要容易得多。我记得Bellcore有一位开发经理,他在人们还没有谈论“快速原型化”和“敏捷开发”前好几年就反对所谓的“需求”文化。他从不提交冗长的规格说明,而是把一些shell脚本和awk代码结合在一起,使其基本能够完成所需要的任务,然后告诉客户派几个职员来使用这些原型,问他们是否喜欢。如果喜欢,他就会说“在多少多少个月之后,花多少多少的钱就可以获得一个商业版本”。他的估计往往很精确,但由于当时的文化,他还是输给了那些相信需求分析应该主导一切的同行。
\end{quote}

借助原型化找出哪些功能不必实现,有助于对性能进行优化;那些不用写的代码显然无需优化。目前,最强大的优化工具恐怕就是delete键了。

\begin{quote}[Ken Thompson]
我最有成效的一天就是扔掉了1000行代码。
\end{quote}

(我们将在第12章对相关内容进行深一步讨论。)

\subsection{多样原则:决不相信所谓“不二法门”的断言}
即使最出色的软件也常常会受限于设计者的想象力。没有人能聪明到把所有东西都最优化,也不可能预想到软件所有可能的用途。设计一个僵化、封闭、不愿与外界沟通的软件,简直就是一种病态的傲慢。

因此,对于软件设计和实现来说,Unix传统有一点很好,即从不相信任何所谓的“不二法门”。Unix奉行的是广泛采用多种语言、开放的可扩展系统和用户定制机制。

\subsection{扩展原则:设计着眼未来,未来总比预想快}
如果说相信别人所宣称的“不二法门”是不明智的话,那么坚信自己的设计是“不二法门”简直就是愚蠢了。决不要认为自己找到了最终答案。因此,要为数据格式和代码留下扩展的空间,否则,就会发现自己常常被原先的不明智选择捆住了手脚,因为你无法既要改变它们又要维持对原来的兼容性。

设计协议或是文件格式时,应使其具有充分的自描述性以便可以扩展。一直,总是,要么包含进一个版本号,要么采用独立、自描述的语句,按照可以随时插入新的、换掉旧的而不会搞乱格式读取代码的方法组织式。Unix经验告诉我们:稍微增加一点让数据部署具有自描述性的开销,就可以在无需破坏整体的情况下进行扩展,你的付出也就得到了成千倍的回报。

设计代码时,要有很好的组织,让将来的开发者增加新功能时无需拆毁或重建整个架构。当然这个原则并不是说你能随意增加根本用不上的功能,而是建议在编写代码时要考虑到将来的需要,使以后增加功能比较容易。程序接合部要灵活,在代码中加入“如果你需要……”的注释。有义务给之后使用和维护自己编写的代码的人做点好事。

也许将来就是你自己来维护代码,而在最近项目的压力之下你很可能把这些代码都遗忘了一半。所以,设计为将来着眼,节省的有可能就是自己的精力。

\section{Unix哲学之一言以蔽之}
所有的Unix哲学浓缩为一条铁律,那就是各地编程大师们奉为圭桌的“KISS”原则:
\begin{linefig}{KISS}
\label{fig:KISS}
\end{linefig}

Unix提供了一个应用KISS原则的良好环境。本书的剩余部分将帮助你学习如何应用这个原则。
kiss
\section{应用Unix哲学}
这些富有哲理的原则决不是模糊笼统的泛泛之谈。在Unix世界中,这些原则都直接来自于实践,并形成了具体的规定,我们已经在上文中阐述了一些。以下列举的只是部分内容:
\begin{itemize}
\item 只要可行,一切都应该做成与来源和目标无关的过滤器。
\item 数据流应尽可能文本化(这样可以使用标准工具来查看和过滤)。
\item 数据库部署和应用协议应尽可能文本化(让人可以阅读和编辑)。
\item 复杂的前端(用户界面)和后端应该泾渭分明。
\item 如果可能,用C编写前,先用解释性语言搭建原型。
\item 当且仅当只用一门语言编程会提高程序复杂度时,混用语言编程才比单一语言编程来得好。
\item 宽收严发(对接收的东西要包容,对输出的东西要严格)。
\item 过滤时,不需要丢弃的信息决不丢。
\item 小就是美。在确保完成任务的基础上,程序功能尽可能少。
\end{itemize}

在本书的余下部分,我们会看到这些Unix的设计原则及其衍生的设计规则被反复运用于实践。毫不奇怪,这些往往与其它传统中最优秀的软件工程实践思想不谋而合。\footnote{我在本书准备工作的后期发现一个值得注意的例子就是Butler Lampson的《Hints for computer System Design》[Lampson]。这本书不仅通过显然是独立发现的形式表达了一系列的Unix格言,甚至还使用了同样的结语来进行阐述。}

\section{态度也要紧}
看到该做的就去做——短期来看似乎是多做了,但从长期来看,这才是最佳捷径。如果不能确定什么是对的,那么就只做最少量的工作,确保任务完成就行,至少直到明白什么是对的。

要良好的运用Unix哲学,你就应该不断追求卓越。你必须相信,软件设计是一门技艺,值得你付出所有的智慧、创造力和激情。否则,你的视线就不会超越那些简单、老套的设计和实现:你就会在应该思考的时候急急忙忙跑去编程。你就会在该无情删繁就简的时候反而把问题复杂化——然后你还会反过来奇怪你的代码怎么会那么臃肿、那么难以调试。

要良好地运用Unix哲学,你应该珍惜你的时间决不浪费。一旦某人已经解决了某个问题,就直接拿来利用,不要让骄傲或偏见拽住你又去重做一遍。永远不要蛮干;要多用巧劲,省下力气到需要的时候再用,好钢用在刀刃上。善用工具,尽可能将一切都自动化。

软件设计和实现应该是\emph{一门充满快乐的艺术},一种高水平的游戏。如果这种态度对你来说听起来有些荒谬,或者令你隐约感到有些困窘,那么请停下来,想一想,问问自己是不是已经把什么给遗忘了。如果只是为了赚钱或是打发时间,你为什么要搞软件设计而不是别的什么呢?你肯定曾经也认为软件设计值得你付出激情……

要良好地运用Unix哲学,你需要具备(或者找回)这种态度。你需要用心。你需要去游戏。你需要乐于探索。

我们希望你能带着这种态度来阅读本书的其它部分。或者,至少,我们希望本书能帮助你重拾这种态度。


\chapter[历史——双流记]{历史——双流记\\[-1ex] \rule[1ex]{\textwidth}{1pt} \\[-2ex]History: A Tale of Two Cultures}
\begin{flushright}
\begin{notecard}[red!30]
忘记过去的人,注定要重蹈覆辙。

《理性生活》(1905年)

{\hfill —George Santayana}
\end{notecard}
\end{flushright}

前事不忘,后事之师。Unix的历史悠久且丰富多彩,许多内容仍然以坊间传说、猜想,以及(更常见的是)Unix程序员集体记忆中的战争创伤等形式鲜活地留存着。本章我们将通过回顾Unix的历史来阐明如今的Unix文化为什么会呈现当前这种状态。

\section{Unix的起源及历史,1969—1995}
小型实验原型系统的后继产品往往备受令人讨厌的“第二版效应”折磨。由于迫切希望把所有首次开发时遗漏的功能都添加进去,往往导致设计十分庞大、过于复杂。其实,还有一个因不常遇到而鲜为人知的“第三版效应”:有时候,在第二系统不堪自身重负而崩溃之后,有可能返璞归真,走上正道。

最初的Unix就是一个第三系统。Unix的祖辈是小而简单的兼容分时系统(CTSS,Compatible Time-Sharing System),也算曾经实施过的分时系统的第一代或者第二代了(取决于不同的定义,具体我们在此不作讨论)。Unix的父辈是颇具开拓性的Multics项目,该项目试图建立一个具备众多功能的“信息功用体/应用工具(information utilitv)”,能够很漂亮地支持大群用户对大型计算机的交互式分时使用。唉,Multics最后因不堪自身重负而崩溃了。但Unix却正是从它的废墟中破壳而出的。

\subsection{创世纪:1969-1971}
Unix于1969年诞生于贝尔实验室的计算机科学家Ken Thompson的头脑中。Thompson曾经是Multics项目的研究人员,饱受当时几乎作为铁律而到处应用的原始批量计算的困扰。然而在六十年代晚期,分时系统还是个新鲜玩意儿。计算机科学家John McCarthy(Lisp语言的发明者\footnote{McCarthy:1971年图灵奖获得者,主要贡献在人工智能方面;The concept was first described publicly in early 1957 by Bob Bemer as part of an article in \textit{Automatic Control Magazine}.  The first project to implement a timesharing system was initiated by John McCarthy. })几乎是在十年前才首次发表了分时系统的构想,而直到Unix诞生前七年的1962年才第一次真正部署使用,因此当时的分时系统尚处实验阶段,
像喜怒无常的野兽,性能极不稳定。

那个时代计算机硬件的原始程度,恐怕亲历者现在也很难以记清。那时最强大的机器所拥有的计算能力和内存还不如现在一个普通的手机。\footnote{Ken Thompson让我知道,如今手机的随机存储器(RAM)容量比PDP-7的随机存储器和磁盘存储量的总和还要多;那个年代所谓“大磁盘”的容量也不过1兆字节。}视频显示终端才刚刚起步,六年以后才得到广泛应用。最早分时系统的标准交互设备就是ASR-33电传打字机——一个又慢又响的设备,只能在大卷的黄色纸张上打印大写字母。Unix命令简洁、少说多作的传统正是从ASR-33开始的。

当贝尔实验室(Bell Labs)从Multics研究联盟中退出时,Ken Thompson带着从Multics激发的灵感——如何创建一个文件系统——留了下来。他甚至没能留下一台机器来玩自己编写的“星际旅行”,这是个科幻游戏——模拟驾驶一艘火箭在太阳系中邀游。Unix就在一台废弃的PDP-7小型机\footnote{网页\href{http://www.fags.org/fags/dec-fag/pdp8}{http://www.fags.org/fags/dec-fag/pdp8}上有关于PDP计算机的常见问题解答( FAQ),对在历史上除此(对Unix诞生所作贡献)之外默默无闻的PDP-7做了一些说明。}(图2-1)上问世了。这台PDP-7成为了“星际旅行”的游戏平台和Thompson关于操作系统设计思路的试验场。

Unix的完整起源故事可参见[Ritchie79],这是从Thompson第一个合作者Dennis Ritchie的角度讲述的。Dennis Ritchie后来以Unix的合作发明者和C语言的发明者而闻名于世。Dennis Ritchie、Doug Mcllroy和其他一些同事,已经习惯了Multics环境下的交互计算方式,不愿意放弃这一能力。Thompson的PDP-7操作系统给了他们一条救生绳。

\begin{fig}[2]{PDP-7}
\label{fig:PDP-7}
\end{fig}

Ritchie评述道:“我们希望保留的不仅仅是一个良好的编程环境,还包括一种能够形成伙伴关系的系统。经验告诉我们,远程访问(remote-access)和分时系统支持的公用计算,其本质不是用终端机代替打孔机来输入程序,而是鼓励频繁的交流。”计算机不应仅被视为一种逻辑设备而更应视为社群的立足点,这种观念深入人心。ARPANET(现今Internet的直系祖先)也发明于1969年。“伙伴关系”这一旋律将一直鸣奏在Unix的后继历史中。

Thompson和Ritchie“星际旅行”的实现引起了关注。起先,PDP-7的软件不得不在通用电气公司(GE)的大型机上交叉编译。Thompson和Ritchie为支持游戏开发而在PDP-7上编制的实用程序成了Unix的核心——虽然直到1970年才产生Unix这个名字。最初的缩写是“UNICS”(单路信息与计算服务,Uniplexed Information and Computing Service),Ritchie后来称之为“一个有点反叛Multics味道的双关语”,因为Multics是多路信息与计算服务(MULTIplexed Information and Computing Service)的英文缩写。

即使在最早期,PDP-7 Unix已经拥有现今Unix的诸多共性,提供的编程环境也比当时读卡式批处理大型机的环境要舒服得多。Unix几乎可以称得上第一个能让程序员直接坐在机器旁,飞快捕获稍纵即逝的灵感,并能一边编写一边测试的系统。Unix的整个发展进程中都能吸引那些不堪忍受其它操作系统局限性的程序员自愿为它进行开发,这也一直是Unix不断拓展其能力的模式。这种模式早在贝尔实验室时就已确立了。

Unix的轻装开发和方法上不拘一格的传统与生俱来。Multics是项庞大的工程,硬件开发出来前必须编写几千页的技术说明书,而第一份跑起来的Unix代码只是在三个人头脑风暴了一把,然后由Ken Thompson花了两天时间来实现罢了——还是在一台破烂机器上完成的,而那个机器本来只作为一台“真正”计算机的图形终端!

Unix的第一功,是1971年为贝尔实验室的专利部门进行“文字处理”的支持工作。首个Unix应用程序是nroff(l)文本格式化程序的前身。这个项目也让他们名正言顺地购买了一台功能强大得多的PDP-11小型机。万幸的是,当时管理层还未意识到Thompson和其同事所编写的字处理系统就快孵化出一个操作系统。贝尔实验室并没有开发操作系统的计划——AT\&{}T加入Multics联盟正是为了避免自行开发一个操作系统。不管怎样,整个系统还是取得了令人振奋的成功。Unix在贝尔实验室计算群落中的重要而永久地位由此确立,并且开创了Unix历史的下一个主旋律——与文档格式化、排版和通讯工具的紧密结合。1972年版的手册宣称装机量达10台。

Doug Mcllroy[Mcllroy91]后来这样描述这个时代:“外界的压力和纯粹出于对技艺的荣誉感,促使人们在有了更好更多的初步思路后,去重写或抛开已有的大量代码。从来没听说什么职业竞争和势力范围保护:好东西太多了,没有人需要把这些创新占为已有。”但是直到四分之一世纪后,人们才真正体会到他的话的含义。


\subsection{出埃及记:1971-1980}
最初的Unix用汇编语言写成,应用程序用汇编语言和解释型语言B混和编写。B语言的优点在于小巧,能在PDP-7上运行,但是作为系统编程语言还不够强大,所以Dennis Ritchie给它增加了数据类型和结构。C语言从1977年起自B语言进化而来;1973年,Thompson和Ritchie成功地用新语言重写了整个Unix系统。这是一个大胆的举动——那时为了最大程度地利用硬件性能,系统编程都通过汇编器来完成。与此同时,可移植操作系统的概念几乎鲜为人知。l979年,Ritchie终于可以这么写了:“很肯定,Unix的成功很大程度上源自其以高级语言作为表述方式所带来的可读性、可改性和可移植性”,虽然理想与现实此时尚有一线距离。

1974年在《美国计算机通信》(Communications of the ACM)上发表的一篇论文中[Ritchie—Thompson]第一次公开展示了Unix。文中作者描述了Unix前所未有的简洁设计,并报告了600多例Unix应用——这些都是安装在即便按照那个年代的标准,性能都算很低的机器上,但是(正如Ritchie和Thompson所写)“性能的局限不仅成就了经济性,而且鼓励了设计的简约”。

CACM论文发表后,全球各个研究实验室和大学都嚷着要亲身体验Unix。根据1958年为解决反托拉斯案例达成的和解协议,AT\&{}T(贝尔实验室的母公司)被禁止进入计算机相关的商业领域。所以,Unix不能够成为一种商品。实际上,根据和解协议的规定,贝尔实验室必须将非电话业务的技术许可给任何提出要求的人。Ken Thompson开始默默回应那些请求,将磁带和磁盘一包包地寄送出去——据传说,每包里都有一张字条,写着“love,ken”(爱你的,ken)。

这离个人机出现还有些年。那时候,不仅运行Unix所必须的硬件设备价格超出个人的承受范围,而且也没人敢奢望这种情况会在可预见的未来改变。因此,只有预算充足的大机构才用得起Unix机器:公司、高校、政府机构等。但是,对这些小型机的使用管制要比那些大型机少得多,因此,Unix的发展迅速笼罩了一层反传统文化的氛围。在上世纪70年代早期,最早搞Unix编程的通常都是头发蓬乱的嬉皮士和准嬉皮士们。摆弄操作系统的乐趣对他们来说不仅意味着可以在计算机科学的前沿上纵情挥洒,而且在于可以去推翻伴随“大计算”的所有技术假定和商业实践:卡式打孔机、COBOL、商务套装、IBM批处理大型机都成了看不上眼的过时事物;Unix黑客们沉浸在同时编织未来和编写系统的狂欢中。

那些日子的兴奋从Douglas Comer的话语中可见一斑:“许多大学都对Unix作出过贡献。多伦多大学计算机系发明了200dpi的打印机,绘图仪,并且开发了用打印机模拟照相排版机的软件;耶鲁大学的计算机专家和学生们改进了Unix的shell;普渡大学的电子工程系对Unix的性能作了重要改进,推出了支持大量用户的Unix版本:普渡大学还开发出了最早的Unix计算机网络之一;加州大学伯克利分校的学生开发了新shell和许多小型实用工具。1970年代后期贝尔实验室发布Unix V7版本时,很显然,该系统解决了许多部门的运算问题,也综合了许多高校的创意。最终诞生了一个更强大的系统。思想潮流开始了新一轮循环,从学术界流向工业实验室,然后又回到学术界,最后流向了不断增加的商业用户。”[Comer]
\begin{fig}[7]{1972年在PDP-11旁的Ken(坐)和Dennis(站)}
\label{fig:1972年在PDP-11旁的Ken(坐)和Dennis(站)}
\end{fig}

现代Unix程序员公认的第一个完全意义上的Unix是1979年发布的V7版本\footnote{}。第一代Unix用户群一年前就已形成。此时,Unix用于支撑贝尔系统(Bell System)所有操作[Hauben],并且传播到高校中,甚至远至澳大利亚——在那里,John Lions对V6版源码的注释[Lions]成了Unix内核的第一个正式文档。许多资深的Unix黑客仍然珍藏着一份拷贝。

\begin{quote}[Ken Arnold]
Lions 的书是地下出版界轰动一时的大事。由于侵犯版权等诸如此类的问题,该书不能在美国出版,所以大家就你拷给我、我拷给你。我也有一份拷贝,至少是第六手了。在那个时代,若没有Lions的书,你就当不成内核黑客。
\end{quote}

Unix产业也初露端倪。1978年,第一个Unix公司(the Santa Cruz Operation,SCO)成立,同年售出第一个商用C编译器( Whitesmiths)。1980年,西雅图一家还不起眼的软件公司——微软也加入到Unix游戏中,他们把AT\&{}T版本移植到微机上,取名为XENIX来销售。但是微软把Unix作为一个产品的热情并没有持续多久(尽管直到1990年左右,微软的大部分内部开发工作都用的是Unix)。

\subsection{TCP/IP和Unix内战:1980-1990}
在Unix的发展过程中,加州大学伯克利分校很早就成为唯一最重要的学术热点。伯克利分校早在1974年就开始了对Unix的研究,而Ken Thompson利用1975—1976的年休在此教学,更对Unix的研究注入了强劲活力。1977年,当时还默默无闻的伯克利毕业生Bill Joy管理的实验室发布了第一版BSD。到1980年,伯克利分校成了为这个Unix变种积极作贡献的高校子网的核心。有关伯克利Unix(包括\textit{vi}(1)编辑器)的创意和代码不断从伯克利反馈到贝尔实验室。

1980年,国防部高级研究计划局(DARPA,Defense Advanced Research Projects Agency)需要请人在Unix环境下的VAX机上实现全新的TCP/IP协议栈。那时,运行ARPANET的PDP-10已处耄耋之年,而数据设备公司(DEC)可能被迫放弃PDP-10以支持VAX的种种迹象也空穴来风。DARPA曾考虑和DEC公司签订实现TCP/IP的合同,但是因为担心DEC可能不太乐意改动他们的专有VAX/VMS操作系统[Libes—Ressler]而打消了这个念头。最后,DARPA选择了伯克利Unix作为平台——显然因为可以毫无阻碍地拿到它的源[Leonard]。

伯克利计算机科学研究组当时拥有天时地利,还有最强大的开发工具;而DARPA的合同无疑成为Unix历史上自诞生以来最关键的转折点。

在1983年TCP/IP实现随Berkeley4.2版发布之前,Unix对网络的支持一直是最薄弱的。早期的以太网实验不尽人意。贝尔实验室开发了一个难看但还能用的工具UUCP(Unix to Unix Copy Program),可在普通电话线上通过调制解调器来传送软件。\footnote{当时,如果调制解调器的速度能达到300波特时,UUCP跑得还是不错的。}UUCP可以在分布很广的杌器之间转发邮件,并且(在1981年Usenet发明后)支持Usenet——一个分布式的电子公告牌系统,允许用户把文本信息传播到任何拥有电话线和Unix系统的机器上。

尽管如此,已经意识到ARPANET光明前景的少数Unix用户感觉自己似乎陷在一潭死水中。没有FTP,没有telnet,只有限制重重的远程作业执行和慢得要死的连接。在TCP/IP诞生之前,Unix和Intemet文化尚未融合。Dennis Ritchie将计算机视为“鼓励密切交流”的工具这一设想还只是围绕单机分时系统或同一计算中心的学术社群,并没有扩展到自1970年代中期开始ARPA用户群逐渐形成的一个分布全美的“网络国家”。早期ARPANET的用户对着自己蹩脚的硬件时,也只能想:凑合着用Unix吧。

有了TCP/IP,一切都变了。ARPANET和Unix文化自边缘开始融合,这种发展最终使两者都免遭灭亡。不过,首先还得经过炼狱,起因是两个毫不相干的灾难:微软的兴起和AT\&{}T的拆分。

1981年,微软同IBM就新型IBM PC达成了历史性交易。比尔•盖茨从西雅图计算机产品公司(SCP,Seattle Computer Products)买下了QDOS(Quick and Dirty Operating System)。QDOS是SCP公司的Tim Paterson花六个星期凑出来的CP/M翻版。盖茨对Paterson和SCP公司隐瞒了同IBM的交易,以五万美元的价格买下了所有版权。后来,盖茨又说服了IBM公司允许微软将MS-DOS从硬件中剥离出来单独出售。接下来的十年中,盖茨利用这个非他所写的程序变成了超级亿万富翁,而比首笔交易更加精明的商业策略更是让微软垄断了桌面计算机市场。作为产品的XENIX很快就弃而不用了,最终卖给了SCO公司。

那时,没什么人能看出微软会多么成功(或有多大破坏性)。因为IBM PC-1硬件条件不足以来运行Unix,所以Unix人群几乎没注意这个产品(尽管,具有讽刺意味的是,DOS 2.0光芒能盖过CP/M,主要因为微软的合创者Paul Allen在DOS 2.0中融入了一些Unix的特征,包括子目录和管道等)。还有更有趣的事呢——比如说1982年SUN微系统公司的出世。

SUN微系统公司的创立者Bill Joy、Andreas Bechtolsheim和Vinod Khosla打算制造出一种内置网络功能的Unix梦幻机器。他们综合了斯坦福大学设计的硬件和伯克利分校开发的Unix,取得了辉煌的成功,开创了工作站产业。随着Sun公司越来越像传统商家而不再像一个无拘无束的新公司时,Unix大树上的这根分支源码来源的树枝逐渐枯萎,然而当时并没有人在意这一点。伯克利分校仍然随同源码一起销售BSD。一份System III源码许可证的官方价格为4万美元:贝尔实验室对非法流传贝尔Unix源码磁带的行为睁只眼闭只眼,各个高校也依然同贝尔实验室交换代码,看起来Sun公司对Unix的商业化似乎对它再好不过了。

C语言也在1982年有望被选为Unix世界外的系统编程语言。仅仅只用了五年左右的时间,C语言就几乎让机器码汇编语言完全失去了作用。到了九十年代早期,C和C++不仅统治了系统编程领域,而且成为应用编程的主流。到九十年代晚期,其他所有传统编译语言实际上都已经过时了。

1983年,在DEC公司取消PDP-10的后继机型的“木星”(Jupiter)开发计划后,运行Unix的VAX机器开始代之成为主流的互联网机器,直到被Sun工作站取代。到1985年,尽管DEC极力抵抗,还是有25\%{}左右的VAX用上了Unix。但是取消木星计划的长期效应并明显。更主要的是,MIT人工智能实验室以PDP-10为中心的黑客文化的消亡激发了Richard StallMan开始编制GNU——一个完全自由的Unix克隆版本。

到1983年,IBM PC可使用不下六种的Unix通用操作系统:uNETix、Venix、Coherent、QNX、Idris和运行在Snitek PC子板上的移植版本。但是System V和BSD版本仍然没有Unix移植——两个群体都悲观地认为8086微处理器不够强大,根本就没打算这么做。IBM PC上的这些Unix通用操作系统无一取得显著的商业成功,但表明了市场迫切需求运行Unix的低价硬件,而主要厂商并不供应。个人用户谁也买不起,更何况源码许可证上还挂着4万美元的价签呢。

1983年,美国司法部在针对AT\&{}T的第二起反托拉斯诉讼中获胜,并拆分了贝尔系统。这时Sun公司(及其效仿者!)已经取得了成功。这次判决将AT\&{}T从1958年的禁止将Unix产品化的和解协议中解脱了出来。AT\&{}T马上忙不迭地将Unix System V商业化——这一举措差点扼杀了Unix。

\begin{quote}[Ken Thompson]
确实如此。但他们的营销策略却将Unix推向了全球。
\end{quote}

大多数Unix支持者都认为AT\&{}T的拆分是个好消息。我们原以为,在拆分后的AT\&{}T、Sun公司及效仿Sun的小公司中,我们看到了一个健康的Unix产业核心——利用基于低廉的68000芯片的工作站——能够挑战并最终打破压迫在计算机行业上的垄断者——IBM。

那时,没有人意识到,Unix的产业化会破坏Unix源码的自由交流,而恰是后者滋养了Unix系统早期的活力。AT\&{}T只知道用保密从软件中获利,只会用集中控制模式开发商业产品,对源码散发严加防护。因为唯恐官司上身,非法交易的Unix源码也越来越乏人问津。来自高校的贡献随之开始枯竭。

更糟的是:刚刚进入Unix市场的几家大公司立马犯下了重大的战略性错误,其中之一就是试图通过产品差异化来寻求有利地位——这个策略导致了各种Unix接口的分歧,它抛弃了Unix的跨平台兼容性,造成了Unix市场分割。

另一个更微妙的错误就是以为个人计算机和微软不关Unix前景的事。Sun微系统公司未能意识到,日用品化的个人机最终会无可避免地动摇其工作站市场的根基。AT\&{}T公司为了成为计算机行业执牛耳者\footnote{古代歃血为盟,盟主执牛耳。},针对小型机和大型机采取了不同的策略,结果两个摊子都砸了。几家小公司试图在PC机上支持Unix,但都资金不足,仅专注于将产品出售给开发者和工程师,从未关注微软所瞄准的商用和家庭市场。

事实上,AT\&{}T拆分后的数年内,Unix社区却在忙着Unix大战的第一阶段——System V Unix和BSD Unix之间的内部争吵。争吵分成不同的层面,有些属于技术层面(socket对stream,BSD tty对System V termio),有些则属于文化层面。分歧可以大致划分为长发派和短发派。程序员和技术人员往往与伯克利和BSD站在一边,而以商业为目标的人则倾向AT\&{}T和System V。长发派,重唱着十年前Unix早期的主题,喜欢自我标榜为企业帝国的叛逆者,比如一家小公司贴的海报那样,上面画着一个标着“BSD”的X翼星际战机快速飞离巨大的AT\&{}T死星,后者在熊熊烈火中粉身碎骨。就这样,罗马在燃烧,而我们还在拉小提琴。

但是,AT\&{}T拆分当年发生的另一件事对Unix产生了更深远的影响。程序员兼语言学家Larry Wall发明了patch (1)实用程序。Patch程序是一个将diff(1)生成的修改记录(changebar)写入基础文件的简单工具,这意味着Unix开发人员之间可通过传送补丁——代码的渐增变化——进行协作,而不必传送整个代码文件。这一点非常重要,不仅因为补丁要比整个文件小,更因为即使基础文件和补丁制作者拿到的版本之间变化很大,仍然可以很干净地应用补丁。运用这个工具,基于共有源码库的开发流可以分开、并行、最后合拢。patch程序比其它任何单一工具都更能促进Internet上的协作开发——这种方式在1990年后让Unix获得新生。

1985年,Intel的第一枚386芯片下线。它具有用平面地址空间寻址4G内存的能力。笨拙的8086和286的段寻址旋即废弃。这是条大新闻,因为这意味着占据主导地位的Intel家族终于有了一款无需作出痛苦妥协就能运行Unix的微处理器。对Sun公司和其它工作站厂商来说,这真是不祥之兆,可惜它们并未觉察到。

同样在1985年,Richard Stallman发表了GNU宣言(the GNU manifesto)[Stallman],并发起了自由软件基金会(Free Software Foundation)。没有谁把他和他的GNU当回事,结果证明这是个大错误。同年,在一项与此不相干的开发行动中,X window系统的创始人发布了X window的源码,而无需版税、约束和授权。这项决策的直接结果就是X window成为不同Unix厂商之间合作的安全中立区,并挫败了专属的竞争对手,成为了Unix的图形引擎。

以调解System V和Berkeley API为目标的严肃的标准化工作始于1983年,产生了/usr/group标准。随之为1985年IEEE支持的POSIX标准。这些标准描述了BSD和SVR3(System V Release 3)调用的交集,综合了伯克利出色的信号处理和作业控制,以及SVR3的终端控制。所有后续的Unix标准其核心都加入了POSIX,后续开发的各种Unix版本也严格遵循这个标准。后来的现代Unix核心API唯一主要的补充就是BSD套接字。

1986年,前面提到的发明patch(1)的Larry Wall开始开发Perl语言,后者是最先也最广泛使用的开源脚本语言。1987年年初,GNU C编译器的第一版问世,到1987年年底,GNU工具包的核心部分——编辑器、编译器、调试器以及其它基本的开发工具——都已就位。同时,X window系统也开始在相对低廉的工作站上露面了。这些因素都为20世纪90年代的Unix开源发展提供了利器。

同样是在1986年,PC技术挣脱了IBM的掌控。IBM仍然试图在产品系列上维持高价格性能比,更青睐高利润的大型机市场,所以在新的PS/2系列产品上拒用386而选择了较弱的286。PS/2系列为了杜绝仿冒而围绕一个专有总线结构进行设计,结果成了代价高昂的大败笔\footnote{PS/2毕竟还是在后来的PC机上留了一记——使鼠标成为标准外设,这也是为什么你机箱后面的鼠标接口会叫做“PS/2端口”}。最积极进取的效仿者康柏(Compaq),发布了第一款386机器,靠这张牌打败了IBM。虽然主频只有16MHz,但是386也算能跑起来Unix了。这是第一款可以叫Unix机器的PC。

这会儿已经能够想象Stallman的GNU项目可以和386机器配合而制造出Unix工作站,它比当时任何方案都要便宜一个数量级。奇怪的是,没人想到这步棋。来自小型机和工作站世界的大多数Unix裎序员,依然鄙视廉价的80x86芯片,而钟情基于68000的高雅设计。尽管许多程序员都为GNU工程做出了贡献,但在Unix人群中,这个GNU项目仍然被视为一个唐吉诃德式的狂想,短期内还无法实用。

Unix社区从未丢弃叛逆气质。但是回头看来,我们几乎和IBM或者AT\&{}T一样,对迫近我们的未来毫无所知。即使是数年前就开始对专有软件开展精神讨伐的Richard Stallman也未能真正理解Unix的产品化会对其所在社区有多大破坏力;他关心的是更抽象的长期论题。其余的人还一直企盼企业规则能有些精明的变化,从此市场分割、营销不利和战略漂忽不定等问题将不复存在,从而救赎回Unix拆分之前的世界。但是祸不单
行。

很多人都知道Ken Olsen(DEC的CEO)在1988年将Unix描绘成“蛇油”(骗人的万灵油)。从1982年起,DEC就一直在销售其开发的用于PDP-11的Unix变种,但真正希望的却是将业务回到自己专有的VMS操作系统上来。DEC和其它小型机厂商碰到了大麻烦,陷入Sun微系统公司和其它工作站厂商功能强劲、价格低廉的机器重重包围中。这些工作站大多运行的是Unix。

但是Unix产业自身的问题却更为严峻。1988年,AT\&{}T持有了Sun公司20\%{}的股份。作为Unix市场领军的这两家公司,终于开始清醒地认识到PC,IBM和微软构成的威胁,也终于认识到过去五年的争斗令他们几无所获。AT\&{}T和Sun的联盟以及以POSIX为核心的技术标准的发展,最终弥合了System V和BSD Unix之间的裂痕。但是,当二线商家(IBM、DEC、HP等)创建开放软件基金会(Open Software Foundation)并结成盟友和以“Unix国际”为代表的“AT\&{}T/Sun轴心”对抗时,Unix内战的第二阶段开始了。更多回合的Unix与Unix三家的战斗随而爆发。

这段时间中,微软从家庭和小型商用市场赚了数十亿美元的钱,而争战不休的Unix各方却从未决意涉足这些市场。1990年,Windows 3.0——来自微软总部Redmond发布的第一个成功的图形操作系统——巩固了微软的统治地位,为微软在九十年代荡平并最终垄断桌面应用市场创造了条件。

1989年到1993年是Unix的中世纪。当时,似乎Unix社群所有的梦想都破灭了。相互争斗的战事已使专有Unix产业衰落得像个吵闹的肉店,无力振起挑战微软的雄心。大多数Unix编程者青睐的优雅的Motorola芯片也已经输给了Intel丑陋但廉价的处理器。GNU项目没能开发出自由的Unix内核,尽管从1985年GNU就不断作出此承诺,其信用令人质疑。PC技术被无情地商业化了。1970年代的Unix黑客先锋们人近中年,步履开始蹒跚。硬件便宜了,但Unix还是太贵。我们幡然醒悟:过去的IBM垄断让位于现在的微软垄断,而微软设计糟糕的软件像浊流一样,围着我们越涨越高。

\subsection{反击帝国:1991-1995}
1990年,William Jolitz把BSD移植到了386机器上,这是黑暗中的第一缕曙光。1991年起一系列杂志文章对此进行了报道。向386移植BSD的移植之所以可能,是由于伯克利黑客Keith Bostic一定程度上受Stallman影响,早在1988年他就开始努力从BSD码中清除AT\&{}T专有代码。但是,Jolitz在1991年年底退出386-BSD项目,并毁掉了自己的成果,使该项目受到严重打击。对于此事的起因众说纷纭,不过公认的一点是Jolitz希望
将其代码以源码形式无限制地发布,因此当项目的企业赞助商选择了更专有的授权模式时,他火了。

1991年8月,当时默默无闻的芬兰大学生Linus Torvalds宣布了Linux项目。据称Torvalds最主要的激励是学校里用的Sun Unix太贵了。Torvalds还说,要是早知道有BSD项目,他就会加入BSD组而不是自己做一个。但是386BSD直到1992年早些时候才下线,而此时Linux第一版已经发布好几个月了。

不回头看,人们无法发现这两个项目的重要性。那时,即使在Intemet黑客文化内部也没有多少人关注它们,遑论更广大的Unix社区。当时Unix社区还在盯着比PC机性能更强大的机器,仍试图把Unix的特有品质与软件业的常规专有模式扯到一起。

又过了两年,经历了1993—1994年的互联网大爆炸,Linux和开源BSD的真正重要性才为整个Unix世界所了解。但不幸的是对BSD支持者来说,AT\&{}T对BSDI(赞助Jolitz移植的创业公司)的诉讼消耗了大量时间,使一些关键的Berkeley开发者转向了Linux。

\begin{quote}[Marshall Kirk McKusick]
代码抄袭和窃取商业秘密的行为从未被证实。他们花了两年的时间也没找到确凿的侵权代码。要不是Novell从AT\&{}T买下了USL、并达成协议,这场官司还会拖得更久。结果是从发布包中18000个组成文件中删掉了三个,对其它文件作了一些小修改。另外,伯克利大学也同意为约70个文件增加USL版权,但同时约定这些文件仍然可以自由重新分发。
\end{quote}

这项和解为开创了从专有控制下获取一个自由而完整可用的Unix的先河,但对BSD自身的影响却是灾难性的。当伯克利的计算机科学研究组于1992—1994年间被关闭时,情况更糟了;随后,BSD社区内的派系斗争又将BSD开发分割成三个方向间的竞争。结果,BSD这一脉在关键时刻落后于Linux,Unix社区的领先地位拱手让人。

与此前各种版本的Unix开发相比,Linux和BSD的开发相当不同。它们植根于互联网,依赖分布式开发和Larry Wall的patch(1)工具,通过email和Usenet新闻组招募开发者。因此,当互联网服务提供商(ISP)的业务于1993年因通信技术的变革和Internet骨干网的私有化(超出Unix历史范围,不述)而扩展时,Linux和BSD也得到了巨大的推动力。但对廉价互联网的需求却是由另一件事创造的:1991年万维网(WWW)的发明。万维网是互联网中的“杀手级应用”,图形用户界面技术对大量的非技术型最终用户有着不可抗拒的魅力。

互联网的大规模市场推广,既增加了潜在开发者的数量,又降低了分布式开发的处理成本,这些影响可从XFree86之类的项目上看出。XFree86利用Internet为中心的模式建立了一个比官方X联盟更有效的开发组织。1992年诞生的第一版XFree86赋予了Linux和BSD 一直缺乏的图形用户界面引擎。下个十年里,XFree86将领导X的开发,X联盟越来越多的行为都是把源自XFree86社区的创新汇聚回X联盟产业赞助者中。

到1993年年末,Linux已经具备了Internet能力和X系统。整套GNU工具包从一开始就内置其中,以提供高质量的开发工具。除了GNU工具,Linux好像一个魅力聚宝盆,囊括了二十年来分散在十几种专有Unix平台上的开源软件之精华。尽管正式说来Linux内核还是测试版(0.99的水平),但稳定性已经让人刮目相看。Linux上软件之多、质量之高,已经达到一个产品级操作系统的水准。

在旧学派的Unix开发者中,一部分脑筋活络的人开始注意到,做了多年的平价Unix之梦从一个意想不到的方向悄然成真。它既不是来自AT\&{}T,也不是来自Sun,或者任何一个传统厂商,也不是出于学术界有组织的工作成果。它就这样从Internet的石头缝中跳了出来,浑然天成,以令人惊奇的方式重新规划拼装了Unix的传统元素。

另一方面,商业运作继续进行。1992年AT\&{}T抛售了其手中Sun公司的股份,然后在1993年把Unix系统实验室(Unix Systems Laboratories)卖给了Novell;Novell又于1994年将Unix商标转手给X/Open标准组(X/open standards group);同年AT\&{}T和Novell
加入了OSF(开放软件基金会),Unix之战尘埃落定。1995年,SCO从Novell手中买下了UnixWare(以及最初Unix源码的权利)。1996年,X/Open和OSF合并,创立了一个大型Unix标准组。

但是,传统Unix厂商和他们战后的烂摊子看来确是越来越无关紧要了。Unix社区的动作和精力都在转向Linux、BSD及开源开发者。1998年,IBM、Intel和SCO宣布启动蒙特里项目(the Moterey project),最后一次努力试图将所有现存的专有Unix整合成一个大系统,开发者和业内媒体坐看笑话。原地兜了三年的圈之后,此项目在2001年戛然而止。

2000年SCO把UnixWare和原创的Unix源码包出售给了Caldera——一家Linux发行商,整个产业变迁终告结束。但1995年后,Unix的故事就成了开源运动的故事。故事还有一半没讲呢,我们要回到1961年,从互联网黑客文化的起源开始讲起。

\section{黑客的起源和历史:1961-1995}
Unix传统是一种隐性的文化,不只是一书袋的技术窍门。这种传统传达着一个有关美和优秀设计的价值体系;里面有它的江湖和侠客。与Unix传统的历史交织在一起的则是另一种隐性文化,一种更难归别的文化。它也有自己的价值体系、江湖和侠客,部分与Unix文化交迭,部分源于它处。人们老是把这种文化称为“黑客文化”,从1998年起,这种文化已经很大程度上和计算机行业出版界所称的“开源运动”重合了。

Unix传统、黑客文化以及开源运动间的关系微妙而复杂。三种隐性文化背后往往是同一群人,然而其间的关系并未因此而简化。但是,从1990年以来,Unix的故事很大程度上成了开源世界的黑客们改变规则、从保守的专有Unix厂商手中夺取主动权的故事。因此,今天Unix身后的历史,有一半就是黑客的历史。


\subsection{游戏在校园的林间:1961-1980}
黑客文化的根源可以追溯到1961年,这一年MIT购买了第一台PDP-1小型机。PDP-1是最早的一种交互式计算机,并且(不象其它机器)在那时并非天价,所以没有对它的使用做太多限时规定。因此PDP-1吸引了一帮好奇的学生。他们来自技术模型铁路俱乐部(TMRC,Tech Model Railroad Club),带着一种好玩的心态摆弄这台设备。《黑客:计算机革命中的英雄》(Hackers:  Heroes of the Computer Revolution)[Levy]一书对这个俱乐部的早期情况作了有趣的描写。他们最著名的成就是“太空大战(SPACEWAR)”——一款宇宙飞船决斗游戏,灵感大概来自Lensman的星际故事《E. E. ‘Doc’Smith》。\footnote{“SPACEWAR”和Ken Thompson的“Space Travel”毫不相干,除了都吸引科幻迷的共同点。}

TMRC来实验的几个人后来是成了MIT人工智能实验室的核心成员,而这个实验室在六七十年代成为前沿计算机科学的世界级中心之一。这些人也把TMRC的行话和内部笑话带了进来,包括一种精巧(但无害)的恶作剧传统“hacks”。人工智能实验室的程序员应该是第一群自称“hacker"的人。

1969年后,MIT AI实验室和斯坦福、Bolt Beranek \& Newman公司(BBN)、卡内基-梅隆大学(CMU:Carnegie-Mellon University)以及其它顶级计算机科学研究实验室通过早期的ARPANET联上了网。研究人员和学生第一次尝到了快速网络联接消除了地域限制的甜头,通过网络,远方的人通常比与身边少有来往的同事更容易合作和建立友谊。

实验性的ARPANET网上到处都是软件、点子、行话和大量幽默。一种类似共享文化的东西开始成形,其中最早、最持久的典型产物之一就是“术语文件(Jargon File)”,列举了1973年发源于斯坦福、1976年后在MIT经过多次修订的共享行内名词,并一路收集了CMU、耶鲁和其它ARPANET站点的行话。

从技术性而言,早期的黑客文化大都基于PDP-10小型机。下列已经成为历史的操作系统他们都用过:TOPS-10、TOPS-20、Multics、ITS和SAIL。他们利用汇编器和各种Lisp方言编程。PDP-10的黑客们后来接手运行ARIPANET,因为别人不愿意干这件事。后来,他们成了互联网工程工作组(IETF,Intemet Engineering Task Force)的创建骨干,并作为创始人,开创了通过RFC(Requests For Comment)进行标准化的传统。

从社会性而言,他们年轻,天资过人,几乎全是男性,献身编程达到痴迷的地步,决不墨守成规——后来被人们唤做“极客(geek)”。他们往往也是头发蓬松的嬉皮士和准嬉皮士。他们有远见,把计算机看作构建社区的工具。他们读Robert Heinlein和J. R. R. Tolkien的书,参加复古协会(Society for Creative Anachronism),双关语说起来没完。抛开这些怪癖(也许正由于这些原因),他们中的许多人都跻身世界上最聪明的程序员之列。

他们并不是Unix程序员。早期的Unix社群成员大部分来自院校、政府和商业研究实验室的同一帮“极客”,但是两种文化有明显的分野。其中之一就是我们前面已经谈到的早期Unix孱弱的网络能力。直到1980年后,才真正出现了基于Unix的ARPANET网络连接,之前一个人同时涉足两个阵营的情况并不多见。

协作式开发和源码共享是Unix程序员的法宝。然而,对于早期的ARRPNET黑客,这还不只是一种策略,它更像一种公众信仰,部分起源于“要么发表要么掉”的学术规则,并且(更极端地)几乎发展成为关于网络思想社区的夏尔丹式理想主义( Chardinist idealism)。这些黑客中最著名的Richard M. Stallman后来成了严守教义的苦行僧。

\subsection{互联网大触合与自由软件运动:1981-1991}
1983年后,随着BSD植入了TCP/IP,Unix文化和ARPANET文化开始融合。既然两种文化都由同一类人(实际上,就有少数几位很有影响的人同属两种文化阵营)构成,一旦沟通环节到位,两种文化的融合就水到渠成。ARPANET黑客学到了C语言,用起了管道、过滤器和shell之类的行话。Unix程序员学到了TCP/IP,也开始互称“黑客”。1983年,木星项目的取消虽然葬送了PDP-10的前途,却加速了两种文化融合的进程。到1987年,这两种文化已经完全融合在一起,绝大多数黑客都用C编程,自如地使用源于25年前技术模型铁路俱乐部(TMRC)创造的行话。

在1979年,我和Unix文化、ARPANET文化都有密切联系,当时这种情况还很少见。到1985年,这就已经不稀奇了。1991年我将以前的ARPANET“术语文件”( Jargon File)扩展成《新黑客词典》(New Hacker's Dictionary)[Raymond96],此时两种文化实际上已经融为一体。把生于ARPANET、长于Usenet的“术语文件”作为这次融合的标志真是再恰当不过了。)

但是TCP/IP联网和行话并不是后1980黑客文化从其ARPAN-\\ET根源继承的全部东西,还有Richard M. StallMan和他的精神革命。

Richard M. Stallman(他的登陆名RMS更为人熟知)早在1970年代晚期就已经证明他是当时最有能力的程序员之一。Emacs编辑器就是他众多发明中的一项。对RMS来说,1983年木星(Jupiter)项目的取消仅仅只是宣告了麻省理工学院人工智能实验室(MIT AI Lab)文化的最终解体。其实早在几年前随着实验室众多最优秀的成员纷纷离去,帮忙管理与之竞争的Lisp机器时,这种解体就已经开始了。RMS觉得自己被逐出了黑客的伊甸园,他把这一切都归咎于专有软件。

1983年,Stallman创建了GNU项目,致力于编一个完全自由的操作系统。尽管Stallman既不是、也从来没有成为一个Unix程序员,但在后1980的大环境下,实现一个仿Unix操作系统成了他追求的明确战略目标。RMS早期的捐助者大都是新踏入Unix土地的老牌ARPANET黑客,他们对代码共享的使命感甚至比那些有更多Unix背景的人强烈。

1985年,RMS发表了GNU宣言(the GNU Manifesto)。在宣言中,他有意从1980年之前的ARPANET黑客文化价值中创造出一种意识形态——包括前所未见的政治伦理主张、自成体系而极具特色的论述以及激进的改革计划。RMS的目标是将后1980的松散黑客社群变成一台有组织的社会化机器以达到一个单纯的革命目标。也许他未意识到,他的言行与当年卡尔•马克思号召产业无产阶级反抗工作的努力如出一辙。\footnote{请注意作者的立场是偏向Torvalds的,所以这里类比马克思多少有点暗黑的意思。读者请自己判断。}

RMS宣言引发的争论至今仍存于黑客文化中。他的纲要远不止于维护一个代码库,已经暗含了废除软件知识产权主张的精髓。为了追求这个目标,RMS将“自由软件(free software)”这一术语大众化,这是将整个黑客文化的产品进行标识的首次尝试。他撰写了“通用公共许可证(General Public License,GPL)”,后者成了一个既充满号召力又颇具争议的焦点,  具体原因我们将在16章研讨。读者可以去GNU站点\href{http://www.gnu.org}{http://www.gnu.org}了解RMS立场及自由软件基金会(Free Software Foundation)的更多情况。

“自由软件(free  software)”这个术语既是一种描述,也是为黑客进行文化标识的一个尝试。从某个层次上说,这是相当成功的。在RMS之前,黑客文化中的人们彼此当作“同路人”,说着同样的行话,但没人费神去争辩“黑客”是什么或者应该是什么。在他之后,黑客文化更加有自我意识。价值冲突(即使反对RMS的人也经常以他的方式说话)成为辩论中的常见特点。RMS,这个魅力超凡又颇具争议的人物本身已经成为了一个文化英雄,因此到2000年时,人们已经很难将他本人和他的传奇区分开来。《自由中的自由》(\textit{Free as in Freedom})[Williams]对他的刻画非常精彩。

RMS的论点甚至影响了那些对其理论持怀疑态度的黑客的行为。1987年,他说服了BSD Unix的管理者,让他们相信,将AT\&{}T的专有代码清除出去、发布一个无限制的版本是个好主意。然而,尽管他花了不下十五年的苦功夫,后1980黑客文化却从未统一在他的理想之下。

其他黑客,更多出于实用角度而非思想观念的原因,重新认识到了开放式协作开发的价值。在八十年代后期离Richard Stallman位于MIT九楼办公室不远的几座楼里,X开发组搞得红红火火。这个项目由一些Unix厂商资助,这些厂商此前一直为X window系统的控制权和知识产权争论不休,结果发现还不如向所有人自由开放。1987至1988年间,X的开发预示了一个极为庞大的分布式社群,后者将在五年后重新定义Unix的前沿方向。

\begin{quote}[Keith Packard]
X是首批由服务于全球各地不同组织的许多个人以团队形式开发的大规模开源项目之一。电子邮件使创意得以在这个群体中快速传播,问题由此得以快速解决,而开发者可以人尽其才。软件更新可以在数小时之内发送到位,使得每个节点在整个开发过程中步调一致。网络改变了软件的开发模式。
\end{quote}

X开发者们不替GNU总计划帮腔,但也不唱反调。1995年以前,GNU计划最强烈的反对者是BSD开发者。BSD开发者觉得自己编写自由发布和修改软件的年头比RMS
宜言长得多,坚决抵制GNU自称的在历史性和思想性上的首创。他们尤其反对GPL的传染性或“病毒般”的特性,坚持BSD许可证比GPL“更自由”,因为BSD对代码重用的限制要比GPL少。

尽管RMS的自由软件基金会已开发了整套软件工具包的绝大部分,但是未能开发出核心部件,因此形势对RMS仍然不利。GNU项目创立十年了,GNU内核仍是空中楼阁。尽管Emacs和GCC之类的单个工具被证明非常有用,但是没有内核的GNU既不能对专有Unix的霸权构成威胁,又不能有效抵抗日渐严重的微软垄断。

1995年后,关于RMS思想体系的争论稍稍发生了变化。反对者的观点跟Linus Torvalds和本书作者越来越近。


\subsection{Linux和实用主义者的应对:1991-1998}
即使在HURD(GNU内核)计划停转之时,新的希望还是出现了。1990年代早期,价廉性优的PC机加上方便快捷的互联网,对寻找机会挑战自我的新生代年轻程序员是极大的诱惑。自由软件基金会编写的用户软件工具包铺平了一条摆脱高成本专有软件开发工具的前进道路。意识服从经济,而不是领导:一些新手加入了RMS的革命运动,高举GPL大旗,另一些人则更认同整体意义上的Unix传统,加入了反对GPL的阵营,但其他大部分人置身事外,一心编码。

Linus Torvalds巧妙地跨越了GPL和反GPL的派别之争。他利用GNU工具包搭起了自创的Linux内核,用GPL的传染性质保护它,但拒绝认同RMS许可协议反映的思
想体系计划。Torvalds明确表示他认为自由软件通常更好,但他偶尔也用专有软件。即使在他自己的事业中,他也拒绝成为狂热分子。这一点极大地吸引了大多数黑客,他们虽然早就反感RMS的言辞,但他们的怀疑论一直缺个有影响力或者令人信服的代言人。

Torvalds令人愉快的实用主义及灵活而低调的行事风格,促使黑客文化在1993至1997年间取得了一连串令人惊奇的胜利,不仅仅在技术上的成功,还让围绕Linux操作系统的发行、服务和支持产业有了坚实的开端。结果,他的名望和影响也一飞冲天。Torvalds成为了互联网时代的英雄:到1995年为止,他只用了四年时间就在整个黑客文化界声名显赫,而RMS为此花了十五年,而且他还远远超过了Stallman向外界贩卖“自由软件”的记录。与Torvalds相比,RMS的言辞渐渐显得既刺耳又无力。

1991至1995年间,Linux从概念型的0.1版本内核原型,发展成为能够在性能和特性上均堪媲美专有Unix的操作系统,并且在连续正常工作时间等重要统计数据上打败了这些Unix中的绝大部分。1995年,Linux找到了自己的杀手级应用——开源的web服务器Apache。就像Linux,Apache出众地稳定和高效。很快,运行Apache的Linux机器成了全球ISP平台的首选。约60\%{}的网站选用Apache,\footnote{当月和以往的web服务器占有量数据可从Netcraft的web服务器月度调查(monthly Netcraft Web Server Survey)中获得。}轻松击败了另两个主要的专有型竞争对手。

Torvalds未作的一件事就是提供新的思想体系——一套关于黑客行为的新理论基础或繁衍神话,以及一套吸引黑客文化圈内圈外人士的正面论述,以消弭RMS对知识产权的不友善。1997年,当我试图探寻为什么Linux开发没有在几年前崩溃时,我偶然地填补了这个空白。我所发表论文[Raymond01]的技术结论归纳在本书第19章。对于这段历史梗概,只要看看第一条结论核心规则的冲击就够了:“如果有足够多眼睛的关注,所有的bug都无处藏身”。

这段观察暗含了过去四分之一世纪在黑客文化中从未有人敢相信的东西:用这种方法做出的软件,不仅比我们专有竞争者的东西更优雅,而且更可靠、更好用。这个结果出乎意料地向“自由软件”的论述发起了直接挑战,而Torvalds本人从未有意于此。对于大多数黑客和几乎所有的非黑客而言,“用自由软件是因为它运行得更好”轻而易举地盖过了“用自由软件是因为所有软件都该是自由的”。

在我的论文中关于“大教堂”(集权、封闭、受控、保密)和“集市”(分权、公开、精细的同僚复审)两种开发模式的对比成为了新思潮的中心思想。从某种重要意义上来说,这仅仅是对Unix在拆分前根源的回归——Mcllroy在1991年阐述了同侪压力如何对1970年代早期Unix的发展产生了积极影响、Dennis Rithchie在1979年对伙伴关系的反思,这是此两者的延续,并与早期ARPANET同侪评审的学术传统及其分布式精神社区的理想主义相得益彰。

1998年初,这种新思潮促使网景公司(Netscape Communications)公布了其Mozilla浏览器的源码。媒体对此事件的关注促成了Linux在华尔街的上市,推动了1999-2001年间科技股的繁荣。事实证明,此事无论对黑客文化的历史还是对Unix的历史都是一个转折点。

\section{开源运动:1998年及之后}
到1998年Mozilla源码公布的时候,黑客社区其实算是一个众多派系或部落的松散集合,包括了Richard Stallman的自由软件运动(Free Software Movement)、Linux社区、Perl社区、Apache社区、BSD社区、X开发者、互联网工程工作组(IETF),还有至少一打以上的其它组织。这些派系相互交叠,一个开发者很可能同时隶属两个或更多组织。

一个部落的凝聚力可能来自他们维护的代码库,或是一个或多个有着超凡影响力的领导者,或是一门语言、一个开发工具,或是一个特定的软件许可,或是一种技术标准,或是基础结构某个部分的管理组织。各派系既论资排辈,也追逐当前的市场份额及认知度。因此,资格最老的大概要算IETF,其历史可以连续追根溯源到1969年ARPANET的发源期;BSD社区尽管市场安装数量要比Linux少得多,但是因为其传统可连续追溯到1970年代末,所以还是拥有相当高的声望;可追溯到1980年代初的Stallman的自由软件运动,无论从历史贡献,还是从作为几个最常用的软件工具维护者的角度,都足以令其跻身高级部落行列。

1995年后,Linux扮演了一个特殊的角色:既是社区内多数软件的统一平台,又是黑客中最被认可的品牌。Linux社区随之显现了兼并其它亚部落的倾向——甚至包括争取并吸纳一些专有Unix相关的黑客派系。整个黑客文化开始凝聚在一个共同目标周围:尽力推动Linux和集市开发模式向前发展。

因为后1980黑客文化已经深深植根于Unix,新目标成了Unix传统争取胜利的不成文纲要。黑客社区的许多高级领导人也都是Unix的老前辈,八十年代分拆后内战的伤痕犹在,他们将Linux作为实现Unix早期叛逆梦想的最后和最大的希望,而汇聚在Linux旗下。

Mozilla源码的公布使各方意见更为集中。1998年3月,为了深入研究共同目标和策略,召开了一次空前的社团重要领导人峰会,与会者几乎代表了所有的主要部落。这次会议为所有派系的共同开发方式确立了一个新标记——开源。

六个月之内,黑客社区中几乎所有部落都接受了用“开源”的新旗帜。IETF和BSD开发组这种老团体更是把他们从过去到现在所作的东西都追加上了这一标记。实际上,到2000年,黑客文化不仅让“开源”这个辞令统一了当前实践和未来计划,而且也对自己的历史重新有了鲜活的认识。

Netscape开放源码的宣告和Linux的新近崛起产生的激励效应远远超越了Unix社区和黑客文化。从1995年开始,所有阻拦在微软Windows滚滚巨轮前的各种平台(MacOS;Amiga;OS/2;DOS;CP/M;较弱小的专有Unix;各类大型机小型机和过时的微型机操作系统)的开发者团结到了Sun微系统公司的Java语言周围。许多不满微软的Windows开发者也加入了Java阵营,希望至少能够和微软保持名义上的独立。但是Sun公司运作Java的几个层面都(我们将在第14章予以讨论)既拙劣又排斥他人。许多Java开发者喜欢上了开源运动中的新生事物,于是就像此前跟随Netscape加入Java一样,又跟随它加入了Linux和开源运动。

开源行动的积极分子热烈欢迎来自各个领域的移民潮。老一辈Unix人也开始认同新移民的梦想:不能只是被动忍受微软的垄断,而是要从微软手中夺回关键市场。开源社区成员们合力争取主流世界的认同,开始乐于同大公司结盟——这些公司,随着微软的绳索勒得越来越紧,也越来越害怕对自己的业务失去控制。

唯一的例外是Richard Stallman和自由软件运动。“开源”明显要用一个意识形态中性的公众标签来取代Stallman钟爱的“自由软件”。新标签无论对于历史上一贯反对“自由软件”的BSD黑客之类的团体,还是对于不愿在GPL是非之争中表态的人均能接受。Stallman尝试着接受这个术语,但随后又以其未能代表其思想的核心为由而排斥它。从此,自由软件运动坚持同“开源”划清界限,这也许成了2003年黑客文化中最重大的政治分歧。

“开源”背后另一个(也是更重要的)意图是希望将黑客社区的方法以一种更亲和市场、更少对抗性的方式介绍给外部世界(尤其是主流商用市场)。幸运的是,在这方面,它取得了绝对成功——这也重新激起了人们对其根源——Unix传统——的兴趣。

\section{Unix的历史教训}
在Unix历史中,最大的规律就是:距开源越近就越繁荣。任何将Unix专有化的企图,只能陷入停滞和衰败。

回顾过去,我们早该认识到这一点。1984年至今,我们浪费了十年时间才学到这个教训。如果我们日后不思悔改,可能还得大吃苦头。

虽然我们在软件设计这个重要但狭窄的领域比其他人聪明,但这不能使我们摆脱对技术与经济相互作用影响的茫然,而这些就发生在我们的眼皮底下。即使Unix社区中最具洞察力、最具远见卓识的思想家,他们的眼光终究有限。对今后的教训就是:过度依赖任何一种技术或者商业模式都是错误的——相反,保持软件及其设计传统的的灵活性才是长存之道。

另一个教训是:别和低价而灵活的方案较劲。或者,换句话说,低档的硬件只要数量足够,就能爬上性能曲线而最终获胜。经济学家Clayton Christensen称之为“破坏性技术”,他在《创新者窘境》(The Innovator's Dilemma)[Christensen]一书中以磁盘驱动器、蒸汽挖土机和摩托车为例阐明了这种现象的发生。当小型机取代大型机、工作站和服务器取代小型机以及日用Intel机器又取代工作站和服务器时,我们也看到了这种现象。开源运动获得成功正是由于软件的大众化。Unix要繁荣,就必须继续采用吸纳低价而灵活的方案的诀窍,而不是去反对它们。

最后,旧学派的Unix社区因采用了传统的公司组织、财务和市场等命令机制而最终未能实现“职业化”。只有痴迷的“极客”和具有创造力的怪人结成的反叛联盟才能把我们从愚蠢中拯救出来——他们接着教导我们,真正的专业和奉献精神,正是我们在屈服于世俗观念的“合理商业做法”之前的所作所为。

如何在Unix之外的软件技术领域借鉴这些经验教训,就作为一个简单的练习留给读者吧。




\chapter{对比:Unix哲学同其他哲学的比较}
\begin{flushright}
\begin{notecard}[red!30]
如果你不知道怎样表现得高人一等,找个Unix用户,让他做给你看。

呆伯通讯3.0,1994年

{\hfill —Scott Adams}
\end{notecard}
\end{flushright}

操作系统的设计,在明显和微妙两方面,造就了该系统下软件开发的风格。本书大部分内容描绘了此两者之间的联系:Unix操作系统设计,以及由此发展出的编程设计哲学。为了便于对照,我们不妨把经典的Unix方式和其它主要操作系统的设计和编程习俗作一番比较。

\section{操作系统的风格元素}
开始讨论特定的操作系统之前,我们需要一个组织框架,来了解操作系统的设计是如何对编程风格产生或健康或病态的影响。

总的来说,与不同操作系统相关的设计和编程风格可以追溯出三个源头:(a)操作系统设计者的意图;(b)成本和编程环境的限制对设计的均衡影响;(c)文化随机漂移,传统无非就是先入为主。

即使我们承认每个操作系统社区中都存在文化随机漂移现象,那么去探究一下设计者的意图和成本及环境造成的局限也能揭示一些有趣的规律,帮助我们通过比对来更好地理解Unix风格。我们可以通过分析操作系统最重要的不同之处把这些规律明确化。

\subsection{什么是操作系统的统一性理念}
Unix有几个统一性的理念或象征,并塑造了它的API及由此形成的开发风格。其中最重要的一点应当是“一切皆文件”模型及在此基础上建立的管道概念\footnote{对没有Unix经验的读者来说,管道就是连接一个程序输出和另一个程序输入的通路。我们将在第7章讨论如何应用这个理念来帮助程序间的协作。}。总的来说,任何特定操作系统的开发风格均受到系统设计者灌注其中的统一性理念的强烈影响——由系统工具和API塑造的模型将反渗到应用编程中。

相应地,将Unix和其他操作系统作比较时,最基本的问题是:这个操作系统存在对其开发有具有决定作用的统一性理念吗?如果有,它和Unix的统一性理念有何不同?

彻头彻尾的反Unix系统,就是没有任何统一性理念,胡乱堆砌起的一些唬人特性而已。

\subsection{多任务能力}
各种操作系统最基本的不同之处之一就是操作系统支持多进程并发的能力。最低端的操作系统(如DOS或CP/M),基本上就是一个顺序的程序加载器,根本不具备多任务能力。这种操作系统在通用计算机上已经毫无竞争力。

再往上一个层次,操作系统可具有\textbf{协作式多任务}(cooperative multitasking)能力。这种系统能够支持多个进程,但是一个进程运行前必须等待前一个进程主动放弃占用处理器(这样一来,简单的编程错误就很容易将机器挂起)。这种操作系统风格是对一种硬件的暂时性适应,这种硬件虽然功能强大到支持并行操作,但要么缺乏周期性时钟中断\footnote{硬件的周期性时钟中断对分时系统来说就像心跳一样重要。时钟中断定义了单位时间片的大小,每发生一次中断,就是告诉系统可以转换到另一个任务了。在2003年,各种Unix通常将这个“心跳”设置为每秒60次或每秒100次。},要么缺乏内存管理单元,或者两者都缺。这种系统也过时了,不再具有竞争力了。

Unix系统拥有\textbf{抢先式多任务}(preemptive multitasking)能力。在Unix中,时间片由调度程序来分配,这个调度程序定期中断或抢断正在运行的进程而把控制权交给下一个进程。几乎所有的现代操作系统都支持抢占式调度。

注意,“多任务”跟“多用户”不是一回事。一个操作系统可以进行多任务处理而只支持单用户,在这种情况下,计算机支持的是单个控制台和多个后台进程。真正的多用户支持需要多个用户权限域,我们将在随后讨论内部边界时进一步讨论这个特性。

彻头彻尾的反Unix系统一,就是绝无多任务处理能力——或者通过对进程管理增设诸多的规定、限制和特殊情况来削弱多任务能力——的一个废物。

\subsection{协作进程}
在Unix中,低价的进程生成和简便的进程间通讯(IPC Inter-Process Communication)使众多小工具、管道和过滤器组成一个均衡系统成为可能。我们将在第7章探讨这个均衡体系。在这里,我们需要指出代价高昂的进程生成和IPC会带来什么后果。

\begin{quote}[Doug Mcllroy]
管道虽然在技术上容易发现,但影响却很大。进程是自主运算单元的统一性记号、而进程控制是可编程的——如果没有这些概念,那么管道技术就不可能这么简单。和Multics一样,Unix的shell(外壳)只是另外一个进程;进程控制并非受JCL(作业控制语言)之赐。
\end{quote}

如果操作系统的进程生成代价昂贵,且/或进程控制非常困难、不灵活,后果通常是:
\begin{itemize}
\item 编写怪物般巨大的单个程序成为更自然的编程方式。
\item 很多策略必须在这些庞大程序中表述。这会助长使用C++和诡谲的内部代码层级,而不是C和相对平坦的内部层级。
\item 当进程间不得不进行通讯时,要么只能采用笨拙、低效、不安全的机制(比如临时文件),要么就得依赖太多彼此的实现细节,要么彼此需了解对方的太多实现细节。
\item 广泛使用多线程来完成某些任务,而这些任务Unix只需用互通的多进程就能处理。
\item 必须学习和使用异步I/O。
\end{itemize}

这些就是操作系统环境的局限性所导致的常见风格缺陷(甚至应用程序编程中也一样)的实例。

管道和所有其他经典Unix IPC方法有一个精微的性质,就是要求把程序间的通讯简化到某一程度而促使功能分离。相反地,如果没有与管道等效的机制,则程序必须在完全相互了解对方内部细节的基础上设计程序,才能实现彼此间的合作。

一个操作系统,如果没有灵活的IPC和使用IPC的强大传统,程序间就得通过共享结构复杂的数据实现通讯。由于一旦有新的程序加入通讯圈,圈子里所有程序的通讯问题都必须重新解决,所以解决方案的复杂度与协作程序数量的平方成正比。更糟糕的是,其中任何一个程序的数据结构发生变化,都说不定会给其它程序带来什么隐蔽的bug。

\begin{quote}[Doug Mcllroy]
Word、Excel、PowerPoint和其他微软程序对彼此的内部具有“密切”——有些人可能称之为“杂乱”——的了解。在Unix中,一组程序设计时不仅要尽量考虑相互协作,而且要考虑和未知程序的协作。
\end{quote}

我们将在第7章再谈这个主题。

彻头彻尾的反Unix系统,就是让进程的生成代价高昂,让进程的控制困难而死板,让IPC可有可无,对它不予支持或支持很少。

\subsection{内部边界}
Unix的准绳是:程序员最清楚一切。当你对自己的数据进行危险操作(例如执行rm -rf *.)时,Unix并不阻止你,也不会让你确认。另一方面,Unix却小心避免你踩在别人的数据上。事实上,Unix提倡设立多个帐户,每一个帐户具有专属、可能不同的权限,以保护用户不受行为不端程序的侵害\footnote{现在时髦的术语是基于角色的安全策略(Role-Based Security)。}。系统程序通常都有自己的“伪用户(pseudo-user)帐号”,以访问专门的系统文件,而不需要无限制的(或者说超级用户的)访问权限。

Unix至少设立了三层内部边界来防范恶意用户或有缺陷的程序。一层是内存管理:Unix用硬件自身的内存管理单元(MMU)来保证各自的进程不会侵入到其它进程的内存地址空间。第二层是为多用户设置的真正权限组——普通用户(非root用户)的进程未经允许,就不能更改或者读取其他用户的文件。第三层是把涉及关键安全性的功能限制在尽可能小的可信代码块上。在Unix中,即使是shell(系统命令解释器)也不是什么特权程序。

操作系统内部边界的稳定不仅是一个设计的抽象问题,它对系统安全性有着重要的实际影响。

彻头彻尾的反Unix系统,就是抛弃或回避内存管理,这样失控的进程就可以任意摧毁、搅乱或破坏掉其它正在运行的程序:弱化甚至不设置权限组,这样用户就可以轻而易举地修改他人的文件和系统的关键数据(例如,掌控了Word程序的宏病毒可以格式化硬盘);依赖大量的代码,如整个shell和GUI,这样任何代码的bug或对代码的成功攻击都可以威胁到整个系统。

\subsection{文件属性和记录结构}
Unix文件既没有记录结构(record structure)也没有文件属性。在一些操作系统中,文件具有相关的记录结构:操作系统(或其服务程序库)通过固定长度的纪录,了解文件,或文本行终止符以及CR/LF(回车/换行)是不是该作为单个逻辑字符读取。

在另一些操作系统中,文件和目录可以具备相关的名字/属性对——(例如)采用编外数据(out-of-band data)将文档文件同能够解读它的应用程序关联起来。(Unix处理这种联系的典型方法是让应用程序识别“特征数”或是文件内的其它类型数据。)

操作系统级的记录结构通常只是一个优化手段,几乎只会使API和程序员的生活复杂化之外没别的用,还会助长不透明的面向记录的文件格式,使得文本编辑器之类的通用工具无法正确读取。

文件属性会很有用,但是(我们在第20章将发现)在面向字节流工具和管道的世界中,它可能引发一些棘手的语义问题。对文件属性的操作系统级支持会诱导程序员使用不透明的文件格式,让他们依靠文件属性将文件格式同对应的解读程序绑在一起。

彻头彻尾的反Unix系统,应用一套拙劣的记录结构,任何特定的工具能否像文件编写者希望的那样读懂文件,完全是靠运气。加入文件属性,并让系统依赖于这些文件属性,就无法通过查看文件内的数据来确定文件的语义。

\subsection{二进制文件格式}
如果你的操作系统使用二进制文件格式存放关键数据(如用户帐号记录),应用程序采用可读文本格式的传统就很可能无法形成。我们将在第5章详细解释为什么这是一个问题。现在只要注意,这种做法可能会带来以下后果就够了。

\begin{itemize}
\item 即使支持命令行接口、脚本和管道,也几乎无法形成过滤器。
\item 数据文件足有通过专用工具才能访问。开发者的思维会以工具而非数据为中心。这样,不同版本的文件格式很难兼容。
\end{itemize}

彻头彻尾的反Unix系统,让所有文件格式都采用不透明的二进制格式,后者要用重量级的工具才能读取和编辑。

\subsection{首选用户界面风格}
我们将在第11章详细讨论\textbf{命令行界面}(CLI)和\textbf{图形用户界面}(GUI)的差异所产生的影响。操作系统的设计者把哪一种选作一般表现模式,将影响设计的许多方面——从进程调度、内存管理直到应用程序使用的\textbf{应用程序编程接口}(API)。

第一款Macintosh已经发布很多年了,不用说人们也会觉得操作系统的GUI没做好是个问题。Unix的教训则相反:CLI没做好是一个不太明显但同样严重的缺陷。

如果操作系统的CLI功能很弱或根本不存在,其后果会是:
\begin{itemize}
\item 程序设计不会考虑以未预料到的方式相互协作——因为无法这样设计。输出不能用作输入。
\item 远程系统管理更难于实现,更难以使用,更强调网络。\footnote{微软重建Hotmail时认真考虑了这个问题。参见[BrooksD]。}
\item 即便简单的非交互程序也将招致GUI开销或复杂的脚本接口。
\item 服务器、守护程序和后台进程几乎无法写出,至少很难以优雅的方式写出。
\end{itemize}

彻头彻尾的反Unix系统,就是没有CLI,没有脚本编程能力——或者,存在CLI不能驱动的重要功能。

\subsection{目标受众}
不同的操作系统设计是为了适应不同的目标受众。有的为后台工作设计,有的则设计成桌面系统。有的为技术用户而设计,有的则为最终用户设计。有的能在实时控制应用中单机工作,有的则为分时系统和普遍联网的环境设计。

一个重要的差异是客户端与服务器之分。“客户端”可以理解为:轻量,只支持单个用户,能够在小型机器上运行,随需开关机器,没有抢先式多任务处理,为低延迟作了优化,大量资源都用在花哨的用户界而上。“服务器”可以解释为:重量,能够连续运行,为吞吐量优化,完全抢占式多任务处理以处理多重会话。所有的操作系统最初都是服务器操作系统。客户端操作系统的概念仅在二十世纪七十年代后期随着价格不高、性能一般的PC硬件的出现才产生。客户端操作系统更关注用户的视觉体验,而不是7*24小时的连续正常运行。

所有这些变数都对开发风格产生影响。其中最明显的就是目标用户能够容忍的界面复杂的级别,以及如何在可感知复杂度和成本、性能等其它变数之间权衡轻重。人们常说,Unix是程序员写给程序员的——这个目标用户群在界面复杂度的承受力方面是出了名的。

\begin{quote}[Ken Thompson]
这与其说是一个目标不如说是一个结果。如果“用户”这个词带有“单纯得傻乎乎”的蔑视含义,我憎恨一个为“用户”设计的系统。
\end{quote}

彻头彻尾的反Unix系统,就是一个自认为比你自己更懂你在干什么的操作系统,然后雪上加霜的是,它还做错了。

\subsection{开发的门坎}
区分操作系统的另一个重要尺度是纯用户转变为开发者的门坎高度。这里有两个重要的成本动因。一个是开发工具的金钱成本,另一个是成为一个熟练开发者的时间成本。有些开发文化还形成了一个社会性门坎,但这通常是背后的技术成本带来的结果,而不是根本原因。

昂贵的开发工具和复杂晦涩的API造就了小群的精英编程文化。在这种文化中,编程项目是大型而严肃的活动——为了证明所投资的软(人力)硬资本物有所值,这些工程必须如此。大型而严肃的工程常常产生大型、严肃的程序(而且,更常见的是,大型而昂贵的失败)。

廉价工具和简单接口支持的是轻松编程、玩家文化和开拓探索。编程项目可以很小(通常,正式的项目结构显然毫无必要),失败了也不是什么大灾难。这改变了人们开发代码的风格;尤其是,他们往往不会过分依赖已经失败的方法。

轻松编程往往会产生许多小程序和一个自我增强、不断扩展的知识社区。在廉价硬件的世界里,是否存在这样一个社区日益成为一个操作系统能否长寿的重要因素。

Unix开创了轻松编程的先河。Unix的众多首创之一就是将编译器和脚本工具放在默认安装中,可供所有用户使用,支持了一种跨越众多机器的玩家开发文化。在很多Unix下写代码的人并不认为自己在写代码——他们认为是在为普通任务的自动化编写脚本,或在定制环境。

彻头彻尾的反Unix系统,不可能进行轻松编程。

\section{操作系统的比较}
当我们将Unix和其他操作系统对比时,Unix设计决策的逻辑就更清楚了。这里,我们只是纵览各种设计,对各操作系统技术特性的具体讨论请参考OSData网站。\footnote{参考OSData网站\href{http://www.osdata.com/}{http://www.osdata.com/}。}

图3-1表明我们要纵览的各种分时系统之间的渊源关系。其它一些操作系统(灰色标记,不一定是分时系统)因脉络的关系也包括进来了。实线框里的系统仍旧存在。“出生”日期是指第一次发布的时间;\footnote{Multics除外,它影响最大的时期是自技术规格公布的1965年到实际发布的1969年间。}“死亡”日期通常指厂商终结系统的时间。

实线箭头表示存在起源关系或很强的设计影响(例如,后开发的系统API有意通过逆向工程以匹配先前的系统);短画线表示重大的设计影响;点线表示微弱的设计影响。不是所有的起源关系被开发者认可;实际上,有些出于法律或企业策略的原因被官方否认,但在业界其实是公开的秘密。

“Unix”框包括所有的专有Unix,包括AT\&{}T版本和早期的伯克利版本。“Linux”框包括所有的开源Unix(均在1991年开始)。这些开源Unix与早期的Unix有渊源关系,它建立在1993年诉讼协议后从AT\&{}T专有控制下解放出来的代码的基础上。\footnote{这次诉讼的详细情况可以参考Marshall Kirk McKusick在[openSources]中的论文。}

\begin{linefig}{分时系统历史示意图}
\label{fig:分时系统历史示意图}
\end{linefig}

\subsection{VMS}
VMS是一个专有操作系统,最初由数字设备公司(Digital Equipment Corporation)为VAX小型机开发。VMS于1978年面世,是二十世纪八十年代和九十年代早期一个非常重要的产业化操作系统产品,无论在Compaq并购DEC,还是Hewlett-Packard并购Compaq之后,这个系统一直得到了维护。直到2003年中期,这个产品仍在销售和支持,尽管今天已经没有多少人用它搞新的开发了\footnote{更多信息可以从OpenVMS.org网站\href{http://www.openvsm.org}{http://www.openvsm.org}获得。}。在这里提出VMS,是为了对比Unix和来自小型机时代的其它CLI操作系统。

VMS具有完全抢占式多任务处理能力,但是进程生成的开销极为昂贵。VMS文件系统有复杂的记录类型(但还不是记录属性)概念。这些特性造成了我们此前描述的后果,尤其(在VMS中)是程序庞大、个体臃肿的倾向。

VMS的特点是具有长长的、可读的、类COBOL的系统命令和命令选项。它具有非常全面的在线帮助(针对的不是API,而是可执行程序和命令行语法)。事实上,VMS命令行界面及其帮助系统就代表了VMS的组织结构。尽管在该系统上已经具有翻新版的X window,但冗长的CLI仍然对编程设计的风格产生了重要影响。主要可以理解为:

\begin{itemize}
\item 命令行功能的使用频率——要打的字越多,愿意用的人就越少。
\item 程序的大小——人们希望少打字,因此想少用几个软件,于是将更多功能捆绑到大型程序中。
\item 程序可接受选项的数量和类型——必须遵守帮助系统规定的语法限制。
\item 帮助系统的易用性——很完备,但缺少辅助的搜索和查找工具,并且索引做得很差。这样不容易获取大量的知识,鼓励了专业化而阻碍了轻松编程。
\end{itemize}

VMS的内部边界系统有口皆碑。它为真正的多用户操作而设计,完全利用硬件MMU来保护进程互不干扰。系统命令解释器具有优先权,但在另一方面,关键功能封装则做得相当不错。VMS的安全漏洞一直都很罕见。

VMS工具最初很贵,界面也很复杂。大卷大卷的VMS程序员文档只有纸张形式,因此要查找任何东西都既费时又费钱。这往往会阻碍探索性编程,降低人们对大型工具包的学习兴趣。直到几乎被厂商抛弃之后,VMS才形成一种轻松编程和玩家文化,但这种文化并非很强。

和Unix一样,VMS早就有了客户端/服务器的划分。作为一个通用分时系统,VMS在它的时代是成功的。VMS的目标受众基本是技术用户和大量应用软件的商业领域,这也意味着用户对其复杂度尚能容忍。


\subsection{MacOS}
Macintosh操作系统是1980年代初由Apple公司设计的,灵感来自此前Xerox公司Palo Alto研究中心在GUI方面的开拓性工作。1984年与Macintosh计算机一起面世。MacOS经历过两场重大的设计变革,第三场正在酝酿中。第一次是从一次只支持一个应用程序转变到能够多任务协作处理多个应用程序( MultiFinder);第二次是从68000处理器到PowerPC处理器的转变,既保留了68K应用程序的二进制向后兼容,又为PowerPC应用程序引入了高级共享库管理系统,代替了原来的68K基于陷阱指令的代码共享系统(trap instruction-based code sharing system);第三次是在MacOS X中把MacOS设计理念和来自Unix的架构融合起来。如果没有特别指出,此处讨论仅限OS-X之前的版本。

MacOS有一个不同于Unix的坚定统一性理念:Mac畀面方针(the Mac Interface Guidelines)。这些方针非常详细地说明了应用程序GUI的表现形式和行为模式。这些原则的一致性在很多重要方面影响了Mac用户的文化。不遵循这些原则、简单移植DOS或Unix程序的产品立即遭到了Mac用户的拒绝,在市场上一败涂地,而这并不罕见。

这些方针的主旨是:东西永远呆在你摆的地方。文档、目录和其它东西在桌面上都有固定的、系统不会弄乱的位置,重启后桌面依然保持原样。

Macintosh的统一性理念非常强大,我们上面讨论过的其它设计方案要么受其影响,要么就无人问津。所有的程序都得有GUI,根本没有CLI。脚本的功能有倒是有,但绝对不像Unix中那样常用,很多Mac程序员根本就不去学习。MacOS界面至上的GUI做法(被组织到单个的主事件循环中)导致了其薄弱的非抢占式的程序调度能力。这个弱程序调度器以及所有的MultiFinder应用程序都在单个大地址空间运行,这意味着使用分离的进程甚或线程来代替轮询\footnote{轮询法的概念是,由CPU定时发出询问,依序询问每一个周边设备是否需要其服务,有即给予服务,服务结束後再问下一个周边,接着不断周而复始。}是不现实的。

然而,MacOS的应用程序并非总是庞然大物。系统的GUI支持代码,部分在硬件自带的ROM实现,部分在共享库中实现,通过事件接口同MacOS中的程序进行通信,这个接口从诞生起就一直相当稳定。这样,这种操作系统的设计提倡的是把应用引擎和GUI接口相对清晰地分离开来。

MacOS也强烈支持把应用程序的元数据(如菜单结构)从引擎中隔离。MacOS文件分“数据分支”(data fork)(Unix风格的字节包,包含文档或程序代码)和“资源分支”(resource fork)(一套用户定义的文件属性)。Mac应用程序通常是这样设计的,(比如)程序中使用的图像和声音存储在资源分支中,可以独立于应用程序码进行修改。

MacOS系统的内部边界系统很弱。因为基于只有单个用户这样的固定设想,所以没有用户权限组。多任务处理是协作式的,不是抢占式的。所有的MultiFinder应用都在同一个地址空间运行,所以任何应用程序的不良代码都能破坏操作系统低层内核以外的任何部分。针对MacOS机器的安全攻击程序很容易编写,这个系统一直没遭到大规模攻击只是因为没人有兴趣罢了。

Mac程序员和Unix程序员在设计上往往走截然相反的路,即,他们的设计是从界面向内进行,而不是从引擎向外进行(我们将在20章讨论这种方式的影响)。MacOS的一切设计共同促成了这种做法。

Macintosh的目标是作为是服务非技术终端用户的客户端操作系统,这就意味着用户对界面复杂度的容忍度很低;Macintosh文化下的开发者于是非常非常擅长设计简洁的界面。

假设你已经有Macintosh机器,那么晋级为开发者的代价一向不高。因此,尽管界面相当复杂,Mac很早也形成了一种浓厚的玩家文化。开发小工具、共享软件和用户支持软件的传统一直非常盛行。

经典的MacOS已经寿终正寝。MacOS大多数功能已被引入Macos X,并同源自Berkeley传统的Unix架构结合在一起\footnote{MacOS x实际是两层专有代码(OpenStep移植码和经典Mac GUI)和开源Unix核心上(Darwin)的组合。}。同时,像Linux这样的前沿Unix也开始从MacOS中借鉴一些理念,如文件属性(资源分支的泛化)。


\subsection{OS/2}
OS/2是作为IBM命名为“ADOS”(Advanced DOS)的开发项目诞生的,也是想成为DOS 4的三个竞争者之一。那时,IBM和微软在正式合作,为PC机开发下一代操作系统。OS/2 1.0版本首发于1987年,为286机开发,并不成功。针对386的2.0版本发布于1992年,但那时IBM和微软联盟已经破裂。微软走向一个不同的(而且更赚钱的)方向——Windows 3.0。OS/2虽然吸引了一小部分忠诚的拥趸,但从来没有吸引到足够多的开发者和用户。直到1996年后IBM把它纳入Java计划前,OS/2在桌面市场一直排在Macintosh之后,位居第三。最新版本是1996年发布的4.0版本。那些早期的版本在嵌入式系统中找到了出路,时至2003年年中,还在全球众多银行自动柜员机上运行。

和Unix一样,OS/2使用抢先式多任务处理,不能在没有MMU的机器上运行(早期版本使用286的内存分段来模拟MMU)。跟Unix不同的是,OS/2从来都不是一个多用户系统。虽然它的进程生成开销相对较低,但是IPC困难而脆弱。网络能力最初仅限于LAN协议,但后续版本也增加了TCP/IP协议栈。因为没有类似于Unix的服务守护程序,所以,OS/2处理多功能网络的能力一直欠佳。

OS/2既有CLI又有GUI。OS/2流传下来的亮点大多围绕它的桌面Workplace Shell(WPS)。有一些技术从AmigaOS Workbench\footnote{作为对某些Armga技术的回报,IBM给予了Commodore公司\\ REXX脚本语言的授权。此项交易详情请查询:\\ \href{http://www.os2bbs.com/os2news/OS2Warp.html}{http://www.os2bbs.com/os2news/OS2Warp.html}。}的开发者处授权得到。AmigaOS Workbench是GUI桌面的先驱,直到2003年,在欧洲还拥有众多忠实的爱好者。这也是OS/2的能力超过Unix(这一点尚有争论)的唯一设计领域。WPS(WorkplaceShell)是一个干净、强大、面向对象的设计,具有易懂的行为特性和良好的可扩展性。几年后,OS/2成为Linux GNOME工程的模型。

WPS的类层次设计是OS/2的统一性理念之一。另一个统一性理念是多线程处理。OS/2程序员大量使用线程,部分代替了对等进程间的IPC,协作程序工具包传统也因此没能形成。

OS/2的内部边界达到了单用户操作系统的预期。运行的进程互不干扰,内核空间也和用户空间互不干扰,但是没有了基于每用户的特权组。这意味着文件系统无法防范恶意代码。另一个结果是没有类似于起始目录的东西,应用程序的数据往往散布在整个系统中。

缺乏多用户能力所产生的进一步后果就是在用户空间不存在权限区别。这样,开发者往往只信任内核代码。Unix中许多由用户态守护进程处理的系统任务在OS/2中只好塞进内核或WPS,结果是两者都臃肿。

OS/2有一种和二进制模式相对的文本模式(文本模式下CR/LF被读作单个的行结束符,在二进制模式下无此含义),但是没有其它的文件记录结构。OS/2支持文件属性,效仿Macintosh风格,用文件属性来支持桌面持久性。系统数据库大都是二进制格式。

首选UI风格贯穿于WPS。从人体工程学角度来说,WPS的用户界面要强于Windows,虽然还没有达到Macintosh的标准(OS/2最活跃的时段在经典MacOS的历史上处于早期)。与Linux和Windows一样,OS/2的用户界面围绕多个独立的窗口任务组,而不是让运行的应用程序占据整个桌面。

OS/2的目标对象是商业和非技术的最终用户,意味着对界面复杂度的容忍度较低。OS/2既可用作客户端操作系统,也可用作文件和打印服务器。

在二十世纪九十年代初期,OS/2社区的开发者开始转向受Unix启发、模仿POSIX接口的EMX环境。到二十世纪九十年代后期,已经有很多Unix软件被移植到OS/2上。

任何人都可以下载EMX,包括GNU编译器集合以及其它开源开发工具。IBM不时在OS/2开发包中发布系统文档,并被转载到了许多BBS和FTP站点上。正因为如此,到1995年,用户开发的OS/2软件的“Hobbes”FTP档案已经超过了1GB。一个崇尚小巧工具、探索编程和共享软件的强大传统形成了,即使OS/2自身已经被丢进了历史的垃圾箱,这个传统仍然还会长期拥有一批忠诚的追随者。

在Windows 95发布以后,OS/2社区在微软的围剿和IBM的支援下,对Java的兴趣与曰俱增。自Netscape在1998年初公开源码后,他们的方向又(陡然)转到了Linux上。

一个多任务处理、但单用户的操作系统到底能走多远?OS/2是一个相当有趣的案例。从其得出的大部分结论都可以很好地运用到其它同类型操作系统中,尤其是AmigaOS\footnote{AmigaOS的主页\href{http://os.amiga.com/}{http://os.amiga.com/}}和GEM\footnote{GEM操作系统\\ \href{http://www.geocities.com/SiliconValley/Vista/6148/gem.html}{http://www.geocities.com/SiliconValley/Vista/6148/gem.html}}。直到2003年,大量的OS/2材料还可从网上获得,包括一些闪光的历史。\footnote{例如,参考OS Voice\href{http://www.os2voice.org/}{http://www.os2voice.org/}\\和OS/2 BBS.COM\href{http://www.os2bbs.com/}{http://www.os2bbs.com/}}


\subsection{Windows NT}
Windows NT(New Technology)是微软为高端个人用户和服务器设计的操作系统:发行的版本实际上有好几个,我们为了讨论方便把它们视为一个系统。自从2000年公布的Windows ME终结后,目前所有的Window操作系统都以Window NT为基础;Windows2000是NT 5,Windows XP(本书写作时是2003年)是NT 5.1。NT起源自VMS,很多重要特性与VMS相同。

NT是逐步堆积而成的,缺乏对应于Unix“一切皆文件”或MacOS桌面的统一性理念。由于它的核心技术没有扎根于一小群稳固的中枢观念中,\footnote{也许,会有人争辩说,所有微软操作系统的统一性理念是:“套牢客户”。}因此每过几年就会过时。每一代技术——DOS(1981),Windows 3.1(1992),Windows 95(1995),Wi-\\ndows NT 4(1996),Windows 2000(2000),Windows XP(20-\\02)和Windows Server 2003(2003)——随着旧方式被宣告过时而不再有良好支持,开发者必须以不同的方式从头学起。

下面是其它一些后果:
\begin{itemize}
\item GUI功能与继承自DOS和VMS的戏留命令行界面不能稳定共存。
\item 套接字编程没有类似Unix那种“一切皆是文件句柄”的统一数据对象,因此在Unix中很简单的多道程序设计和网络应用到NT下则要牵涉更多基础性概念。
\end{itemize}

NT的一些文件系统类型也有文件属性,但仅限用于为实现某些文件系统的访问控制列表,因此对开发风格不会产生太大影响。NT也有文本和二进制这两种记录类型区别,时不时地讨人嫌(NT和OS/2都从DOS那里继承了这个不良特性)。

尽管支持抢先式多任务处理,但进程生成却很昂贵——虽然比不上VMS,但是(平均生成一个进程需要0.1秒左右)要比现在的Unix高出一个数量级。脚本功能薄弱,操作系统广泛使用二进制文件格式。除了此前我们总结过的,还有这些后果:

\begin{itemize}
\item 大多数程序都不能用脚本调用。程序间依赖复杂脆弱的远程过程调用(RPC)来通信,这是滋生bug的温床。
\item 根本就不存在通用工具。没有专用软件就不可能读取或编辑文档和数据库。
\item 随着时间的推移,CLI越来越被忽略了,原因是环境稀缺。薄弱CLI引起的问题不仅没有得到改善,反而越来越糟糕。(Windows Server 2003试图稍稍扭转这种趋势。)
\end{itemize}

Unix的系统配置和用户配置数据分散存放在众多的dotfiles(名字以“.”开头的文件)和系统数据文件中,而NT则集中存放在注册表中。以下后果贯穿于设计中:

\begin{itemize}
\item 注册表使得整个系统完全不具备正交性。应用程序的单点故障就会损毁注册表,经常使得整个操作系统无法使用、必须重装。
\item \textbf{注册表蠕变}(registry creep)现象:随着注册表的膨胀,越来越大的存取开销拖慢了所有程序的运行。
\end{itemize}

互联网上的NT系统因易受各种蠕虫、病毒、损毁程序以及破解(crack)的攻击而臭名昭著。原因很多,但有一些是根本性的,最根本的就是:NT的内部边界漏洞太多。

NT有访问控制列表,可用于实现用户权限组管理,但许多遗留代码对此视而不见,而操作系统为了不破坏向后兼容性又允许这种现象的存在。在各个GUI客户端之间的消息通讯机制也没有安全控制,如果加上的话,也会破坏向后兼容性。

虽然NT将要使用MMU,出于性能方面的考虑,NT 3.5后的版本将系统GUI和优先内核一起塞进了同一个地址空间。为了获得和Unix相近的速度,最新版本的NT甚至将Web服务器也塞进了内核空间。

由于这些内部边界漏洞产生的协合效应,要在NT上达到真正的安全实际上是不可能的\footnote{实际上,微软已经于2003年3月公开承认NT系统的安全是不可靠的。}。如果入侵者随便作为什么用户把一段代码运行起来(例如,通过Outlook email宏功能),这段代码就可以通过窗口系统向其它任何运行的应用程序发送虚假信息。只要利用缓存溢出或GUI及Web服务器的缺口就可以控制整个系统。

因为Windows没有处理好程序库的版本控制问题,所以长期备受被称为“DLL地狱(DLL hell)”配置问题的折磨,在这个问题中,安装新程序可以任意升级(或降级)现有程序运行依赖的库文件。专用的应用程序库和厂商提供的系统库都存在这个问题:应用程序和特定版本的系统库一起发布非常普遍,一旦没有特定的系统库,应用程序就会无声无息地垮掉。”\footnote{在有处理库版本问题能力的.NET开发框架发布后,DLL hell问题有所缓解一一但是直到2003年.NET只随NT最高端的服务器版本提供。}

从好的一面来看,NT提供了足够的特性来支持Cygwin。Cygwin是一个在实用工具和API两个层次上实现Unix的兼容层,而且只有极少的特性损失\footnote{Cygwin很大程度上符合“单一Unix规范”,但是要求直接硬件存取的程序会被上层的Windows内核限制。以太网卡就是出了名的问题多。}。Cygwin允许C程序既可以使用Unix API又可以使用原生API,许多为形势所迫不得不使用Windows的Unix黑客在Windows系统上安装的第一个程序就是Cygwin。

NT操作系统的目标用户主要是非技术型最终用户,意味着对界面复杂度的容忍度非常低。NT既可作客户端又可作服务器。

在其历史早期,微软依靠第三方开发商提供应用软件。起初,微软还公布Windows API的完整文档,并保持其开发工具的低价格。但是,随着时间的推移、竞争者的相继倒下,微软转而青睐内部开发的战略,开始向外界隐藏API,开发工具也越来越昂贵。早在Windows 95时期,微软就要求将保密协议作为购买专业级开发工具的一个条件。

围绕DOS和Windows早期版本形成的玩家文化和轻松开发文化已经足够壮大,即使在微软日益加强的排挤(包括为了把业余开发者非法化而设立的各种认证计划)下也足以自我维系。共享软件从未消亡,而在2000年后,迫于开源操作系统和Java的市场压力,微软的策略也略有转变。但是,随着时间的推移,供“专业”编程使用的Windows接口越来越复杂,将轻松(或严肃!)编程的门槛越抬越高。

这段历史的后果就是业余NT和职业NT开发者的设计风格存在尖锐的分歧——两个群体之间几乎不通气。尽管小型工具和共享软件的玩家文化非常活跃,但职业NT项目却往往产出庞然大物,甚至比那些VMS一样的“精英”操作系统还要臃肿。

Windows下的Unix风格的shell功能、命令集和API函数库来自第三方,包括UWIN、Interix和开源Cygwin。
 
\subsection{BeOS}
Be公司作为一家硬件厂商成立于1989年,基于PowerPC芯片开发了颇具开拓精神的多处理机器。BeOS操作系统是Be公司为给硬件增值而发明的一种新型、内置网络功能的操作系统模型,吸收了Unix和MacOS两个家族的经验教训,但又不和任何一个雷同。他们的努力造就了一个雅致、简洁、令人激动的设计,在其定位的多媒体平台这个角色上表现卓越。

BeOS的统一性理念是“深入地线程化”、多媒体流和数据库形式的文件系统。BeOS的设计目标是尽可能减少内核延迟,从而能非常适合实时处理大量数据,如音频和视频流。既然支持线程本地存储从而不需共享所有地址空间,BeOS的“线程”实际上就是Unix术语中的轻量级进程。IPC通过共享内存实现,快速而高效。

BeOS采用的是Unix模型,在字节级以上没有文件结构。BeOS和MacOS一样支持和使用文件属性。事实上,BeOS文件系统就足一个数据库,可以按任意属性索引。

BeOS借鉴Unix的设计是巧妙的内部边界设计。BeOS充分应用了MMU,而且有效地使各个运行进程互不干涉。虽然BeOS是个单用户操作系统(不用登录),但在文件系统和操作系统内部的其它地方都支持类似Unix的权限组。这些措施用于保护系统的关键文件免受不信任代码的侵袭:用Unix的术语来讲,就是用户在启动时作为匿名用户登录,另一个“用户”是root。如果需要完整的多用户操作,其实对系统上层产生的变化也会很小,实际上确实存在一个BeLogin实用程序。

BeOS倾向使用二进制文件格式和文件系统自带的数据库,而不使用类Unix的文本格式。

BeOS的首选UI风格是GUI,在界面设计上大量借鉴了MacOS的经验,但是完全支持CLI和脚本功能。BeOS的命令行shell是移植自Unix最主要的开源shell—bash(1),通过POSIX兼容库运行。移植Unix CLI软件在设计上相当容易。Unix模式的整套脚本、过滤器和守护进程的基础设施都到位了。

BeOS的目标定位是作为一个专门针对近实时(near-real-time)多媒体处理(尤其是音频和视频操控)的客户端操作泵统。BeOS的目标受众包括技术和商业用户,这也意味着用户对界面复杂度的容忍度属中等。

BeOS的开发门槛很低:尽管操作系统是专有的,但是开发工具并不贵,而且很容易获得整套文档。BeOS项目起初部分为了通过RISC技术把Intel硬件拉下马,在互联网大爆炸后,继续往一个纯软件方向努力。在1990年代初Linux形成时期,BeOS的战略家就已经一直关注着、而且也充分意识到一个庞大的轻松开发者团体的价值。事实上,他们成功地吸引了一批非常忠诚的追随者;到2003年,至少有五个以上的不同工程正在努力试图用开源复兴BeOS。

不幸的是,BeOS的经营战略却不像其技术设计那样精明。起初,BeOS软件捆绑在专用硬件上,市场推广时对目标应用的说明也含混不清。后来(1998年),BeOS被移植到通用PC机上,更紧密关注多媒体应用,但是从未吸引到足够数量的应用和用户群。最后,到2001年,BeOS死于微软的反竞争运动(2003年仍在进行诉讼)和各种已具备多媒体功能的Linux的联合打击之下。

\subsection{MVS}
MVS(多重虚拟存储)是IBM大型计算机的旗舰操作系统,起源可以追溯到OS/360。OS/360诞生于1960年代中期,是IBM当时很新型的System/360计算机系统上向客户推荐的操作系统。今天IBM大型机操作系统的核心还保留着OS/360的后裔代码。虽然整个代码几乎都已经重写了,但是基本设计大多原样未动;向后兼容性被虔诚地保留了下来。这种兼容性甚至达到这种地步:即使历经三代结构升级,为OS/360编制的应用程序还能不加修改就在装有MVS的64位z/系列大型机上运行。

在上述讨论过的所有操作系统中,MVS是唯一可视为比Unix还要悠久的操作系统(不确定性在于随着时间的推移,MVS究竟发展到了什么地步)。这个操作系统也是受Unix概念和技术影响最小的操作系统,因而代表了跟Unix反差最强烈的一种设计。MVS的统一性理念是:一切皆批处理。系统的设计目标是尽可能最有效利用机器批处理巨大规模的数据,尽量减少与人类用户的交互。

原生的MVS终端(3270系列)只能以块模式运行。用户通过屏幕修改终端的本地存储。用户按下发送键前主机不会产生任何中断。不可能实现Unix原始模式(raw mode)下那种字符层面上的交互。

TSO是和Unix交互环境最近似的等价物,自身能力非常有限。对于系统其它部分来说,每个TSO用户都是模拟批作业。这个设施非常昂贵——太贵了,主要限于开发者和系统维护者使用。仅仅需要通过终端运行应用程序的普通用户几乎从不使用TSO。相反,他们通过事务监视器工作。这是一种多用户应用服务器,可以进行协作式多任务处理并支持异步输入/输出。从效果上来说,每种事务监视器都是一个专用的分时插件(和运行CGI的Web服务器很像,但不完全一样)。

面向批处理体系所带来的另一个后果就是生成进程非常缓慢。I/O系统有意用较高的准备成本(及其带来的延迟)来换取更好的吞吐能力。这些选择对于批处理操作来说非常适宜,但是对于交互响应来说却是致命的。可以预见,如今TSO用户将把几乎所有的时间都花在ISPF(一个对话驱动的交互环境)上。除了启动一个ISPF实例外,程序员几乎不在原生的TSO上做任何事情。这避免了生成进程的开销,代价是引进了一个非常庞大的程序。这个程序,除了不会启动机房的咖啡壶,什么事都能做。

MVS使用机器MMU,进程有独立的地址空间,只能通过共享内存支持进程间通信,也有线程功能(MVS称之为“子任务”),但用得很少,主要因为只有用汇编语言编写的程序才能方便地使用这个功能。与此相反,典型的批处理应用是由JCI。(Job Control Language,作业控制语言)粘合在一起的由重量级程序调用组成的短序列,也提供脚本功能,但却是出了名的困难和死板。每个作业里的程序通过临时文件通信:过滤器之类的东西几乎毫无用武之地。

每个文件都有记录格式,有时是隐式的(例如,JCL的内联输入文件继承了穿孔卡做法,默认为80字节固定长度的记录格式),但更通常的情况是明确指定。许多系统配置文件都采用文本格式,但应用程序文件通常采用特定的二进制文件。一些检查文件的通用工具出于迫切需求才被开发出来,但这依然还是一个难以解决的问题。

文件系统的安全性在最初设计中根本未予考虑。然而,当人们发现安全性十分必要时,IBM以一种颇具灵感的方式加了进去:他们规定了一套通用安全性API,然后在处理每个文件存取请求前调用这个接口。结果是,产生了三种相互竞争的安全性程序包,各代表不同的设计理念——三种都相当好,在1980年到2003年中期始终没被攻破。这种多样性就允许用户安装时选择最适合实际安全策略的安全包。

网络功能也是后来才加进去的。网络连接和本地文件操作使用同一套接口的概念不存在;两者的编程接口相互独立而且区别很大。这的确帮助TCP/IP成为了首选网络协议,不着痕迹地挤掉了IBM原生的SNA(System Network Architecture,系统网络体系)。在2003年,同一机器上两者都使用的情况虽然常见,但是SNA正在逐渐消亡。

除了在运行MVS的大企业内部,MVS上的轻松编程几乎不存在。这主要不在于工具自身的成本,而在于环境的成本——在往计算机系统上扔进几百万美元后,每个月为编译器花费几百美元就是小钱了。然而,在这个社区内也存在一个繁荣的自由软件文化,主要是编程和系统管理工具。第一个计算机用户组,SHARE,就是IBM用户在1955年成立的,到今天也依然很兴旺。

考虑到架构上的巨大差别,MVS是第一款符合单一Unix规范(Single Unix Specification)的非System-V操作系统,这件事非同寻常(但还是得看到,从Unix软件移植过来的软件往往碰到ASCII对EBCDIC字符集的麻烦)。从TSO启动Unix shell是可能的——Unix文件系统专门设置成MVS数据集格式。MVS Unix字符集是特殊EBCDIC代码页,交换了“新行”和“换行”(Unix中的“换行”对MVS就是“新行”),但是系统调用却是在MVS内核上实现的实时系统调用。

随着开发环境的费用下降到爱好者能够承受的范围,公共领域的MVS版本(版本3.8,始于1979年)拥有了一小群用户,人数虽少却在不断增长。这个系统及其开发工具和运行所用的仿真器,花一张CD的价钱就可以全部获得\footnote{\href{http://www.cbttape.org/cdrom.htm}{http://www.cbttape.org/cdrom.htm}}。

MVS的目标始终定位在后勤部门。和VMS和Unix一样,MVS提前区分了服务器和客户端。后勤用户对界面复杂度不仅可以忍受,而且非常期待,因为他们愿意把昂贵的计算机资源尽可能花在需要处理的工作上而不是界面上。

\subsection{VM/CMS}
VM/CMS是IBM另一个大型机操作系统。从历史来说,这是Unix的伯父;它们共同的祖先是CTSS——由MIT于1963年间开发出来并在IBM 7094大型机上运行的一个系统。CTSS开发组后来又去开发了Multics,也就是Unix的直系祖先。IBM在剑桥大学组建了一个开发团队,为IBM 360/40——开发分时系统拥有分页MMU\footnote{要开发的机器和最初目标是开发定制微码的40系列,但是40机器不够强劲;生产部署转向了360/67系列}(在IBM系统上第一次)的改进型360系列机器。此后很多年,MIT和IBM程序员一直保持交流。新系统拥有一个与CTSS非常类似的用户界面,备有名为EXEC的shell和大量的实用程序,与Multics及后来Unix使用的实用程序非常类似。

从另一层意义看来,VM/CMS和Unix之间就像是游乐宫里的镜像。VM/CMS系统的统一性理念是虚拟机,由VM组件提供,每台虚拟机看起来就和运行其上的物理机是一样的。它们都是抢先式多任务处理,要么运行单用户的操作系统CMS,要么运行一个完整的多任务处理操作系统(如MVS,Linux或者VM自己)。虚拟机对应Unix的进程、后台程序和仿真器,它们之间的通信通过连接一个虚拟机的虚拟穿孔机和另一个虚拟机的虚拟读卡机来完成。另外,CMS内提供了一个叫作“CMS管道”的分层工具环境,直接取自Unix的管道模型,但在结构上已经扩展到可以支持多道输入和输出。

当虚拟机之间的通讯还没明确建立时,它们是完全隔绝的。操作系统具有和MVS一样的高可靠性、伸缩性和安全性,而且灵活性和易用性比MVS要好得多。另外,CMS中类似内核的部分不需要得到VM组件的信任,对它的操控是完全隔离的。

尽管CMS是面向记录的,但这些记录实际上等价于Unix文本工具所用的行。CMS的数据库更好地集成到CMS管道中,而Unix中的大多数数据库都独立于操作系统。近年来,CMS已经扩展到完全支持单一Unix规范。

CMS采用交互式和会话式UI风格,和MVS相差很远、但和VMS、Unix近似,大量使用一个叫XEDIT的全屏幕编辑器。

VM/CMS出现在客户端/服务器的区分之前,现今和IBM模拟终端一起几乎完全作为服务器操作系统使用。在Windows主宰桌面市场之前,VM/CMS不仅在IBM内部、而且也为大型机客户站点提供字处理服务和电子邮件服务——实际上,由于VM早就有提供成千上万用户的伸缩性,许多机器专门安装VM系统,只用它运行这些应用程序。

Rexx脚本语言支持编程的风格和shell、awk、perl或python有几分相似。因此,轻松编程(特别是系统管理员的轻松编程)在VM/CMS上非常重要。由于允许自由流通,管理员通常更愿意在虚拟机上而不是直接在裸机上运行产品级MVS,因此,人们很容易获得CMS并充分利用其灵活性(有一些CMS工具可允许访问MVS文件系统)。

VM/CMS在IBM中的历史同Unix在数字设备公司(DEC,他们生产了首次运行Unix的硬件)中的历史惊人地相似。IBM花了数年时间才明白自己的非正式分时系统的战略意义,与早期Unix社区行为非常类似的VM/CMS编程者社区就在那时兴起了。这些编程者分享想法和对系统的发现,最重要的是他们分享实用工具的源码。尽管IBM多次试图宣布VM/CMS结束,但这个社区——包括IBM自己的MVS系统开发者——坚持维持这个系统的存活。VM/CMS甚至也经过和Unix同样的循环,从事实上的开源到闭源,再回到开源——只不过没有Unix开源那样彻底罢了。

然而,VM/CMS所缺乏的是一个像C语言那样的东西。VM和CMS都用汇编语言编写,而且一直如此。和C最像的是PL/I的各种删节版,IBM用其进行系统编程,但从来没提供给客户。因此,尽管360系列已经升级到370系列、XA系列,最后到现在的z系列,这个操作系统却仍然截止在最初架构的框框中。

自2000年以来,IBM以前所未有的力度在大型机上推广VM/CMS系统——作为能同时容纳成千上万虚拟Linux机的手段。


\subsection{Linux}
Linux由Linus Torvalds于1991年发明,是1990年后出现的新学派开源Unix阵营(也包括FreeBSD、NetBSD、OpenBSD和Darwin)的领头羊,代表了整个阵营的设计方向。Linux的技术趋势可视为整个阵菅的典型。

Linux并不含任何来自原始Unix源码树的代码,但却是一个依照Unix标准设计、行为像Unix的操作系统。在本书的其余部分,我们重点强调的是Unix和Linux的延续性。无论从技术还是从关键开发者两个方面看,这种延续性都极其紧密——但此处,我们的重点是介绍Linux正在前进的几个方向,这些也正是Linux开始与“经典”Unix传统分道扬镳的标志。

Linux社区的许多开发者和积极分子都有夺取足量桌面用户市场份额的雄心壮志。这就使Linux的目标受众比“旧学派”Unix广泛得多,后者主要瞄准服务器和技术型工作站市场。这一点影响了Linux黑客设计软件的方式。

最明显的变化就是首选界面风格的转变。最初,设计Unix是为了在电传打字机和低速打印终端上使用。Unix生涯的大多数时间被用在字符型视频显示终端上,没有图形和色彩能力。大多数Unix程序员仍然固执地坚持使用命令行,即使大型终端用户应用程序很早就已经移植到基于X的GUI中了。这种状况也一直体现在Unix操作系统及应用程序的设计中。

另一方面,Linux的用户和开发者不断自我调整来消弭非技术用户对CLI的恐惧。他们比旧学派Unix、甚至同时代专有Unix更看重GUI及其工具的开发。其它开源Unix也在发生同样变化,变化虽小,但意义深远。

贴近终端用户的愿望使得Linux开发者比专有Unix更注重系统安装的平稳性和软件发布问题。由此产生的结果就是Linux的二进制包系统远比专有Unix的类似系统复杂,所设计的界面(2003年只取得部分成功)更合乎非技术型用户的口味。

Linux社区比旧学派Unix社区更希望将他们的软件变成能够联接其它环境的通用渠道。因此,Linux的特色就是能支持其它操作系统特有文件系统格式的读(更常见的是)写以及联网方式。Linux也支持同一硬件上的多重启动,并在Linux自身的软件中进行模拟。Linux的长期目标是包容;Linux模拟的目的就是为了吸收\footnote{ Linux模拟并包容的策略与一些竞争者实施的收买并扩展的策略所产生的结果显著不同。对于初学者,Linix并不会为了把用户锁定到增强版上而丧失与被模拟物的兼容性。}。

包容竞争者的目标加上贴近终端用户的动力,促使Linux开发者广泛吸收非Unix操作系统的设计理念,甚至到了使传统Unix显得十分孤立的地步。Linux应用程序采用Windows的.INI格式文件进行配置是一个小例子,我们将在第10章予以讨论。Linux 2.5采纳了扩展文件属性,加上其它一些特性,就可以模仿Macintosh的“资源分叉”语义。这也是写作本书时最近的一个重要例子。

\begin{quote}[Doug Mcllroy]
但是,Linux绘出“因为没安装对应的软件,所以打不开文件”这种Mac式诊断之时,就是Linux不再是Unix之日。
\end{quote}

其余的专有Unix(如Solaris、HP-UX、AIX等)都是为庞大IT预算设计的庞大产品。人们愿意掏钱努力优化,追求在高端的先进硬件上达到最大效能。因为很多Linux部件源自PC爱好者,所以强调用尽量少的资源做尽可能多的事。当专有Unix牺牲在低端硬件上的性能而专门为多处理器和服务器集群调优时,Linux核心开发者的选择很明确:不能为在高端硬件上获得最大性能收益,而在低端机器上增加复杂度和开销。

事实上,不难理解Linux用户社区中相当一部分人要从过时了的硬件中榨取有用东西的做法,就像1969年Ken Thompson对PDP-7一样。因此,Linux应用程序不得不始终保持瘦小精干的体态,而这是无法在专有Unix下的应用软件中体验到的。

这些趋向对Unix整体的发展产生了影响,我们将在第20章回顾这个话题。


\section{种什么籽,得什么果}
我们做过尝试,选择一个现在或者过去同Unix一争高下的分时系统进行比较,但似乎能入围者并不多。大多数(Multics、ITS、DTSS、TOPS-IO、TOPS-20、MTS、GCOS、MPE还有其它不下十几种)操作系统早已消亡,已渐渐从计算机领域的集体记忆中淡出。在我们已经讨论的操作系统中,VMS和OS/2也已濒临死亡,而MacOS已经被Unix的派生系统所收纳。MVS和VM/CMS仅仅局限于单一的专有大型机领域。只有独立于Unix传统外的Microsoft Windows系统还算是一个真正活着的竞争对手。

我们在第一章说明了Unix的优势,那当然是问题的部分答案:然而,把问题反换一下:究竟Unix竞争者的什么劣势让它们失败,其实更有说服力。

这些竞争对手最明显的通病是不可移植性。大部分1980年前的Unix竞争者都被拴到单个硬件平台上,随着这个硬件的消亡而消亡。为什么VMS可以坚持这么久?值得我们作为案例研究一个原因是:VMS成功地从最初的VAX硬件移植到了Alpha处理器(2003年正从Alpha移植到Itanium上)。MacOS也在1980年代后期成功完成了从摩托罗拉68000到PowerPC芯片的迁跃。微软的Windows处在计算机商品化将通用计算机市场扁平化到单一PC文化的时期,真是生逢其时。

自1980年起,对于那些要么被Unix压倒要么已经先Unix而去的其它系统,不断重现的另一个特有弱点是:不具备良好的网络支持能力。

在一个网络无处不在的世界,即使为单个用户设计的系统也需要多用户能力(多种权限组)——因为如果不具备这一点,任何可能欺骗用户运行恶意代码的网络事务都将颠覆整个系统(Windows宏病毒只是冰山一角)。如果不具备强大的多任务处理能力,操作系统同时处理网络传输和运行用户程序的能力将被削弱。操作系统还需要高效的IPC,这样网络程序彼此能够通信,并且能够与用户的前台应用程序通信。

Windows在这些领域具有严重缺陷却逃脱了惩罚,这仅仅因为它们在连网变得真正重要以前就形成了垄断地位,并拥有一群已经对机器经常崩溃和无数安全漏洞习以为常的用户。微软的这种地位并不稳定,Linux阵营正是利用这一点成功地(于2003年)在服务器操作系统市场取得了重大突破。

在个人机刚刚进入全盛时期的1980年左右,操作系统设计者认为Unix和其它传统的分时系统笨重、麻烦、不适合单用户个人机这个美丽新世界,而弃之不理——根本不顾GUI接口往往要求改造多任务处理能力,来适应不同窗口及其部件的绑定执行线程的事实。青睐客户端操作系统的趋势非常强烈,服务器操作系统就像已经逝去的蒸汽机时代的遗物一样遭到冷落。

但是,正如BeOS设计者们所注意到的那样,如果不实现某些近似通用分时系统的东西,就无法满足普遍联网的要求。单用户客户端操作系统在互联网世界里不可能繁荣。

这个问题促使客户端操作系统和服务器操作系统重新汇到了一起。首先,互联网时代之前的1980年代晚期,人们首次尝试通过局域网进行点对点联网,这种尝试暴露了客户端操作系统设计模式的不足:网络中的数据必须放到集合点上才能实现其享,因此如果没有服务器就做不到这一点。同时,人们对Macintosh和Windows客户端操作系统的体验也抬高了客户所能容忍的最低用户体验质量的门坎。

到了1990年,随着非Unix分时系统模型的实际消亡,客户端操作系统设计者还是拿不出来多少可能解决这一挑战的方案。他们可以吸收Unix(如MacOS X所做的),或通过一次一个补丁重复发明一些大致等价的功能(如Windows),或试图重新发明整个世界(如BeOS,但失败了)。但与此同时,各种开源Unix的类客户端能力不断增强,开始能够使用GUI并能在廉价的个人机上运行。

然而,这些压力在两类操作系统上并未达到上面描述所意味的那种对称。将服务器操作系统特性,如多用户优先权组和完全多任务处理,改装到客户端操作系统上非常困难,很可能打破对旧版本客户端的兼容性,而且通常做出的系统既脆弱又令人不满意,不稳定也不安全。另一方面,将GUI应用于服务器操作系统,所出现的问题却大部分可通过灵活处理和投入更廉价硬件资源得到解决。就像造房子一样,在坚实的地基上修理上层建筑当然要比更换地基而不破坏上层建筑来得容易。

除了拥有与生俱来的服务器操作系统体系优势外,Unix一直不明确界定自己的目标受众。Unix的设计者和实现者从不自认为已经完全清楚Unix的所有潜在用途。

因此,与之竞争的客户端操作系统把自己改造成服务器操作系统,Unix比起来更有能力把自己改造成客户端操作系统。尽管1990年代Unix的复苏有多方面的技术和经济因素,但正是这一点,使Unix在前述所有操作系统设计风格的讨论中最为抢眼。



\chapter{模块性:保持清晰,保持简洁}
\begin{flushright}
\begin{flashcard}[red!30]
软件设计有两种方式:一种是设计得极为简洁,没有看得到的缺陷;另一种是设计得极为复杂,有缺陷也看不出来。第一种方式的难度要大得多。

《皇帝的旧衣》,CACM 1981年2月

{\hfill —C. A. R. Hoare}
\end{flashcard}
\end{flushright}

代码划分的方法有一个自然的层次体系,随着程序员必须面对的复杂度日益增加,这个体系也在演变中。一开始,一切都是一大块机器码。最早的过程语言首先引入了用子程序划分代码的概念。后来,我们发明了服务程序库,在多个程序间其享公用函数。再后来,我们发明了独立地址空间和可以相互通信的进程。今天,我们习以为常地把程序系统分布在通过成千上万英里的网络电缆连接的多台主机上。

Unix的早期开发者也是软件模块化的先锋。在他们之前,模块化原则只是计算机科学的理论,还不是工程实践。在研究工程设计中模块经济性的《设计原理》(Design Rules)[Baldwin-Clark]这本探路性质的著作中,作者以计算机行业的发展为研究案例,并认为,相对硬件而言,Unix社区实际上第一个将模块分解法系统地应用到了生产软件中。毫无疑问,自从19世纪晚期人们采用标准螺纹以来,硬件的模块性就一直是工程技术的基石之一。

模块化原则在这里展开来说就是:要编写复杂软件又不至于一败涂地的唯一方法,就是用定义清晰的接口把若干简单模块组合起来,如此一来,多数问题只会出现在局部,那么还有希望对局部进行改进或优化,而不至于牵动全身。

相对其他程序员而言,Unix程序员骨子里的传统是:更加笃信重视模块化、更注重正交性和紧凑性等问题。
\begin{quote}[Ken Thompson]
早期的Unix程序员擅长模块化是因为他们被迫如此。操作系统就是一堆最复杂的代码。如果没有良好的架构,操作系统就会崩溃。在人们早期开发Unix时就犯过几次这种错,代码不得不全数报废。虽然大家可以把这些怪罪于早期的(非结构化)C语言,但主要还是因为操作系统太复杂,太难编写。所以,我们既需要改进工具(如C语言的结构化),也需要养成使用工具的好习惯(如Rob Pike提出的编程原理),这样才能应对这种复杂性。
\end{quote}

早期的Unix黑客为此在很多方面进行了艰苦的努力。1970年的时候,函数调用开销昂贵,不是因为调用语句太复杂(PL/1.Alg-\\o),就是因为编译器牺牲了调用时间来优化其它因素,如快速内层循环(fast inner loops)。这样,代码往往就写成一大块。Ken和其他早期Unix开发者知道模块化是个好东西,但是他们记得PL/1的经验,不愿意编写小函数,怕影响性能。
\begin{quote}[Steve Johnson]
Dennis Ritchie告诉所有人C中的函数调用开销真的很小很小,极力倡导模块化。于是人人都开始编写小函数,搞模块化。然而几年后,我们发现在PDP-11中函数调用开销仍然昂贵,而VAX代码往往在“CALLS”指令上花费掉50\%的运行时间。Dennis对我们撒了谎!但为时已晚,我们已经欲罢不能……
\end{quote}

今天所有的编程者,无论是不是Unix下的程序员,都被教导要在程序的子程序层上进行模块化。有些人学会了在模块或抽象数据类型层上玩这一手,并称之为“良好的设计”。设计模式运动正在进行一项宏伟的努力,希望更进一步,找到成功的设计抽象原则,以组织大规模程序的结构。

将这些问题作一个更好的划分是一个有价值的目标,而且到处都可以找到有关模块划分的优秀方法。我们不期望太深入地涵盖与程序模块化相关的所有问题:首先,因为该论题本身就足够写整整一本(或好几本)书:其次,因为这是一本关于Unix编程艺术的书。

我们在此会更详细地分析Unix传统是如何教导我们遵循模块化原则的。本章中的例子仅限于进程单元内。我们将在第7章分析其它一些情形,那里,程序划分为几个协作进程是个不错的想法,我们还将讨论实现这种划分所采用的具体技术。


\section{封装和最佳模块大小}
模块化代码的首要特质就是封装。封装良好的模块不会过多向外部披露自身的细节,不会直接调用其它模块的实现码,也不会胡乱共享全局数据。模块之间通过应用程序编程接口(API)——一组严密、定义良好的程序调用和数据结构来通信。这就是模块化原则的内容。

API在模块间扮演双重角色。在实现层面,作为模块之间的滞塞点(choke point),阻止各自的内部细节被相邻模块知晓;在设计层面,正是API(而不是模块间的实现代码)真正定义了整个体系。

有一种很好的方式来验证API是否设计良好:如果试着用纯人类语言描述设计(不许摘录任何源代码),能否把事情说清楚?养成在编码前为API编写一段非正式书面描述的习惯,是一个非常好的办法。实际上,一些最有能力的开发者,一开始总是定义接口,然后编写简要注释,对其进行描述,最后才编写代码——因为编写注释的过程就阐明了代码必须达到的目的。这种描述能够帮助你组织思路,本身就是十分有用的模块说明,而且,最终你可能还想把这些说明做成路标文档(roadmap document),方便以后的人阅读代码。

模块分解得越彻底,每一块就越小,API的定义也就越重要。全局复杂度和受bug影响的程度也会相应降低。软件系统应设计成由层次分明的嵌套模块组成,而且每个层面上的模块粒度应降至最低,计算机科学领域从二十世纪七十年代起就已经渐渐明白了这个道理(有[Parnas]之类文章为证)。

然而,也可能因过度划分造成模块太小。证据[hatton97]如下:绘制一张缺陷密度和模块大小关系图,发现曲线呈U形,凹面向上(见图4-1)。跟中间大小的模块相比,模块过大或者过小都和更多的bug相关联。另一个观察这些同样数据的方法是,绘制每个模块的代码行数和bug的关系曲线图。曲线看上去大致成对数上升至平坦的“最佳点”(对应缺陷密度曲线中的最小值),然后按代码行数的平方上升(这正是人们根据Brook定律对整个曲线的直观预期)。\footnote{ Brook定律预言道:对一个已经延期的项目,增加程序员只会使该项目更加延期。更一般地,这个定律预言:项目成本和错误率按程序员人数的平方增长。}

\begin{fig}{缺陷数量和缺陷密度与模块大小的定性曲线图}
\label{fig:缺陷数量和缺陷密度与模块大小的定性曲线图}
\end{fig}

在模块很小时,bug发生率也出乎意料地增多,这在大量以不同语言实现的各种系统中均是如此。Hatton曾经提出过一个模型,将这种非线性同人类短期记忆的记忆块大小相比较\footnote{在Hatton的模型中,程序员可以短期记忆的最大模块大小的微小差别对其他的效率具有倍增效应。这可能是Fred Brooks等人对效率的数量级(甚至更大)变化规律研究所作的最重要贡献。}。这种非线性的另一种解释是,模块小时,几乎所有复杂度都在于接口;想要理解任何一部分代码前必须理解全部代码,因此阅读代码非常困难。我们将在第7章讨论程序划分的更高级形式;在那里,当组件进程规模更小以后,接口协议的复杂度也就决定了系统的整体复杂度。

用非数学术语来说,Hatton的经验数据表明,假设其它所有因素(如程序员能力)都相同,\reduline{200到400之间逻辑行的代码是“最佳点”,可能的缺陷密度达到最小}。\footnote{也就是不考虑注释,一个代码模块(文件)最好小于500行。我以后会按照这个原则来python编程,看看合不合适。}这个大小与所使用的语言无关——这个结论有力支持了本书中其它地方提出的建议,即尽可能用最强大的语言和工具编程。当然,不能完全照搬这些具体数字。根据分析人员对逻辑行的理解以及其它偏好(比如注释是否剔除)的不同,代码行的统计方法会有较大差别。根据经验,Hatton建议逻辑行与物理行之间为两倍的折算率,即最佳物理行数建议应在400至800行之间。


\section{紧凑性和正交性}
具有最佳尺寸的模块并不意味着代码有高质量。由于受到同样的人类认知限制,语言和API(如程序库集和系统调用)也会产生Hatton U形曲线。

因此,在设计API、命令集、协议以及其它让计算机工作的方法时,Unix程序员已经学会了认真考虑另外两个特性:紧凑性和正交性。

\subsection{紧凑性}
紧凑性就是一个设计是否能装进人脑中的特性。测试软件紧凑性的一个很实用的好方法是:有经验的用户通常需要操作手册吗?如果不需要,那么这个设计(或者至少这个设计的涵盖正常用途的子集)就是紧凑的。

紧凑的软件工具和顺手的自然工具一样具有同样的优点:让人乐于使用,不会在你的想法和工作之间格格不入,使你工作起来更有成效——完全不像那些蹩脚的工具,用着别扭,甚至还会把你弄伤。

紧凑不等于“薄弱”。如果一个设计构建在易于理解且利于组合的抽象概念上,则这个系统能在具有非常强大、灵活的功能的同时保持紧凑。紧凑也不等同于“容易学习”:对于某些紧凑设计而言,在掌握其精妙的内在基础概念模型之前,要理解这个设计相当困难;但一旦理解了这个概念模型,整个视角就会改变,紧凑的奥妙也就十分简单了。对很多人来说,Lisp语言就是这样一个经典的例子。

\begin{quote}[Ken Arnold]
紧凑也不意味着“小巧”。即使一个设计良好的系统,对有经验的用户来说没什么特异之处、“一眼”就能看懂,但仍然可能包含很多部分。
\end{quote}

极少有绝对意义上紧凑的软件设计,不过从宽松一些的意义上,许多软件设计还是相对紧凑的。他们有一个紧凑的工作集:一个功能子集,能够满足专家用户80\%以上的一般需求。实际上,这类设计通常只需要一个参考卡(reference card)或备忘单(cheat sheet),而不是一本手册。相对严格紧凑性而言,我们将此类设计称为\textbf{“半紧凑型”}。

也许最好还是用例子来阐明这个概念。Unix系统调用API是半紧凑的,而C标准程序库无论如何都算不上是紧凑的。Unix程序员很容易记住满足大多数应用编程(文件系统操作、信号和进程控制)的系统调用子集,但现代Unix上的C标准库却包括成百上千个条目,如数学函数等,一个程序员不可能把所有这些都记在脑中。

《魔数七,加二或减二:人类信息处理能力的局限性》(The Magical Number Seven,Plus or Minus Two: Some Limits on Our Capacity for Processing Information[Miller])是认知心理学的基础性文章之一(顺带一句,这也正是美国本地电话号码只有七位的原因)。这篇文章表明,人类短期记忆能够容纳的不连续信息数就是七,加二或减二。这给了我们一个评测API紧凑性的很好的经验法则:编程者需要记忆的条目数大于七吗?如果大于七,则这个API不太可能算是严格紧凑的。

在Unix工具软件中,make(1)是紧凑的;autoconf(1)和automake(1)则不是。在标记语言中,HTML是半紧凑的,DocBook(我们将在第18章讨论这个文件标记语言)则不是。man(7)宏是紧凑的,troff(1)标记则不是。

在通用编程语言中,C和Python是半紧凑的;Perl,java,Emacs Lisp,和shell则不是(尤其是严格的shell编程,要求你必须知道其他六个工具,如sed(1)和awk(1)等)。C++是反紧凑性的——该语言的设计者已经承认,他根本不指望有哪个程序员能够完全理解C++。有些不具备紧凑性的设计具有足够的内部功能冗余,结果程序员通过选择某个工作的语言子集就能够搞出能满足80\%普通任务的紧凑方言。比如,Perl就有这种伪紧凑性。此类设计存在一个固有的陷阱:当两个程序员试图就一个项目进行交流时,他们可能会发现,对工作子集的不同选择成了他们理解和修改代码的巨大障碍。

然而,不紧凑的设计也未必注定会灭亡或很糟糕。有些问题域简直是太复杂了,一个紧凑的设计不可能有如此跨度。有时,为了其它优势,如纯性能和适应范围等,也有必要牺牲紧凑性。troff标记就是一个很好的例子,BSD套接字API也是如此。把紧凑性作为优点来强调,并不是要求大家把紧凑性看作一个绝对要求,而是要像Unix程序员那样:合理对待紧凑性,设计中尽量考虑,决不随意抛弃。

\subsection{正交性}
正交性是有助于使复杂设计也能紧凑的最重要特性之一。在纯粹的正交设计中,任何操作均无副作用;每一个动作(无论是API调用、宏调用还是语言运算)只改变一件事,不会影响其它。无论你控制的是什么系统,改变每个属性的方法有且只有一个。

显示器就是正交控制的。你可以独立改变亮度而不影响对比度,而色彩平衡控制(如果有的话)也独立于前两个属性。想象一下,如果按亮度按钮会影响色彩平衡,这样的显示器调节起来会有多么困难:每次调节亮度之后还得调节色平衡进行补偿。更糟糕的是,如果对比度控制也影响色平衡,那么要改变对比度或色平衡同时保持另一个不变,你必须严格按照正确的方法同时调节两个旋钮。

非正交的软件设计不胜枚举。例如,代码中常见的一类设计错误出现在从某一(源)格式到另一(目标)格式进行数据读取和解析过程中。如果设计者想当然地认为源格式总是存储在某个磁盘文件中,那么他可能会编写一个打开和读取指定文件名的转换函数。但是,通常情况下,输入也完全有可能就是一个文件句柄。如果转换函数是正交设计的,例如,无需额外打开一个文件,那么以后当转换函数要处理来自标准输入、网络套接字或其它来源的数据流时,可能会省事一些。

人们通常认为Doug Mcllroy“只做好一件事”的忠告是针对简单性的建议。但是,这句话也暗含了对正交性至少同等程度的强调。

如果一个程序做好一件事之外,顺带还做其它事情的时候既不增加系统的复杂度也不会使系统更易产生bug,就没什么问题。我们将在第9章检视一个名为ascii的程序,这个程序能打印ASCII字符的同名符,包括十六进制值、八进制值和二进制值;其副作用是可以对0-255范围内的数字进行快速进制转换。这第二个作用并不算违反正交性,因为所有支持该用途的特性全部是主功能所必需的,而且这样也没有增加程序文档化或维护的难度。

如果副作用扰乱了程序员或用户的思维模式,带来种种不便甚至可怕的结果(最好还是忘掉吧),这就是出现了非正交性问题。尤其在没有忘记这些副作用时,你总要被迫做额外工作来抑制或修正它们。

《程序员修炼之道》(The Pragmatic Programmer)[Hunt-\\Thomas]一书中对正交性以及如何达到正交性有精彩的讨论。正如该书所指出的,正交性缩短了测试和开发的时间,因为那种既不产生副作用也不依赖其它代码副作用的代码校验起来要容易得多——需要测试的情况组合要少得多。如果正交性代码出现问题,把它替换掉而不影响系统其余部分也很容易做到。最后,正交性代码更容易文档化和复用。

\textbf{重构}(refactoring)概念是作为“极限编程(Extreme Programming)”学派的一个明确思想首次出现的,跟正交性紧密相关。重构代码就是改变代码的结构和组织,而不改变其外在行为。当然,自从软件领域诞生之日起,软件工程师就一直在从事这项工作,给这种做法命名并把重构的一套技术方法确定下来,则非常有效地帮助了人们集中思路。因为重构概念与Unix设计传统关注的核心问题非常契合,所以Unix开发者很快就吸收了这一术语和它的思想\footnote{在这一概念的奠基性著作“重构》(Refactoring)[Fowler]一书中,作者差一点就道出了“重构的原则性目标就是提高正交性”的天机。但是由于缺少这个概念,他只能从几个不同的方向接近这个思想:比如消除重复代码和各种“坏味道”,大部分就是指一些违背正交性的做法。}。

Unix的基本API设计在正交性方面虽不完美,但也颇为成功。比如,我们理所当然地认为能够打开文件进行写入操作,而无需为此进行排他锁定。并不是所有的操作系统都如此优雅。老式(System III)的信号就不是正交的,因为信号接收的副作用是把信号处理器(signal handler)重置成缺省的“接收即崩溃”(die-on-receipt)。许多大幅修正也不是正交的,如BSD套接字API,还有一些更大的修正也不是正交的,如X window系统的绘图库。

但是,就整体而言,Unix API是一个很好的例子:否则,将不仅不会、也不可能这么广泛地被其它操作系统上的C库效仿。所以,即便不是Unix程序员,Unix API也值得学习,因为从中可以学到一些关于正交性的东西。


\subsection{SPOT原则}
《程序员修炼之道》(The Pragmatic Programmer)针对一类特别重要的正交性明确提出了一条原则——“不要重复自身(Don't Repeat Yourself)”,意思是说:任何一个知识点在系统内都应当有一个\textbf{唯一}、明确、权威的表述。在本书中,我们更愿意根据Brian Kemighan的建议,把这个原则称为“真理的单点性(Single Point of Truth)”或者SPOT原则。

重复会导致前后矛盾、产生隐微问题的代码,原因是当你修改重复点时,往往只改变了一部分而并非全部。通常,这也意味着你对代码的组织没有想清楚。

常量、表和元数据只应该声明和初始化一次,并导入其它地方。无论何时,重复代码都是危险信号。复杂度是要花代价的,不要为此重复付出。

通常,可以通过重构去除重复代码;也就是说,更改代码的组织而不更改核心算法。有时重复数据好像无法避免,但碰到这种情况时,下面问题值得你思考:

\begin{itemize}
\item 如果代码中含有重复数据是因为在两个不同的地方必须使用两个不同的表现形式,能否写个函数、工具或代码生成程序,让其中一个由另一个生成,或两者都来自同一个来源?
\item 如果文档重复了代码中的知识点,能否从部分代码中生成部分文档,或者反之,或者两者都来自同一个更高级的表现形式?
\item 如果头文件和接口声明重复了实现代码中的知识点,是否可以找到一种方法,从代码中生成头文件和接口声明?
\end{itemize}

数据结构也存在类似的SPOT原则:“ 无垃圾,无混淆”(No junk,no confusion)。“无垃圾”是说数据结构(模型)应该最小化,比如,不要让数据结构太通用,居然还能表示不可能存在的情况。“无混淆”是指在真实世界中绝对明确清晰的状态在模型中也应该同样明确清晰。简言之,SPOT原则就是提倡寻找一种数据结构,使得模型中的状态跟真实世界系统的状态能够一一对应。

更深入Unix传统一步,我们可以从SPOT原则得出以下推论:

\begin{itemize}
\item 是不是因为缓存了某个计算或查找的中间结果而复制了数据?仔细考虑一下,这是不是一种过早优化;陈旧的缓存(以及保持缓存同步所必需的代码层)是滋生bug的温床,而且如果(实际经常是)缓存管理的开销比预想的要高,甚至可能降低整体性能\footnote{不良缓存的一个典型例子是csh(1)rehash指令。欲了解详情可键入man 1 csh。另一个例子参见12.4.3。}。
\item 如果有大量重复的样板代码,是不是可以用单一的更高层表现形式生成这些代码、然后通过提供不同的细调选项生成不同个例呢?
\end{itemize}

到此,读者应该能看出一个轮廓逐渐清晰的模式。

在Unix世界中,SPOT原则作为一个统一性理念很少被明确提出过——但是Unix传统中SPOT原则在各种形式的代码生成器中充分体现。我们将在第9章讨论这些技法。

\subsection{紧凑性和强单一中心}
要提高设计的紧凑性,有一个精妙但强大的方法,就是围绕“解决一个定义明确的问题”的强核心算法组织设计,避免臆断和捏造。

\begin{quote}[Doug Mcllroy]
形式化往往能极其明晰地阐述一项任务。如果一个程序员只认识到自己的部分任务属于计算机科学一些标淮领域的问题一一这儿来点深度优先搜索,那儿来点快速排序一一是不够的。只有当任务的核心能够被形式化,能够建立起关于这项工作的明确模型时,才能产生最好的结果。当然,最终用户没有必要理解这个模型。统一核心的存在本身就给人很舒服的感觉,不会出现像在使用看似无所不能的瑞士军刀式程序中非常普遍的“他们到底为什么这样做”的情形。
\end{quote}

这是Unix传统中常常被忽视的一个优点。其实,Unix许多非常有效的工具都是围绕某个单一强大算法直接转换的一个瘦包装器(thin wrapper)。

最清楚的例子也许就是diff(1)——一个Unix用于报告相关文件不同之处的工具。这个工具及其搭裆patch(1)已经成为当代Unix网络分布式开发风格的核心。diff的可贵性之一在于它很少标新立异。它既没有特殊情况,也没有令人痛苦的边界条件,因为它使用一个简单、从数学上看很可靠的序列比较方法。这导致了以下结果:

\begin{quote}[Doug Mcllroy]
由于采用了数学模型和可靠的算法,Unix diff和其仿效者形成鲜明的对比。首先,diff的核心引擎小巧可靠,没有一行代码需要维护。其次,结果清晰一致,不会出现试探法可能带来的意外。
\end{quote}

这样,使用diff的人无需完全理解核心算法,就能对diff在任何给定条件下的行为形成一种直觉。在Unix中,其它通过强大核心算法达到这种特定清晰性的著名例子非常多:

\begin{itemize}
\item 通过模式匹配从文件中挑选文本行的grep(1)实用程序是一个简单包装器,围绕正则表达式( regular-expression)模式的形式代数问题(参见8.2.2部分的讨论)。如果它没有这个一致的数学模型,它可能就会很像最古老的Unix中原始的glob(1)设计,只是一堆无法组合在一起的专门通配符。
\item 用于生成语法解析器的yacc(1)实用程序是围绕LR(1)语法形式理论的瘦包装器。它的搭档——词法分析生成器lex(1),则是围绕不确定有限态自动机的瘦包装器。
\end{itemize}

以上这三个程序都极少出bug,大家认为它们绝对理所当然地应该正确运行,而且它们也非常紧凑,程序员用起来得心应手。这些良好性能只有一部分归功于长期服务和频繁使用所产生的改进,绝大部分还是因为建立在强大且被证明为正确的算法核心上,它们从一开始就无需多少改进。

与形式法相对的是\textbf{试探法}——凭经验法则得出的解决方案,在概率上可能正确,但不一定总是正确。有时我们使用试探法是因为不可能找到绝对正确的解决方案。例如,想一想垃圾邮件过滤:一个算法上完美的垃圾邮件过滤器需要完全解决自然语言的理解问题。其它一些时候,我们使用试探法是因为所有已知的形式上正确的方法开销都贵得难以想象。虚拟内存管理就是这样一个例子:虽然确实存在接近完美的解决方案,但是它们需要的运行时间太长,以至其相比试探法所能获得的任何理论上的收益优势完全被抵消掉了。

试探法的问题在于这种方案会增生出大量特例和边界情况。通常情况下,当试探法失效,如果没什么其它方法的话,你必须采用某种恢复机制作为后备。复杂度一增加,所有常见的问题都会随之而来。为了折衷,一开始就要小心使用试探法。始终要记着问一问,如果试探法以增加代码复杂性为代价,根据会获得的性能来判断一下是否值得这么做——不要猜想可能产生的性能差异,在做出决定前应该实际衡量一下。

\subsection{分离的价值}
本书开头,我们引用了禅的“教外别传,不立文字”。这不仅是为了追求风格上的异国情调,而是因为Unix的核心概念一向都有清瘦如禅般的简洁性,在围绕这些核心概念发生的历史事件中如影随形,熠熠生辉。这种特性也反映在Unix的基础性著作中,如《C程序设计语言》(C Programming Language)[Kemighan-Ritchiel]和向世人介绍Unix的1974年CACM论文。文中最常被人引用的一句话是这样的:“......限制不仅提倡了经济性,而且某种程度上提倡了设计的优雅”。要达到这种简洁性,尽量不要去想一种语言或操作系统最多能做多少事情,而是尽量去想这种语言或操作系统最少能做的事情——不是带着假想行动,而是从零开始(禅称为“初心”( beginner's mind)或者叫“虚一心”(empty mind))。

要达到紧凑、正交的的设计,就从零开始。禅教导我们:依附导致痛苦;软件设计的经验教导我们:依附于被人忽略的假定将导致非正交、不紧凑的设计,项目不是失败就是成为维护的梦魇。

禅授超然,可以得教化,去苦痛。Unix传统也从产生设计问题的特定、偶然的情形讲授分离的价值。抽象、简化、归纳。因为我们编制软件是为了解决问题,所以我们不可能完全超然于问题之外——但是值得费点心思,看看可以抛弃多少先入之见,看看这样做能不能使设计变得更紧凑、更正交。这样做下来,代码复用经常由此变为可能。

关于Unix和禅的关系的笑话同样也是Unix传统中一个仍然鲜活的部分\footnote{要了解Unix和禅交融的最近例子,可参阅附录D。}。这绝非偶然。


\section{软件是多层的}
一般来说,设计函数或对象的层次结构可以选择两个方向。选择何种方向、何时选择,对代码的分层有着深远的影响。

\subsection{自顶向下和自底向上}
一个方向是自底向上,从具体到抽象——从问题域中你确定要进行的具体操作开始,向上进行。例如,如果为一个磁盘驱动器设计固件,一些底层的原语可能包括“磁头移至物理块”、“读物理块”、“写物理块”、“开关驱动器LED”等。

另一个方向是自顶向下,从抽象到具体——从最高层面描述整个项目的规格说明或应用逻辑开始,向下进行,直到各个具体操作。这样,如果要为一个能处理不同介质的大容量存储控制器设计软件,可以从抽象的操作开始,如“移到逻辑块”、“读逻辑块”、“写逻辑块”、“开关状态指示”等。这和以上命名方式类似的硬件层操作的不同之处在于,这些操作在设计时就考虑到要能在不同的物理设备间通用。

以上这两个例子可视为同一类硬件的两种设计方式。在这种情况下,你的选择无非是两者取其一:要么抽象化硬件(这样,对象封装了实际事物,程序只不过是针对这些事物的操控动作列表),要么围绕某个行为模型组织代码(然后在行为逻辑流中嵌入实际执行的硬件操控动作)。

许多不同的情形中都会出现类似的选择。设想你在编写MIDI音序器软件,可以围绕最项层(音轨定序)或围绕最底层(切换音色或采样以及驱动波形发生器)组织代码。

有一个非常具体的方法可以考量二者的差异,那就是问问设计是围绕主事件循环(常常具备与其非常接近的高级应用逻辑)组织,还是围绕主循环可能调用的所有操作的服务库组织代码。自顶向下的设计者通常先考虑程序的主事件循环,以后才插入具体的事件。自底向上的设计者通常先考虑封装具体的任务,以后再按某种相关次序把这些东西粘合在一起。

如果要举一个更大的例子,可以考虑网页浏览器的设计。网页浏览器的顶层设计是对浏览器预期行为的规格说明:可以解析什么类型的URL(http:,ftp:还是file:),可以渲染哪些类型的图像,是否可以或者带哪些限制来支持Java或Javascript等等。与这个顶层意图相对应的实现层是浏览器的主事件循环;在每个周期内,这个循环等待、收集、分派用户的动作(例如点击网页链接或在某个域内键入字符)。

但是,网页浏览器要正常工作还必须调用大量域原语操作。其中一组跟建立连接、通过连接发送数据和接收响应有关。另一组则是浏览器将使用的GUI工具包操作。然而,可能还有第三组集合,即“将接收的HTML从文本转换为文档对象树”的解析机制。

从哪端开始设计相当重要,因为对端的层次很可能受到最初选择的限制。尤其是,如果程序完全自顶向下设计,你很可能发现自己陷入非常不舒服的境地,应用逻辑所需要的域原语和真正能实现的域原语无法匹配。另一方面,如果程序完全自底向上设计,很可能发现自己做了许多与应用逻辑无关的工作——或者,就像你想要造房子,却仅仅只设计了一堆砖头。

自从二十世纪六十年代有关结构化程序设计的论战后,编程新手往往被教导以“正确的方法是自顶向下”:逐步求精,在拥有具体的工作码前,先在抽象层面上规定程序要做些什么,然后用实现代码逐步填充。当以下三个条件都成立时,自顶向下不失为好方法:(a)能够精确预知程序的任务,(b)在实现过程中,程序规格不会发生重大变化,(c)在底层,有充分自由来选择程序完成任务的方式。

这些条件容易在相对接近最终用户和软件设计的较上层——应用软件编程——中得到满足。但即便如此,这些前提也常常满足不了。在用户界面经过最终用户测试前,别指望能提前知道什么算是字处理软件或绘图程序的“正确”行为方式。如果纯粹地自顶向下编程,常常产生在某些代码上的过度投资效应,这些代码因为接口没有通过实际检验而必须废弃或重做。

为了应对这种情况,出于自我保护,程序员尽量双管齐下——一方面以自顶向下的应用逻辑表达抽象规范,另一方面以函数或库来收集底层的域原语,这样,当高层设计变化时,这些域原语仍然可以重用。

Unix程序员继承了一个居于系统程序设计核心的传统,在这一传统中,底层的原语是硬件层操作,后者特性固定且极其重要。因此,出于后天学得的本能,Unix程序员更倾向于自底向上的编程方式。

无论是否是系统程序员,当你用一种探索的方式编程,想尽量领会你还没有完全理解的软件、硬件抑或真实世界的现象时,自底向上法看起来也会更有吸引力。它给你时间和空间去细化含糊的规范,同时也迎合了程序员身上人类通有的懒惰天性——当必须丢弃和重建代码时,与之相比,如果用自顶向下的设计,需要抛弃的代码往往更多。

因此实际代码往往是自顶向下和自底向上的综合产物。同一个项目中经常同时兼有自顶向下的代码和自底向上的代码。这就导致了“胶合层”的出现。



\subsection{胶合层}
当自顶向下和自底向上发生冲突时,其结果往往是一团糟。顶层的应用逻辑和底层的域原语集必须用胶合逻辑层来进行阻抗匹配(impedance match)。

Unix程序员几十年的教训之一就是:胶合层是个挺讨厌的东西,必须尽可能薄,这一点极为重要。胶合层用来将东西粘在一起,但不应该用来隐藏各层的裂痕和不平整。

在网页浏览器这个例子中,胶合层包括渲染代码(rendering code),它使用GUI域原语将从发过来的HTML中解析出的文档对象绘制成平面的可视化表达——即显示缓冲区中的位图。渲染代码作为浏览器中最易产生bug的地方而臭名昭著。它的存在,是为了解决HTML觯析(因为形式不良的标记太多了)和GUI工具包(可能未必具有真正需要的原语)中存在的问题。

网页浏览器的胶合层不仅要协调内部规范和域原语集,而且还要协调不同的外部规范:HTTP标准化的网络行为、HTML文档结构、各种图形和多媒体格式以及用户对GUI的行为预期。

一个容易产生bug的胶合层还不是设计所能遇到的最坏命运。如果设计者意识到胶合层的存在,并试图围绕自身的一套数据结构或对象把胶合层组织成一个中间层,结果却导致出现两个胶合层——一个在中间层之上,另一个在中间层之下。那些天资聪慧但经验不足的程序员特别容易掉进这种陷阱:他们将每种类别(应用逻辑、中间层和域原语集)的基本集都做得很好,就像教科书上的例子一样漂亮,结果却因为整合这些漂亮代码所需的多个胶合层越来越厚,而最终在其中苦苦挣扎。

薄胶合层原则可以看作是分离原则的升华。策略(应用逻辑)应该与机制(域原语集)清晰地分离。如果有许多代码既不属于策略又不属于机制,就很有可能除了增加系统的整体复杂度之外,没有任何其它用处。


\subsection{实例分析:被视为薄胶合层的C语言}
C语言本身就是一个体现薄粘合层有效性的良好例子。

上个世纪九十年代后期,Gerrit Blaauw和Fred Brooks在《计算机体系:概念和演化》(Computer Architecture:  Concepts and Evolution)  [BlaauwBrooks] 一书中提出,每一代计算机的体系结构,从早期的大型机到小型机、工作站再到PC,都在趋近同一种形式。技术年代越靠后,设计越接近Blaauw和Brooks所称的“经典体系”:二进制表示、平面地址空间、内存和运行期存储(寄存器)的区分、通用寄存器、定长字节的地址解析、双地址指令、高位字节优先\footnote{高位字节优先(big-endian)和低位字节优先(little-endian)术语指比特在机器字内解析顺序的架构选择。虽然没有规范的位置,但你在网上搜索“On Holy Wars and a Plea for Peace”,会找到有关这个论题的一篇经典而有趣的文章。}以及大小一致为4位或6位整数倍(6位系列现在已经不存在了)的数据类型。

Thompson和Ritchie将C语言设计成一种结构汇编程序,可为理想化的处理器和存储器体系服务,他们期望这种体系能有效建立在大多数普通计算机上。幸运的是,他们的理想化处理器模型机是PDP-11——一款设计非常成熟、优雅的小型机,非常接近Blaauw \& Brook的经典体系。凭借敏锐的判断力,Thompson和Rithcie拒绝在其语言中加入PDP-11不匹配的少数特性(比如低位优先字节序)中的绝大多数\footnote{人们普遍以为自增自减符特性被C语言采用是因为它们代表了PDP-11的机器指令,这其实没有根据。按照Dennis Ritchie的说法,在PDP-11出现之前,这些操作符就在前辈B语言中出现了。}。

PDP-11成为接下来几代微处理器架构的重要模型。结果证明,C语言的基本抽象相当优美地反映出了经典体系。这样,C语言一开始就非常适合微处理器,而且随着硬件更紧密地向经典架构靠拢,C语言不仅没有随其假设的过时而失去价值,反而更加适合微处理器了。这种硬件向经典体系会聚的非常著名的例子就是:1985年后Intel的386机器用平面存储地址空间代替了286糟糕的分段内存寻址。跟286相比,纯C语言实际上更适合386。

计算机架构的实验性时代在二十世纪八十年代中期结束,同期,C语言(和近亲后代C++)作为通用程序设计语言所向无敌,两者在时间上并非巧合。C语言,作为经典体系之上一个薄而灵活的胶合层,在经过了20年后,现在看来似乎可以算是其定位的结构汇编程序中的最佳设计。除了紧凑、正交和分离(与最初设计时的机器架构分离),C语言还拥有我们将在第6章讨论的透明性这一重要特性。C语言之后的少数语言设计(是否比C语言更好还有待证明),为了不被C语言所吞并,不得不进行大的改动(比如引进
垃圾收集功能等),以和C语言保持功能上的足够距离。

这段历史很值得回味和了解,因为C语言向我们展示了一个清晰、简洁的最简化设计能够多么强大。如果Thompson和Ritchie当初没有这么明智,他们设计的语言也许能完成更多任务,但要依赖更强的前提,永远都无法满意地从原始的硬件平台移植出去,也必将随着外部世界的改变而消亡。但相反的是,C语言一直生机勃勃——而Thompson和Ritchie所树立的榜样从此影响了Unix的开发风格。正如法国作家、冒险家、艺术家和航空工程师安东尼•德•圣埃克苏佩里(Antoine de Saint-Exupéry)在论飞机设计时所说的:“La perfection est atteinte non quand il ne reste rien à ajouter, mais quand il ne reste
rien à enlever”(完美之道,不在无可增加,而在无可删减)。

Ritchie和Thompson坚信该格言。即便当早期Unix软件所受的种种资源限制得到缓解之后很久,他们仍努力使C语言成为尽可能薄的“硬件之上的胶合层”。

\begin{quote}[Mike Lesk]
以前每当我要求在C语言中加一些特别奢侈的功能时,Dennis就对我说,“如果你需要PL/1,你知道到哪里去找”。他不必和那些说着:“但我们需要在销售材料中加一个卖点”的销售人员打交道。
\end{quote}

在标准化之前最好先有个有效的参考实现,C语言的历史在这方面教了我们一课。我们将在第17章讨论C语言和Unix标准的发展时再谈这个话题。


\section{程序库}
Unix编程风格强调模块性和定义良好的API,它所产生的影响之一就是:强烈倾向于把程序分解成由胶合层连接的库集合,特别是共享库(在Windows和其它操作系统下叫做“动态连接库”(DLL)。

如果谨慎而聪明地处理设计,那么常常可以将程序划分开来,一个是用户界面处理的主要部分(策略),另一个是服务例程的集合(机制),中间不带任何胶合层。当程序要进行图形图像、网络协议包、硬件接口控制块等多种数据结构的具体操作处理时,这种方法特别合适。《可复用库架构的守则和方法》(The Discipline and Method Architecture for Reusable Libraries)[Vo]一书中收集了Unix传统中关于体系的一些不错的通用性建议,尤其适合这种程序库的资源管理。

在Unix下,通常是清晰地划分出这种层次,并把服务程序集中在一个库中并单独文档化。在这样的程序中,前端专门解决用户界面和高层协议的问题。如果设计更仔细一些,可以将原始的前端分离出来,用适于不同用途的其它部件代替。通过实例研究,你还会发现其它一些优势。

这捎带引起了一个小问题。在Unix世界里,作为“程序库”发布的库必须携带练习程序(exerciser program)。

\begin{quote}[Henry Spencer]
API应该随程序一起提供,反之亦然。如果一个API必须要编写C语言代码来使用,考虑到C代码不能方便地从命令行调用,刚这个API学习和使用起来就更困难。反之,如果接口唯一开放、文档化的形式是程序,而无法方便地从C程序中调用这些接口,也会非常痛苦——例如,老版本Linux中的route(1)。
\end{quote}

除了学习起来更容易外,库的练习程序常常可以作为优秀的测试框架。因此,有经验的Unix程序员并不仅仅把这些练习程序看作是为库使用者提供便利,也会认为代码应已经过很好的测试。

库分层的一个重要形式是\textbf{插件},即拥有一套已知入口、可在启动以后动态从入口处载入来执行特定任务的库。这种模式必须将调用程序作为文档详备的服务库组织起来,以使得插件可以回调。


\subsection{实例分析:GIMP插件}
GIMP(GNU图像处理程序,GNU Image Manipulation program)是一个由交互方式GUI驱动的图形图像编辑器。但是GIMP被做成了一个图像处理和辅助程序的库,由一个相对较薄的控制层代码调用。驱动码知道GUI,但不直接知道图像格式;反过来,程序库程序知道图像格式和图像操作,但不知道GUI。

这个库层次已经文档化了(而且,实际上已作为“libgimp"发布,供其它程序使用)。这意味着C程序写成的所谓“插件”可以由GIMP动态载入,然后调用该库进行图像处理,实际上掌握了和GUI同一级别的控制权(参见图4.2)。

\begin{fig}{载入插件的GIMP中调用和被调用关系图}
\label{fig:载入插件的GIMP中调用和被调用关系图}
\end{fig}

插件可用来完成多种专用转换,如色图调整(colormap hacking)、模糊和去斑;可用于读写非GIMP自带的文件格式;也可用于扩展功能,如编辑动画和窗口管理器主题:通过在GIMP内核中编写图像调整逻辑脚本,还可实现其他多种图像调整处理的自动化。万维网中有各种GIMP插件的注册中心。

虽然大多数GIMP插件都是小巧简单的C程序,但是也有可能编制一个插件让库API能被脚本语言调用。我们将在第11章分析“多价程序”模式时讨论这种可能性。


\section{Unix和面向对象语言}
1980年代中期起,大多数新的语言设计都已自带了对“面向对象”(OO)编程的支持。回想一下,在面向对象的编程中,作用于具体数据结构的函数和数据一起被封装在可视为单元的一个对象中。相反,非OO语言中的模块使数据和作用于该数据的函数的联系变得相当无规律,而且模块间还经常互相泄漏数据或内部细节。

OO设计理念的价值最初在图形系统、图形用户界面和某些仿真程序中被认可。使大家惊讶并逐渐失望的是,很难发现OO设计在这些领域以外还有多少显著优点。其中原因值得我们去探究一番。

在Unix的模块化传统和围绕OO语言发展起来的使用模式之间,存在着某些紧张对立的关系。Unix程序员一直比其他程序员对OO更持怀疑态度,原因之一就源于\textbf{多样性原则}。OO经常被过分推崇为解决软件复杂性问题的唯一正确办法。但是,还有其它一些原因,这些原因值得我们在第14章讨论具体OO(面向对象)语言之前作为背景问题加以探讨,这也将有助于我们对Unix的一些非OO编程风格特征有更深刻的认识。

前面我们提到,Unix的模块化传统就是薄胶合层原则,也就是说,硬件和程序顶层对象之间的抽象层越少越好。这部分是因为C语言的影响。在C语言中模仿真正的对象很费力。正因为这样,堆砌抽象层是一件非常累人的事。这样,C语言中的对象层次倾向于比较平坦和透明。即使Unix程序员使用其它语言,他们也愿意继续沿用Unix模型教给他们的薄胶合/浅分层风格。

OO语言使抽象变得很容易——也许是太容易了。OO语言鼓励“具有厚重的胶合和复杂层次”的体系。当问题域真的很复杂、确实需要大量抽象时,这可能是好事,但如果编码员到头来用复杂的办法来做简单的事情——仅仅是因为他们能够这样做,结果便适得其反。

所有的OO语言都显示出某种使程序员陷入过度分层陷阱的倾向。对象框架和对象浏览器并不能代替良好的设计和文档,但却常常被混为一谈。过多的层次破坏了透明性:我们很难看清这些层次,无法在头脑中理清代码到底是怎样运行的。简洁、清晰和透明原则统统被破坏了,结果代码中充满了晦涩的bug,始终存在维护问题。

可能正是因为许多编程课程都把厚重的软件分层作为实现表达原则的方法来教授,这种趋势还在恶化。根据这种观点,拥有很多类就等于在数据中嵌入了很多知识。问题在于,胶合层中的“智能数据”却经常不代表任何程序处理的自然实体——仅仅只是胶合物而已。(这种现象的一个确定标志就是抽象子类或混入(mix-in's)类的不断扩散。)

OO抽象的另一个副作用就是程序往往丧失了优化的机会。例如,a+a+a+a可以用a*4来表示,如果a是整数,也可以表示成a<<2。但是如果构建了一个类并重新定义了操作符,就根本没什么东西可表明运算操作的交换律、分配律和结合律。既然不能查看对象内部,就不可能知道两个等价表达式中哪一个更有效。这本身并不是在新项目中避免使用OO技法的正当理由,那样只会导致过早优化。但这却是在把非OO代码转换为类层次之前需要三思而后行的原因。

Unix程序员往往对这些问题有本能的直觉。在Unix下,OO语言没能代替非OO的主力语言,如C、Perl(其实有OO功能,但用得不多)和shell等,这种直觉似乎也是原因之一。跟其它正统领域相比,Unix世界对OO语言的批判更直接了当;Unix程序员知道什么时候不该用OO;就算用OO,他们也尽可能保持对象设计的整洁清晰。正如《网络风格的元素》(The Elements of Networking Style)一书的作者在另一个略有不同的背景下所说的[Padlipshy]:“如果你知道自己在做什么,三层就足够了;但如果你不知道自己在做什么,十七层也没用。”

OO在其取得成功的领域(GUI、仿真和图形)之所以能成功,主要原因之一可能是因为在这些领域里很难弄错类型的本体问题。例如,在GUI和图形系统中,类和可操作的可见对象之间有相当自然的映射关系。如果你发现增加的类和所显示的对象没有明显对应关系,那么很容易就会注意到胶合层太厚了。

Unix风格程序设计所面临的主要挑战就是如何将分离法的优点(将问题从原始的场景中简化、归纳)同代码和设计的薄胶合、浅平透层次结构的优点相结合。

我们将在第14章探讨面向对象的语言时继续讨论并应用以上一些观点。


\section{模块式编码}
模块性体现在良好的代码中,但首先来自良好的设计。在编写代码时,问问自己以下这些问题,可能会有助于提高代码的模块性:
\begin{itemize}
\item 有多少全局变量?全局变量对模块化是毒药,很容易使各模块轻率、混乱地互相泄漏信息\footnote{全局变量同时也意味着代码不能重入;也就是说,同一进程的多个实例可能彼此干涉。}。
\item 单个模块的大小是否在Hatton的“最佳范围”内?如果回答是“不,很多都超过”的话,就可能产生长期的维护问题。知道自己的“最佳范围”是多少吗?知道与你合作的其他程序员的最佳范围是多少吗?如果不知道,最好保守点儿,坚持Hatton最佳范围的下限。
\item 模块内的单个函数是不是太大了?与其说这是一个行数计算问题,还不如说是一个内部复杂性问题。如果不能用一句话来简单描述一个函数与其调用程序之间的约定,这个函数可能太大了\footnote{很多年前,我从Kernighan和Plauger的《编程风格的元素》(The Elements of Programming Style)一书中学到一个非常有用的原则,就是在函数原型之后立即写一行注释。每个函数都这样,决无例外。}。
\begin{quote}[Ken Thompson]
就我个人而言,如果局部变量太多,我倾向于拆分子程序。另一个办法是看代码行是否存在(太多)缩进。我几乎从来不看代码长度。
\end{quote}
\item 代码是不是有内部API——即可作为单元向其他人描述的函数调用集和数据结构集,并且每一个单元都封装了某一层次的函数,不受其它代码的影响?好的API应是意义清楚,不用看具体如何实现就能够理解的。对此有一个经典的测试方法:通过电话向另一个程序员描述。如果说不清楚,API很可能就是太复杂,设计太糟糕了。
\item API的入口点是不是超过七个?有没有哪个类有七个以上的方法?数据结构的成员是不是超过七个?
\item 整个项目中每个模块的入口点数量如何分布\footnote{收集这种信息有一个简便的方法,就是分析etags(1)或ctags(1)等工具程序生成的标记文件。}?是不是不均匀?有很多入口点的模块真的需要这么多入口点吗?模块复杂性往往和入口点数量的平方成正比——这也是简单API优于复杂API的另一个原因。
\end{itemize}

你可能会发现,如果把以上这些问题和第6章关于透明性和可见性问题的清单加以此较,将颇有启发性。



\chapter{文本化:好协议产生好实践}
\begin{quote}[Tom Galloway]
众所周知,人类几千年前就已经发明了算盘之类的计算设备。但很少有人知道,人类第一次使用通用计算机协议是在《旧约》中——当时摩西用控制海中止了埃及人的进程。
\end{quote}

我们将在本章分析Unix传统所教导的两种不同而又紧密相关的设计:设计将应用数据存储在永久存储器中的文件格式,和在协作程序中(可能会通过网络)传递数据和命令的应用协议。

这两种设计的共通之处在于:两者都与内存数据结构的序列化有关。对于计算机程序的内部操作而言,一个复杂数据结构最简便的表达就是所有字段都用机器自带的数据格式(如整数的二进制补码)来表示,而所有指针都是实际的存储地址(相对于名称引用而言)。但是这些表示法并不适合数据的存储和传输;数据结构的存储地址一旦离开存储器就毫无意义,发送未经处理的原生数据格式又会因为不同机器采用不同约定(如高位字节序对低位字节序,或32位对64位)而产生传输数据的互用问题(interoperability)。

为了便于数据的传输和存储,像链表这样的数据结构,其可遍历的准空间部署需要平整化或序列化成字节流表达,以便日后能从这个表达中恢复数据结构。序列化(保存)操作有时也称为\textbf{列集}(marshaling),其反向操作(载入)称为散集(unmarshaling)。这些术语通常使用在面向对象的语言中,如C++,Python或者Java中对对象的操作,但同样可用于其它一些操作,如图形编辑器将图形文件载入内存并在修改后存盘。

C和C++程序员要维护的代码中,进行列集和散集操作的特别代码占了很大比例——即便所选择的序列化表达很简单,如二进制结构的转储(非Unix环境下的一种通用技巧)也是如此。Python和Java等现代语言往往内置了散集和列集函数,可应用于任何对象或代表对象的字节流,大大减少了工作量。

但是由于种种原因,这些天真的方法常常不尽人意,原因既包括我们上面提到的机器间的互用问题,也包括对其它工具不透明这一负面特征。如果应用程序是网络协议,出于经济性考虑,可能会要求内部数据结构(比如携带源地址和目标地址的信息)不是序列化成单个的大型数据包(blob),而是序列化成接收设备可拒绝的一系列尝试性处理事务或信息(这样一来,如果目的地址无效,则较大的整块信息就会被拒收)。

互用性、透明性、可扩展性和存储/事务处理的经济性——这些都是设计文件格式和应用协议时需要考虑的重要方面。互用性和透明性要求我们在此类设计中要重点考虑数据表达的清晰问题,而不是首先考虑实现的方便性和可能达到的最高性能。既然二进制协议很难扩展和干净地抽取子集,可扩展性当然也青睐文本化协议。事务处理的经济性有时则会提出相反的要求——但我们应看到,首先考虑这个标准就是一种过早优化,不这么做往往是明智选择。

最后,我们必须注意数据文件格式与常用于设置Unix程序启动选项的运行控制文件之间的区别。最根本的区别是(偶尔也有例外,如GNU Emacs的配置接口)程序通常不修改自己的运行控制文件——信息流是单向的,从启动时的文件读取流向应用程序的设置。相反,数据文件格式的属性同命名资源联系在一起,应用程序既可能读也可能写。配置文件通常都可以手工编辑,体积很小,而数据文件通常由程序生成,多大都有可能。

历史上,Unix对这两种表达采用过相关但又不同的约定。第10章将论述控制文件的各种约定:本章仅论述数据文件的约定。

\section{文本化的重要性}
管道和套接字既可以传输文本也可以传输二进制数据。但是,我们将在第7章看到的例子却都是文本化的,理由非常充分:那就是我们在第1章引用的Doug Mcllroy的建议。文本流是非常有用的通用格式,因为人无需专门工具就可以很容易地读写和编辑文本流,这些格式是透明的(或可以设计成透明的)。

同时,正是文本流的限制帮助了强化封装:因为文本流不鼓励内容丰富、编码结构密集的复杂表达,也不提倡程序互相干涉内部状态。我们在第7章结尾部分讨论RPC时继续这个话题。

当你很想设计一个复杂的二进制文件格式,或一个复杂的二进制应用协议时,通常,明智的做法是躺下来等待这种感觉过去。如果担心性能问题,就在应用协议之上或之下的某个层面上压缩文本协议流,最终产生的设计会比二进制协议更干净,性能可能也更好(文本压缩起来更好、更快)。

\begin{quote}[Henry Spencer]
Unix历史上有一个二进制格式的反面教材,那就是设备无关的\textit{troff}程序读取设备信息二进制文件的方式,当时可能出于速度方面的考虑。最初的实现以一种不太可移植的方法从文本描述中生成该二进制文件。为了能在新机器上快速移植础\textit{ditroff}而避免重新编写二进制文件的麻烦,我把它剥离了出来,只让\textit{ditroff}读取文本文件。读取文件的代码经过精心编制,速度的损失可以忽略不计。
\end{quote}

设计一个文本协议往往可以为系统的未来省不少力气。一个具体原因就是格式本身不能表示数字域的范围。二进制格式通常指定了给定值的分配位数,要扩展位数非常困难。例如,IPv4的地址是32位的,要将地址位数扩展到128位(如IPv6)就需要进行大修补\footnote{传说一些早期的飞机订票系统对乘客人数只分配一个字节。随着第一架载客超过255名的波音747投入使用,可以想象这些系统碰到了多少麻烦。}。相反,如果在文本格式中需要更大的值,直接写就是了。也许某个特定程序不能接受那个范围内的值,但是跟修改存储在格式中的所有数据相比,修改这个程序通常要容易得多。

使用二进制协议的唯一正当理由是:如果要处理大批量的数据集,因而确实关注能否在介质上获得最大位密度,或是非常关心将数据转化为芯片核心结构所必须的时间或指令开销。大图像和多媒体数据的格式有时可以算是前者的例子,对延时有严格要求的网络协议有时则可以算是后者的例子。

\begin{quote}[Jim Gettys]
SMTP或类HTTP的文本协议存在的问题则相反。这些协议往往占据昂贵的带宽资源,解析速度很慢。最小的X请求是4个字节:最小的HTTP请求大约是100个字节。X请求,包括已摊销的传输成本,100条指令的数量级就可执行了:一度,一位Apache[Web服务器]开发者自豪地声称他们已经精简到了7000条指令。对图形而言,输出时带宽就是一切;硬件设计已使得图形卡母线成为如今限制小操作的唯一瓶颈,因此,如果协议不想成为更糟糕的瓶颈,最好设计得非常紧凑。这是极端情况。
\end{quote}

这些问题在X以及其它极端情况下也存在——比如,为支持特大图形而设计图形文件格式,但却往往只是另一种过早优化热。文本格式的位密度未必一定比二进制格式低多少;毕竟,它们还是用了八位字节中的七位。一旦你需要生成测试加载、或需要检查程序生成的格式实例并想弄明白个中究竟,本来无需解析文本所带来的收益往往马上就全部损失殆尽。

另外,设计紧凑二进制格式的思路往往不能够兼顾干净扩展的要求。X的设计者就有这方面的教训:
\begin{quote}[Jim Gettys]
从目前的X框架来看,我们确实没有设计出足够好的结构,使得对协议微小的扩展仍会对它造成影响;当然,有时我们可以做到这一点,但如果有一个更好的框架会更好。
\end{quote}


当认为找到一种极端情况,有足够理由使用二进制文件格式或协议时,需仔细考虑扩展性,并在设计中为以后发展留出余地。


\subsection{实例分析:Unix口令文件格式}
在许多操作系统中,验证用户登录并开始用户会话所必需的用户数据都是不透明的二进制数据库。相反,在Unix中,这种数据是文本文件,采用一行一条、字段用冒号分隔的记录格式。

例5.1是几行随机选择的文件行:
\begin{Verbatim}
games:*:12:100:games:/usr/games:
gopher:*:13:30:gopher:/usr/lib/gopher-data:
ftp:*:14:50:FTP User:/home/ftp:
esr:0SmFuPnH5JlNs:23:23:Eric S. Raymond:/home/esr:
nobody:*:99:99:Nobody:/:
\end{Verbatim}
即使不知道字段的语义,我们也能发现这些数据在二进制格式中很难压缩得更紧。要达到冒号分隔符的功能至少需要相同的空间(通常是字节数或NUL)。每个用户的记录要么需要终止符(不太可能比一个新行符更短),要么很浪费地补齐到定长。

如果知道数据的实际语义,则通道二进制编码节省空间的实际可能性几乎不存在。数字形式的用户ID(第三)和组ID(第四)字段都是整数,这样,大多数机器上的一个二进制表达至少需要4个字节,比文本格式表达999以下数字所需要的长度更长。不过,让我们暂且忽略这些,假设是最佳情形,即数字域在0到255范围内。

我们可以压缩数字域(第三字段和第四字段),把这些数字域的位长缩小到用单字节表示,口令字符串(第二字段)采用8位编码。在本例中,这样可以节省8\%{}左右的空间。

这8\%{}的假定低效率却带给我们很多好处:可以避免武断地限制数字域范围,可以使我们能够使用自己选择的任何老式文本编辑器修改口令文件,而无需编制专用工具来编辑二进制文件(虽然在口令文件这个例子本身,我们需要对并发编辑特别小心),而且还让我们能够用grep(1)这样的文本流工具对用户帐号信息进行特别的搜索、过滤和报告。

我们的确要十分小心,不要在任何文本字段中嵌入冒号。良好的做法应该是这样:告诉文件先用换码符嵌入冒号再写代码,然后告诉文件读取代码对其解释。对此,Unix传统偏爱使用反斜杠。

通过字段位置而不是明确的标记来传达结构信息使得这种格式的读写都很轻松,但是有些死板。如果一个键所关联的属性集要发生改动,那么以下描述的标记格式可能是更好的选择。

既然一般情况下很少读取\footnote{口令文件通常在每个用户对话的登录阶段才读一次,之后ls(1)之类的实用程序为了将数字用户ID和组ID映射成名称才偶尔读取这些口令文件。},也不经常修改,所以经济性不是口令文件一开始就要考虑的主要因素。既然文件中的不同数据(特别是用户ID和组ID)不会从原始机器上搬移出去,互用性也不是问题。因此很明显,对口令文件而言,遵循透明性原则才是正确的选择。


\subsection{实例分析:.newsrc格式}
Usenet新闻是一个全球性分布式公告牌系统(BBS),比今天的P2P网络要早20年。它使用的信息格式与RFC 822电子邮件信息格式非常相似,只不过不是直接发送给个人接受者,而是发给主题组。所有在入网站点上张贴的文章先转发到登记为友邻的站点上,最终发送到整个网内的所有站点。

几乎所有的Usenet读者都理解.newsrc文件,该文件记录了使用者已经阅读过哪些信息。尽管该文件名字很像一个运行控制文件,但不仅启动时要读取该文件,而且通常在新闻阅读器运行结束时还要更新该文件。自1980年左右出现第一个新闻阅读器以后,.newsrc格式就固定了下来。例5.2是.newsrc文件中具有代表性的一段代码。
\begin{Verbatim}[label=.newsrc实例]
rec.arts.sf.misc! 1-14774,14786,14789
rec.arts.sf.reviews! 1-2534
rec.arts.sf.written: 1-876513
news.answers! 1-199359,213516,215735
news.announce.newusers! 1-4399
news.newusers.questions! 1-645661
news.groups.questions! 1-32676
news.software.readers! 1-95504,137265,137274,140059,140091,140117
alt.test! 1-1441498
\end{Verbatim}

每行都为以第一个字段为名的新闻组设置属性。新闻组名称之后紧跟一个字符,表明文件对应的用户目前是否订阅了该组;冒号表示订阅,惊叹号表示没有订阅。其余部分是一系列逗号分隔的文章编号或文章编号范围,表明用户已经阅读过哪些文章。

非Unix程序员也许会自然而然地去试图设计一个快速二进制格式,其中每个新闻组的状态采用固定的长二进制记录或由一系列内含长度字段的自描述二进制信息包来表示。这种二进制表示的要点在于:在成对字长字段内用二进制数据来表示范围,目的是避免启动时解析所有范围表达式的开销。

这种布局也许读写都能比文本格式快,但是会产生其它问题。固定记录长度这种简单的实现会造成人为限制新闻组名称的长度,  (更严重的是)限制已读文章数量范围的最大值。用一种更复杂的二进制包格式可避免这种长度限制,但用户无法自行查看或编辑——而当需要重新设置某个新闻组中的某个已读状态字段时,这种能力非常有用。而且,这种格式还不一定能移植到其它类型的机器上。

最初的新闻阅读器设计者舍经济性而取透明性和可操作性。当然,反过来从经济性角度来考虑,这种取舍也不无道理。由于.newsrc文件有可能变得很大,某些新型阅读器(GNOME的Pan阅读器)就使用对速度优化的专用格式来避免启动等待。但对其他实现者而言,文本表达在1980年看起来就是很好的折衷方案,而且随着机器速度的提高和存储器价格的下降,这种选择现在愈发显得明智。


\subsection{实例分析:PNG图形文件格式}
PNG(可移植网络图形)是位图图形的一种文件格式。PNG更像GIF,而不像JPEG,其不同之处在于采用了无损压缩,并为艺术线条(line art)和图标而不是照片图像的应用程序进行了优化。高质量的文档和开源参考库可从其站点\href{http://www.libpng.org/pub/png}{http://www.libpng.org/pub/png}上获得。

PNG格式是二进制文件格式中一个经过周密设计的优秀例子。既然图形文件包含了大量的数据,如果像素数据用文本格式来存储的话,尺寸和网络下载时间都会显著提高,因此二进制格式非常合适。传输经济性是要考虑的主要问题,透明性则牺牲了\footnote{别搞错,PNG格式支持另一种透明性——透明像素。}。但是,设计者对互用性非常谨慎:PNG格式指定了字节顺序、整数的字长、优先顺序,和(但缺少)字段间的填充。

PNG文件由一系列字节块(chunk)构成,每个都是自描述格式,以块类型名和块长度开头。由于这种组织形式,PNG不需要版本号。随时都可以增加新的块类型:块类型名称中的第一个字母告知使用PNG的软件当前块是否可被安全忽略。

PNG文件头同样值得研究。它设计得非常聪明,能使各种常见的文件损坏情况(如7位传输连接,或CR字符和LF字符的损坏)很容易被发现。

PNG标准精确全面,编写得非常好,可以作为如何撰写文件格式标准的范例来使用。


\section{数据文件元格式}
数据文件元格式是一套句法和词法约定,这套约定或者已经正式标准化,或者己经通过实践得到了充分的确定,已有标准服务库来处理列集和散集操作。

Unix已经形成或采纳了适合多种应用程序的不同元格式。尽可能使用这些元格式(而不是标新立异自己的格式)是个好习惯。第一个好处是使用服务库可以避免编写大量的用户解析代码和生成代码。但最重要的好处还是开发者甚至很多用户都能立即认出这些格式,有亲切感,这就减少了学习新程序的磨合成本(friction cost)。

在以下讨论中,当我们说到“传统Unix工具”时,我们指grep(1)、sed(1)、awk(1)和cut(1)这些文本搜索和变换工具的组合。对以上工具所提倡的面向行格式的解析,Perl和其他脚本语言通常自带支持功能。

以下就是一些可以作为典范使用的标准格式。



\subsection{DSV风格}
DSV代表“Delimiter-Separated Values(\textbf{分隔符分隔值})”。我们在文本元格式引用的第一个例子/etc/passwd文件就是一个使用冒号作为值分隔符的DSV格式。在Unix中,对字段值可能包含空格的DSV格式,冒号是默认的分隔符。

/etc/passwd格式(每个记录一行,字段用冒号分隔)是Unix中非常传统的格式,经常用于处理表列数据。其它经典的例子包括描述安全组的/etc/group文件和在操作系统不同运行级别中控制Unix服务程序启动和关闭的/etc/inittab文件。

这种风格的数据文件一般应通过反斜杠(\textbackslash )转义符支持在数据域中包含冒号。更为普遍的是,读取这种文件的代码可通过忽略反斜线转义的换行符支持连续记录,并且允许通过C风格的反斜杠转义符嵌入非打印字符数据。

当数据为列表、名称(在首字段)为关键字、而且记录通常很短(少于80个字符)时,这种格式最适用。这种格式和传统的Unix工具配合得很好。

有时候也可以看到冒号以外的字段分隔符,如管道字符“|”,甚至用ASCII NUL。Unix的旧学派做法偏爱TAB,这可在cut(1)和paste(1)的默认设置中反映出来。但随着格式设计者逐渐意识到,TAB和SPACE在视觉上无法区别而引起了很多令人恼火的小麻烦,这种做法也逐渐改变了。

这种格式之于Unix,就像CSV(逗号分隔值)格式之于Unix世界外的Microsoft Windows和其它操作系统。Unix中很少用到以逗号分隔字段、双引号用来转义逗号、没有连续行的CSV格式。

事实上,Microsoft版CSV是一个如何设计文本文件格式的典型反面例子。问题首先出现在字段正好含有分隔字符(在这种情况下是逗号)的情况中。Unix的方法是简单的用反斜杠转义分隔符,用双反斜杠表示反斜杠字面值。在解析文件时,这种设计只要检查一种特殊情况(转义符),发现转义符时只要一个操作(解析跟在转义符后的字符)。后者不仅方便了分隔符的处理,而且还能自由处理转义符和新行符。CSV则相反,如果字段值中存在分隔符,就将整个字段值包括在双引号内。如果字段值包含双引号,整个字段值也得包括在双引号内,字段中的单个双引号需要重复两遍才能表明自己并不结束整个字段。

到处使用特殊情况所带来的不良结果是双重的。首先,分析程序的复杂度(以及bug的易发性)提高了。其次,由子格式规定既复杂又不明确,不同实现对边缘情况的处理也不同。有时,通过在一行的最后一个字段前使用双引号来支持连续行——但只有部分产品这么做!在微软自己的应用软件,有时甚至是同一个应用程序的不同版本之间(Excel就是最明显的例子),CSV文件都存在不兼容的情况。


\subsection{RFC 822格式}
RFC 822格式源自互联网电子邮件信息采用的文本格式;(在被RFC 2822取代前)RFC 822一直是描述这种格式的主要互联网RFC。MIME(多用途网际邮件扩充协议,Multipurpose Internet Mail Extension)提供了在RFC 822格式信息中嵌入类型化二进制数据的方法(在网上搜索以上这些名称的任何一个都可以找到相关标准)。

在这种元格式中,记录属性每行存放一个,以类似邮件头字段名的标记命名,用冒号后接空白作为结束。字段名不得包含空格;通常用横线代替空格。该行的其余部分都是属性值,除了结尾的空格和换行。以tab(制表符)或whitespace(空格符)开始的物理行被解释为当前逻辑行的延续。空行可能被解释为记录的结束,也可能表明接下来是非结构化的文本。

在Unix中,对那些带属性的或任何与电子邮件类似的倍息,这都是传统而且首选的文本元格式。一般来说,这种格式非常适合具有不同字段集合而字段中数据层次又扁平(没有递归或树形结构)的记录。

Usenet使用的就是这种格式,万维网使用的HTTP1.1(以及后续版本)也使用这种格式。这种格式非常便于人工编辑。在属性搜索上,传统的Unix搜索工具仍能使用,只不过寻找记录边界要比“每行一个记录”的格式要多费些周折。

RFC 822格式的一个弱点是,当一个文件中有不止一个RFC 822信息或记录时,记录边界可能不太明显一一可怜的死脑筋的计算机如何知道一条信息的非结构化正文结束,而下一个记录头开始的地方在哪里昵?历史上已经存在过好几个不同的分隔邮箱中信息的约定。每条信息的第一行以字符串“From"和发送者信息开头的这种最古老、受到最广泛支持的方法并不适合其它类型的记录;这种方法也要求转义信息文本行以“From”开头(通常用“>”)——这种做法经常引起混淆。

有些邮件系统使用那些不太可能出现在信息中的控制符作为分隔行,如连续使用几个ASCII 01 (control-A)字符。MIME标准通过在邮件头中包含一个确定的信息长度避开了这个问题,但这是一个不太稳妥的解决方案,一旦对信息进行了手工编辑,这种解决方案很容易出问题。更好的解决方案参见本章下面描述的record-jar风格。

看看自己的电子邮箱就可以找到RFC 822格式的例子。


\subsection{Cookie-Jar格式}
Cookie-jar格式是fortune(1)程序为随机引用数据库而使用的一种格式。这种格式很适用记录只是一堆非结构化文本的情况。这种格式简单使用跟随\%{}\%{}的新行符(或者有时只有一个\%{})作为记录分隔符。例5.3是来自电子邮件签名引用文件的部分例行。


\begin{Verbatim}[label=例5.3 fortune文件实例]
"Among the many misdeeds of British rule in India, history will look
upon the Act depriving a whole nation of arms as the blackest."
        -- Mohandas Gandhi, "An Autobiography", pg 446
%
The people of the various provinces are strictly forbidden to have 
in their possession any swords, short swords, bows, spears, firearms,
or other types of arms. The possession of unnecessary implements 
makes difficult the collection of taxes and dues and tends to foment 
uprisings.
        -- Toyotomi Hideyoshi, dictator of Japan, August 1588
%
"One of the ordinary modes, by which tyrants accomplish their 
purposes without resistance, is, by disarming the people, and making 
it an offense to keep arms."
        -- Supreme Court Justice Joseph Story, 1840
\end{Verbatim}

寻找记录分隔符时接受\%{}后的空格是个好做法,有助于解决人为编辑的错误。更好的做法就是使用\%{}\%{},并忽略从\%{}\%{}到行结束处的所有文本。
\begin{quote}[Ken Arnold]
cookie-jar分隔符最初是“\%{}\%{}\textbackslash n”。我当时希望找个能比“\%{}”更显眼的东西。事实上,任何“\%{}\%{}”之后的内容都作注释处理(至少我是这样写的)。
\end{quote}

简单的cookie-jar格式适用于词以上结构没有自然顺序,而且结构不易区别的文本段,或适用于搜索关键字而不是文本上下文的文本段。


\subsection{Record-Jar格式}
cookie-jar记录分隔符和RFC 822记录元格式结合得非常好,产生一种我们称之为“record-jar”的格式。如果文本格式要支持显式字段数目可变的多重记录,众望所归的方法就是采用例5.4类似的格式。

\begin{Verbatim}[label=例5.4以record-jar格式表达的三颗行星基本数据]
Planet: Mercury
Orbital-Radius: 57,910,000 km
Diameter: 4,880 km
Mass: 3.30e23 kg
%%
Planet: Venus
Orbital-Radius: 108,200,000 km
Diameter: 12,103.6 km
Mass: 4.869e24 kg
%%
Planet: Earth
Orbital-Radius: 149,600,000 km
Diameter: 12,756.3 km
Mass: 5.972e24 kg
Moons: Luna
\end{Verbatim}

当然,记录分隔符也可以是一个空行,但是由“\%{}\%\textbackslash n”构成的一行更为清晰,也不大可能在编辑时无意产生(两个可打印字符比一个好,因为这样不可能由单个笔误引起)。在类似这样的格式中,直接忽略空行是个不错的办法。

如果记录中含有部分非结构化文本,record-jar格式就非常接近邮箱格式。在这种情况下,关键有一个定义良好的转义分隔符的方法,这样分隔符才能出现在文本中;否则,读取记录的程序总有一天会卡在形式不良文本部分上。本书指出了和字节填充(本章后面部分将有描述)类似的一些技巧。

Record-jar格式适合于那些类似DSV文件、但又有可变字段数目而且可能伴随无结构文本的字段属性关系集合。
 
\subsection{ XML}
XML语法类似于HTML,非常简单——尖括号括起的(<>)标签和“\&{}”记号引导字面值序列。它几乎和纯文本标记一样简单,但又能表达递归嵌套的数据结构。XML只是一种低级的语法,需要文档类型定义(例如XHTML)和相关的应用逻辑赋予其语义。

XML非常适合复杂的数据格式(旧学派Unix传统会为此使用类似RFC 822的节格式),尽管对简单的数据来说未免有些大材小用。它尤其适合那些RFC 822元格式不太好处理、有复杂递归或嵌套数据结构的格式。对这种格式的详细介绍,可参见《XML in a Nutshell》一书[Harold-Means].
\begin{quote}[Keith Packard]
设计文本格式最难处理的问题是引句(quoting)、空格符和其它低级语法细节。用户文件格式常常因为语法结构上的轻微错误而不能跟类似格式匹配。使用XML之类的标准格式,可以由标准程序库来校验并解析,解决了这些问题中的绝大部分。
\end{quote}

例5.5是一个基于XML的配置文件简单实例。它是Linux下随开源KDE office suite(套装办公软件)发布的\textit{kdeprint}工具的一部分,描述了从图像转换到PostScript的过滤操作选项,以及如何将它们映射成过滤器命令行参数。另一个有益示例请参见第8章对\textit{Glade}的讨论。

XML的一个优势在于经常无需知道数据语义,仅通过语法检查就能发现形式不良、损坏或错误生成的数据。

XML最严重的问题是无法很好和传统的Unix工具协作。读取XML格式的软件需要XML解析器,这就意味着需要庞大复杂的程序。同样,XML本身也相当庞大,要在所有的标记中找到数据很困难。

XML占据明显优势的应用领域是文档文件的标记格式(我们将在第18章予以更详细的讨论)。在大块纯文本中,此类文件的标记往往比较稀疏,因此Unix传统工具仍能出色完成简单的文本搜索和转换。

\begin{Verbatim}[label=例5.5 XML实例]
<?xml version="1.0"?>
<kprintfilter name="imagetops">
    <filtercommand 
           data="imagetops %filterargs %filterinput %filteroutput" />
    <filterargs>
        <filterarg name="center" 
                   description="Image centering" 
                   format="-nocenter" type="bool" default="true">
            <value name="true" description="Yes" />
            <value name="false" description="No" />
        </filterarg>
        <filterarg name="turn" 
                   description="Image rotation" 
                   format="-%value" type="list" default="auto">
            <value name="auto" description="Automatic" />
            <value name="noturn" description="None" />
            <value name="turn" description="90 deg" />
        </filterarg>
        <filterarg name="scale" 
                   description="Image scale" 
                   format="-scale %value" 
                   type="float" 
                        min="0.0" max="1.0" default="1.000" />
        <filterarg name="dpi" 
                   description="Image resolution" 
                   format="-dpi %value" 
                   type="int" min="72" max="1200" default="300" />
    </filterargs>
    <filterinput>
        <filterarg name="file" format="%in" />
        <filterarg name="pipe" format="" />
    </filterinput>
    <filteroutput>
        <filterarg name="file" format="> %out" />
        <filterarg name="pipe" format="" />
    </filteroutput>
</kprintfilter>
\end{Verbatim}

这些领域间有一个很有趣的沟通桥梁,就是PYX格式——面向行的XML转换,可由传统的Unix面向行文本工具进行修改,然后再无损转换成XML。在网上搜索“Pyxie”即可找到相关资源。xmltk工具包则采取相反办法,提供类似\textit{grep}(1)和\textit{sort}(1)的面向流工具来过滤XML文档;在网上搜索“xmltk”即可找到。

选择XML可以简化问题,也可能使问题复杂化。对它的大肆吹捧很多,但不要不加批判地采用或拒绝,否则就会成为时尚的牺牲品。请谨慎选择,牢记KISS原则。


\subsection{Windows INI格式}
许多微软的Windows程序都使用类似例5.6的文本数据格式。这个例子将名为account、directory、numeric\_{}id和developer的可选资源和名为python、sng、fetchmail和py-howto的项目关联在一起。如果某个指定的输入项没有提供值,则DEFAULT输入项将提供相应值。


\begin{Verbatim}[label=例5.6 .INI文件实例]
[DEFAULT]
account = esr

[python]
directory = /home/esr/cvs/python/
developer = 1

[sng]
directory = /home/esr/WWW/sng/
numeric_id = 1012
developer = 1

[fetchmail]
numeric_id = 18364

[py-howto]
account = eric
directory = /home/esr/cvs/py-howto/
developer = 1
\end{Verbatim}

这种风格的数据文件格式并不是Unix自带的,但在Windows影响下,一些Linux程序(特别是Samba,一种在Linux系统上获取Windows文件共享的工具套件)也开始支持这种格式。这种格式可读性好,设计得不错,但和XML一样,不能与\textit{grep}(1)或常规Unix脚本工具很好地配合使用。

如果数据围绕指定的记录或部分能够自然分成“名称-属性对”两层组织结构,.INI格式非常适用。但这种格式并不适合数据存在完全递归树形结构的情况(XML更适合)。 而对于简单的名称-值关系列表,这种格式又是大材小用(这时应使用DSV格式)。

\subsection{Unix文本文件格式的约定}
Unix关于文本数据格式应该怎样的传统由来已久。这些约定大多来源于我们刚刚讨论过的Unix标准元格式中的一个或多个格式。若非确有特殊原因,最好还是遵循这些约定。

我们将在第10章讨论程序运行控制文件使用的一套不同约定。但大家应该注意,以下一些原则(尤其在词法级别上,字符如何组合成标记的规则)也同样适用这套约定。
\begin{itemize}
\item \textbf{如果可能,以新行符结束的每一行只存一个记录}。这样用文本流工具提取记录就非常容易。为了和其它操作系统交换数据,最好让文件格式的解析器不受行结束符是LF还是CR-LF的影响。在这种格式中,习惯上忽略结尾的空白,以防范常见的编辑错误。
\item  \textbf{如果可能,每行不超过80个字符}。这样使格式可以在普通尺寸的终端视窗上浏览。如果很多记录一定要超过80个字符,考虑使用分节格式(stanza format)(见下文)。
\item \textbf{使用“\#{}”引入注释}。能在数据文件中嵌入注解和说明会非常好。最好是把它们作为文件结构的一部分,便可被知道这种格式的工具保存下来。对于解析时不保存的说明,惯例上采用“\#{}”作为起始字符。
\item \textbf{支持反斜杠约定}。支持嵌入不可打印控制字符的最自然方法,就是解析C语言风格的反斜杠转义——\textbackslash{}n表示新行,\textbackslash{}r表示回车,\textbackslash{}t表示制表符,\textbackslash{}b表示退格,\textbackslash{}f表示走纸,\textbackslash{}e表示ASCII escape (27),\textbackslash{}nnn或\textbackslash{}onnn或\textbackslash{}0nnn表示八进制值为nnn的字符,\textbackslash{}xnn表示十六进制值为nn的字符,\textbackslash{}dnnn表示十进制值为nnn的字符,\textbackslash{}\textbackslash{}表示实际意义上的反斜杠。还有一个较新但也应当遵守的约定是使用\textbackslash{}unnn表示十六进制的Unicode字面值。
\item \textbf{在每行一条记录的格式中,使用冒号或任何连续的空白作为字段分隔符}。冒号约定似乎起源于Unix的口令文件。如果某个字段必须包含分隔符,使用反斜杠前缀进行转义。
\item \textbf{不要过分区别tab和whitespace}。否则,当用户编辑器的tab设置不同时,会产生很多令人头痛的麻烦。这条原则是治愈头痛的良方。况且,一般来说,眼睛很难区别tab和whitespace。仅使用tab作为分隔符尤其容易产生问题;相反,允许使用连续的tab和空格作为分隔符却非常有效。
\item \textbf{优先选用十六进制而不是八进制}。和三位的八进制数字相比,两位或四位的十六进制数字更容易直观地与字节以及今天的32位和64位字对应起来;而且,效率也或多或少高一点。强调该准则是因为\textit{od}(1)等一些较老的Unix工具违反了这条准则,这是较老的PDP小型机的机器语言指令字段大小所产生的历史遗留问题。
\item \textbf{对于复杂的记录,使用“节(stanza)”格式:一个记录若有多行,就使用\%{}\%{}\textbackslash{}n或\%{}\textbackslash{}n作为记录分隔符}。在人们肉眼检查文件时,这种分隔符是非常有用而且直观的边界标志。
\item \textbf{在节格式中,要么每行一个记录字段,要么让记录格式和RFC 822电子邮件头类似,用冒号终止的字段名关键字作为引导字段}。当字段经常空缺或者超过80个字符,或者当记录很稀疏时(如经常有空字段),适用第一二种方案。
\item \textbf{在节格式中,支持连续行}。解释文件时,或者抛弃空格符之后的反斜杠,或者将空格符之后的新行符解释为单个空格;这样,一个很长的逻辑行就能够折叠成多个很短(容易编辑!)的物理行。在这些格式中,习惯上忽略结尾的空格,可防范常见的编辑错误。
\item \textbf{要么包含一个版本号,要么将格式设计成相互独立的自描述字节块}。哪怕只存在一丁点格式发生改变或扩展的可能性,也要包含一个版本号,这样代码才能够有条件地在所有版本上正确运行。换句话说,将格式设计成自描述字节块,无需立即破坏旧代码就可以增加新的块类型。
\item \textbf{注意浮点数取整问题}。由于所用转换库质量的不同,浮点数从二进制格式转换成文本格式再转换回二进制格式时可能会有精度损失。如果列集/散集的结构中包含浮点数,应该从两个方向都测试一下转换。如果看上去任何一个方向的转换都可能存在取整误差,做好将浮点字段作为未处理的二进制格式或字符串编码形式转储的准备。如果在C语言或调用了C printf/scanf的语言中编程,C99的\%{}a指示符可以解决这个问题。
\item \textbf{不要仅对文件的一部分进行压缩或二进制编码}。见下文……
\end{itemize}


\subsection{文件压缩的利弊}
许多现代的Unix项目,如OpenOffice.org和AbiWord,现在都使用\textit{zip}(1)或\textit{gzip}(1)压缩的XML作为数据文件格式。压缩的XML综合了空间经济性和文本格式的一些优势——尤其是避免了二进制格式常常必须要为那些特定情况下(如特殊选项或超大范围)可能用不到的信息分配空间的问题。但目前对此还存在争论,也正是这种争论引发了本章讨论的一些主要折衷方案。

一方面,实验表明,经过压缩的XML文件通常比Microsoft Word自带的二进制文件格式明显要小,虽然大家可能认为二进制文件占用的空间更小。原因同Unix“就做好一件事”的基本哲学原理有关。创建一个简单的工具来做好压缩,要比仅对文件某些部分进行特别压缩更有效,原因在于,压缩工具可以扫描所有数据,然后找到信息中的所有重复部分进行压缩。

同时,将表现形式的设计和具体压缩的方法分离,将来就可能只要对实际文件解析做最少量修改便可以使用不同的压缩方法——或许根本就不需要任何修改。
  
从另一方面来看,压缩确实在某种程度上损害了透明性。尽管人可以根据上下文推测解开压缩文件是否会看到有用的东西,但是直到2003年中期,file(1)之类的工具仍然还无法看穿这个“包装层”。

可能有人会提倡那些没有这么结构化的压缩格式——比方说,不要\textit{zip}(1)提供的内部结构和自识别头部块,直接用\textit{gzip}(1)压缩XML数据。尽管使用类似\textit{zip}(1)的格式能解决识别问题,但对于用比较简单脚本语言编写的程序来说,解码这些文件会相当棘手。

这些解决方案中的任何一种(纯文本,纯二进制或压缩文本)都可能是最佳方案,具体取决于对存储经济性、可显性或让浏览工具编写起来尽可能简单等问题的权衡考虑。上述讨论并非要鼓吹哪一种方法比其它方法更好,而是就如何考虑清楚各种选择方案和设计出折中方案提出一些建议。

人们一直说,真正的Unix式解决方案也许是用\textit{file}(1)透过压缩看文件前缀——如果不行,就围绕\textit{file}(1)写一个shell脚本,对压缩内容执行gunzip(1)再看。


\section{应用协议设计}
我们将在第7章讨论“将复杂应用程序划分成几个协作进程、通过应用程序专用命令集或协议通信”的优点。所有将数据文件格式设计成文本格式的好理由同样适用于应用程序专用协议的设计。

如果应用协议是文本式的,而且仅凭肉眼就能很容易地分析,那么很多好事情就更容易实现了。事务转存更容易解释。测试负载也更容易编写。

服务器进程通常由\textit{inetd}(8)之类的统一控制程序( harness programs )调用,其方式是服务器程序从标准输入中接收命令,然后将响应发送到标准输出。我们将会在第11章更详细地描述这种“CLI服务器”模式。

CLI服务器的命令集是为达到简洁性而设计的,这种服务器程序有一个可贵的特性,就是测试人员能够直接向服务器进程键入命令来探知软件的工作情况。

另一个需要牢记在心的问题是端对端( end-to-end )设计守则。每一个协议设计者都应该读一读经典的《系统设计中的端对端论》( End-to-End Arguments in System Design )[Saltzer]。人们经常会严肃地对究竟协议栈哪一层应该处理安全和认证之类的功能提出问题。这篇文章为如何考虑这个问题提供了一些很好的概念性手段。除此以外,还有第三个问题,就是为获得良好的性能而设计应用协议。我们将在第12章更详细讨论这个问题。

1980年以前,互联网应用协议的设计传统一直独立于Unix发展\footnote{互联网协议的行结束符通常是CR-LF,而不是Unix的单LF,这就是这段前Unix历史的遗留痕迹。}。但自那以后,这些传统已经完全融入了Unix实践。

下面我们将研究三个使用最广泛,也是被广大互联网hacker看作典范的应用协议实例:SMTP、POP3和IMAP,来说明互联网的风格。这三个协议分别致力于邮件传输(和万维网一起构成网络最重要的两个应用)的不同方面,而所涉及的问题(传输消息、设置远程状态、报告错误状态)对非电子邮件应用协议也颇具普遍意义,并且通常也可采用类似的技巧来处理。



\subsection{实例分析:SMTP,一个简单的套接字协议}
例5.7是RFC 2821描述的SMTP协议(简单邮件传送协议)中的一个处理实例。在这个例子中,C: 行由发送邮件的邮件传输代理(MTA)发送,S: 行由接收邮件的MTA返回。用\textit{斜体字}强调的是注释,并非事务处理的实际部分。

\begin{Verbatim}[label=例5.7  SMTP会话实例, commandchars=\\\{\}]
C: <client connects to service port 25>
C: HELO snark.thyrsus.com               \textit{sending host identifies self}
S: 250 OK Hello snark, glad to meet you  \textit{receiver acknowledges}
C: MAIL FROM: <esr@thyrsus.com>         \textit{identify sending user}
S: 250 <esr@thyrsus.com>... Sender ok   \textit{receiver acknowledges}
C: RCPT TO: cor@cpmy.com                \textit{identify target user}
S: 250 root... Recipient ok             \textit{receiver acknowledges}
C: DATA
S: 354 Enter mail, end with "." on a line by itself
C: Scratch called.  He wants to share
C: a room with us at Balticon.
C: .                                   \textit{ end of multiline send}
S: 250 WAA01865 Message accepted for delivery
C: QUIT                                \textit{ sender signs off}
S: 221 cpmy.com closing connection      \textit{receiver disconnects}
C: <client hangs up>
\end{Verbatim}

邮件就是这样在互联网机器上传输的。注意以下特征:请求的命令参数格式,应答由状态码和紧接其后的指示信息构成,以及“DATA”命令的有效数据部分以一个只有单个“.”的行结束。

SMTP是互联网上仍在使用的最古老的两三个应用协议之一。这个协议简单有效经受住了时间的考验。我们在这里重点指出的几个特征,也经常在互联网其它协议中出现。如果说设计良好的互联网应用协议有个原型的话,那么这个原型一定是SMTP。

\subsection{实例分析:POP3,邮局协议}
另一个经典的互联网协议是POP3,即邮局协议(Post Office Protocol)。这个协议也用于邮件传输,但是SMTP是邮件发送者启动事务处理的“推(push)”协议,而POP3是邮件接收者启动事务处理的“拉(pull)”协议。不连续访问互联网的用户(如拨号连接)可以让他们的邮件存在一个邮箱机器上,然后使用POP3连接将邮件通过网线接收到自己的电脑上。

例5.8是POP3会话的一个例子。其中,C: 行由客户端发送,S: 行由邮件服务器端发送。可以看到它跟SMTP有很多相似之处。这个协议也是文本协议,也是面向行的,发送的有效消息部分由单点行加上行终止符结束,甚至使用同一个退出命令——QUIT。如同SMTP,每次客户端操作都经过回复行确认,回复行以状态码开头,其中包括了可供人眼识别的提示信息。


\begin{Verbatim}[label=例5.8 POP3会话实例]
C: <client connects to service port 110> 
S: +OK POP3 server ready <1896.6971@mailgate.dobbs.org>
C: USER bob
S: +OK bob
C: PASS redqueen
S: +OK bob's maildrop has 2 messages (320 octets)
C: STAT
S: +OK 2 320
C: LIST
S: +OK 2 messages (320 octets)
S: 1 120
S: 2 200
S: .
C: RETR 1
S: +OK 120 octets
S: <the POP3 server sends the text of message 1>
S: .
C: DELE 1
S: +OK message 1 deleted
C: RETR 2
S: +OK 200 octets
S: <the POP3 server sends the text of message 2>
S: .
C: DELE 2
S: +OK message 2 deleted
C: QUIT
S: +OK dewey POP3 server signing off (maildrop empty)
C: <client hangs up>
\end{Verbatim}

与SMTP有一些不同之处,最明显的区别是POP3使用状态标记,而不是像SMTP那样使用3位数字的状态码。当然,请求的语义也不同。但是两者的族谱相似性(本章稍后讨论通用互联网元协议时会对此详细说明)很明显。


\subsection{实例分析:IMAP,互联网消息访问协议}
为了完整展示互联网应用协议的三剑客,我们最后再看看IMAP——另一个设计风格略有不同的邮局协议。请看例5.9:跟前面一样,C: 行由客户端发送,S: 行由邮件服务器发送。用\textit{斜体字}强调的是注释,并非事务处理的实际部分。

\begin{Verbatim}[label=例5.9 IMAP会话实例]
C: <client connects to service port 143>
S: * OK example.com IMAP4rev1 v12.264 server ready
C: A0001 USER "frobozz" "xyzzy"
S: * OK User frobozz authenticated
C: A0002 SELECT INBOX
S: * 1 EXISTS
S: * 1 RECENT
S: * FLAGS (\Answered \Flagged \Deleted \Draft \Seen)
S: * OK [UNSEEN 1] first unseen message in /var/spool/mail/esr
S: A0002 OK [READ-WRITE] SELECT completed
C: A0003 FETCH 1 RFC822.SIZE                    \textit{Get message sizes}
S: * 1 FETCH (RFC822.SIZE 2545)
S: A0003 OK FETCH completed
C: A0004 FETCH 1 BODY[HEADER]                   \textit{Get first message header}
S: * 1 FETCH (RFC822.HEADER {1425}
<server sends 1425 octets of message payload>
S: )
S: A0004 OK FETCH completed
C: A0005 FETCH 1 BODY[TEXT]                     \textit{Get first message body}
S: * 1 FETCH (BODY[TEXT] {1120}
<server sends 1120 octets of message payload>
S: )
S: * 1 FETCH (FLAGS (\Recent \Seen))
S: A0005 OK FETCH completed
C: A0006 LOGOUT
S: * BYE example.com IMAP4rev1 server terminating connection
S: A0006 OK LOGOUT completed
C: <client hangs up>
\end{Verbatim}

IMAP对有效载荷部分的分隔方法略有不同,它不是用一个点号来结束,而是将有效载荷的长度直接放在有效载荷之前发送。这稍稍增加了服务器的负担(消息必须提前完成组合,无法在初始化后流转),但使客户端工作更容易了——客户端可以提前知道需要分配多少存储空间作为整个处理消息的缓冲区。

同时,应注意,每个响应都标上了由请求提供的序列标签,本例中这个标签的形式为A000n,但客户端也可以在这个位置上生成任何其它标记。这个特性使IMAP命令无需等待响应就可以流向服务器端;然后客户端的状态机就能够在数据回来时直接解析响应和有效数据载荷。这样可以减少等待时间。

IMAP(为取代POP3协议而设计)是一个成熟而强大的互联网应用协议的优秀设计典范,值得学习和效仿。


\section{应用协议元格式}
就像数据文件元格式是为了简化存储的序列化操作而发展出来一样,应用协议元格式是为了简化网络间事务处理的序列化操作而发展出来的。但在这种情况中采取的折衷略有不同:因为网络带宽要比存储昂贵得多,所以需更加重视事务处理的经济性。尽管如此,文本格式的透明性和互用性优势仍然十分显著,所以大多数设计者还是抵制住了牺牲可读性来优化性能的诱惑。

\subsection{经典的互联网应用元协议}
Marshall Rose的RFC 3117《论应用协议的设计》( On the Design of Application Protocols)\footnote{从<ftp://ftp.rfc-editor.org/in-notes/rfc3117.txt>处参阅RFC 3117。}很好地概括了互联网应用协议设计中的种种问题。它明确了我们在分析SMTP、POP和IMAP时所注意到的一些经典互联网协议的描述手段( trope ),并对其
进行了具启发意义的分类。推荐大家一读。

经典的互联网元协议是文本格式,使用单行请求和响应,但有效数据载荷可以多行。

有效数据载荷要么是8位组数据作为前导,要么以“\verb+\r\n+”行作为结束符。在后一种情况下,有效数据载荷在字节上已被补齐;所有以句点“.”开始的行前面需要另加一个句点,接收方既负责识别结束符又负责去除补齐字节。应答行由状态码和后接人可识别的消息构成。

这种经典风格的关键优势是可以随时扩展。解析器和状态机框架无需太多修改就能够适应新的请求,而且代码也容易编写,使其可以解析未知请求并返回错误信息或直接忽略这些未知请求。SMTP、POP3和IMAP在使用过程中都经历了相当频繁的小扩展,互用性问题极少。相比之下,那些设计比较简单的二进制协议,却以不耐用而臭名昭著。

\subsection{作为通用应用协议的HTTP}
自从万维网在1993年左右吸引到足量用户以来,应用协议的设计者已经越来越倾向于在HTTP上构建专用协议,并使用网页服务器作为通用服务平台。

这是一种可行的方案,因为在事务层上,HTTP非常简单和通用。HTTP请求采用类似RFC-822/MIME格式的消息:通常,消息头包含识别和认证信息,第一行是对通用资源指示符( URI )指定的某个资源的方法调用。最重要的方法是GET(获取资源)、PUT(修改资源)和POST(将数据发送给某个表单或后端进程)。URI最重要的形式是URL,即统一资源定位符( Uniform Resource Locator);URL通过服务类型、主机名称和在主机上的位置对资源进行识别。HTTP响应只是一种RFC-822/MIME消息,可以包含由客户端解释的任意内容。

网页服务器处理HTTP的传输和请求多路复合层,同时也处理标准的服务类型,如http和ftp。要编写可以处理自定义服务类型的网页服务器插件相对比较容易,也很容易分派URI格式的其它元素。

除了避免很多底层细节之外,这种方法也意味着应用协议可以通过标准的HTTP服务端口,不需要自身的TCP/IP服务端口。这成为一个非常显著的优势:大多数防火墙都开放80端口,而试图穿透其它端口则可能遇到技术上和政治上的困难。

风险也伴随这种优势而来:这意味着网页服务器和插件会越变越复杂,任何这些代码的破解都可能带来巨大的安全问题。要隔离并关闭出问题的服务也可能会更加困难。通常此时就要在安全和便利间做出折衷。

RFC 3205《论使用HTTP作为底层》( On the Use of HTTP As a Substrate)\footnote{参见RFC 3205<http://www.fags.org/rfcs/rfc3205.html>}向正在考虑将HTTP作为应用协议底层使用的人提出了很好的设计建议,文中还总结了各种权衡方案和所涉及的问题。

\subsubsection{实例分析:CDDB/freedb.org数据库}
音频CD由一序列称为CDDA-WAV的数字格式音轨组成。在一般计算机拥有足够速度和声音处理能力来实时解码之前几年,音频CD是为让简单消费型电子产品能够播放而设计的。正因为如此,它没有提供相应的格式来记录甚至非常简单的一些元信息,如唱片专辑和歌曲音轨标题等。但是现代计算机中的CD播放器需要这些信息,这样用
户可以整理和编辑播放列表。

连上互联网,网上有(至少两个)资料库可提供根据CD上音轨长度表计算出的散列码与艺术家/专辑名/音轨名之间的对应记录。最初的一个资料库是cddb.org,但另一个名为freedb.org的站点也许现在资料更完整,用的人也更多。随着新CD的不断推出,这两个站点都依靠用户来完成更新数据库的巨大任务。当CDDB决定将用户提供的
所有信息都收为专有后,作为开发者的反抗,freedb.org发展了起来。

对这些服务的查询本可以作为基于TCP/IP的自定义应用协议来实现,但是那样就要求采取以下步骤,如给它分配一个新的TCP/IP端口号,还要为它费力地从成千上万的防火墙上争取到通路。为了避免这些麻烦,该服务基于HTTP作为简单的CGI查询来实现(就好像CD的散列码是用户通过在网页上填表提供的)。

由于作了这种选择,所有现行各种编程语言的HTTP和Web访问程序库的基础代码都可以支持数据库的查询和更新。结果,在CD播放器中增加这样的支持功能几乎不费吹灰之力,而实际上现在每款CD播放软件都知道如何使用这些功能。

\subsubsection{实例分析:互联网打印协议}
互联网打印协议(IPP,Internet Printing Protocol)是一个非常成功、被广泛采用的网络访问打印机控制标准。指向RFC的链接、各种实现和大量的其它相关材料都可以在IETF的打印机工作组站点<http://www.pwg.org/ipp/>获得。

IPP使用HTTP1.1作为传输层。所有的IPP请求都通过HTTP POST方法调用发送;响应是普通的HTTP响应。(《互联网打印协议模型和结构基本原理》( Rationale for the Structure of the Model and Protocol for the Internet Printing Protocot ),RFC 2568的4.2节很好解释了做出这种选择的理由,值得正在考虑编写新应用协议的所有人员研究)。

在软件方面,HTTP1.1得到了广泛应用。HTPP1.1已经解决了许多传输层问题,不然,这些问题将使协议的设计者和实现者无法集中精力解决打印的域语义问题。HTTP1.1能够很干净地进行扩展,因此IPP还有足够的发展空间。人们非常熟悉处理POST请求的CGI编程模型,开发工具也很丰富。

大多数具有网络能力的打印机已经内置了网页服务器,因为这是能人工远程查询打印机状态的一个自然的方法。这样,向打印机固件增加IPP服务的额外成本并不是很高。(这一点也适用于许许多多具有网络能力的硬件,包括自动售货机、自动咖啡机\footnote{参见RFC 2324<http://www.ietf.org/rfc/rfc2324.txt>和RFC 2325<http://www.ietf.org/rfc/rfc2325.txt>}和自动澡盆!)

基于HTTP的IPP协议的唯一严重缺点就是协议完全由客户端的请求来驱动。这样,模型中就不存在可供打印机向客户返回异步报警信息的余地(然而,聪明的客户可以运行一个很小的HTTP服务程序,来接收打印机发送的HTTP请求格式报警信息)。

\subsection{BEEP:块可扩展交换协议}
BEEP(原先叫BXXP)是一种通用协议,对于通用底层这一角色来说和HTTP一样具有竞争力。之所以说竞争,是因为目前为止还没有一个制定较完善的元协议非常适合真正的对等网(P2P)应用,不像客户端-服务器应用——HTTP在这一领域游刃有余。访问<http://www.beepcore.org/beepcore/docs/sl-beep.jsp>项目网站可以找到有关标准以及数种语言的开源实现。
 
BEEP的特点是既支持客户端/服务器模式,又支持对等网模式( peer-to-peer)。协议作者们对BEEP协议和支持库进行了良好设计。这样,只要选取正确组件就能省去很多杂务,如数据编码、流控、堵塞(congestion)处理、端对端加密支持和用多路传输组成长应答等。

从内部而言,BEEP用户端(peer)之间相互交换自描述二进制包序列,后者与PNG中字节块类型非常相像。跟经典的互联网协议或HTTP相比,BEEP设计对经济性的强调高于透明性,当数据量非常大时可能是更好的选择。BEEP也避免了HTTP所有请求都必须由客户端发起的问题:在服务器端需要向客户端异步返回状态信息的情况下,BEEP协议会更好。

到2003年年中,BEEP仍然是一项新兴技术,只有几个演示项目。但是BEEP论文是非常不错的关于设计最佳协议的分析材料:即使BEEP本身无法得到广泛采用,这些论文仍然具有相当重要的指导价值。

\subsection{XML-RPC,SOAP和Jabber}
应用协议设计的一种发展趋势是在MIME中使用XML来架构请求和有效数据载荷。BEEP用户端使用这种格式进行信道协商。有三个重要协议始终采用XML路线,包括XML-RPC,用于远程过程调用的SOAP(Simple Object Access Protocol,简单对象访问协议)和用于即时消息和在线状态报告(instant messaging and presence)的Jabber。这三个协议都基于XML文档。

XML-RPC非常具有Unix精神(其作者宣称在二十世纪七十年代通过阅读原始的Unix源码学会了编程)。它着意追求最简化,但是仍然非常强大,对于绝大多数专为传送布尔/整数/浮点/字符串标量数据类型的各种RPC应用提供一种方法,使它们能够以一种轻量、易于理解和监控的方式来完成任务。XML-RPC类型实体比文本流丰富,但仍然简单而具有移植性,可被有效地用于检查接口复杂性。它有很多开源实现。XML-RPC主页<http://www/xmlrpc.com/>做得不错,提供了规格书和多个开源实现。

SOAP是一个较重量级的RPC协议,数据类型更为丰富,包括数组和类似C的结构。这个协议受到了XML-RPC的启发,但一直被批评为过分追求“第二系统效应”设计的受害者,这么说不无道理。直到2003年中,SOAP标准还在制定当中,但是在Apache上的试验版实现一直紧跟其设计草案。大家可以很容易地在网上搜索到Perl、Python、Tcl和Java中的开源客户端模块。W3C规范的草案可从<http://www.w3.org/TR/SOAP/>获得。

作为远程过程调用方法的XML-RPC和SOAP都会带来某种风险,我们将在第7章结尾部分予以讨论。

Jabber是一个为支持即时消息和在线状态报告而设计的对等协议。Jabber作为应用协议的有趣之处在于:它支持XML表单和随时更新文档(live document)的输送。规格说明、文档和开源实现都可以通过Jabber款件基金会的站点<http://www.jabber.org/about/overview.html>获得。


\chapter{透明性:来点儿光}





\end{common-format}  
\end{document}
    Beauty fJmorefmportant加computing than anywhere else in technology because
sofiware is so complicated. Beauty is the ultimate defense against complexity.
    美在计算科学中的地位,要比在其它任何技术中的地位都重要,因为软件是太复杂
了。美是抵御复杂的终极武器。
    Machine Beauty:Elegance and the Heart of Technoloogy  (1998)
    《机器美学:优雅和技术本质》(1998年)
    -David Gelernter
    我们在前一章讨论了把数据格式和应用协议进行文本化的重要性,这种方式的表
达容易让人分析和参与,使一些设计品质得以提升,虽然Unix传统非常重视这些品质,
但很少(如果有的话)明确谈论过,那就是:透明性和可显性。
    如果没有阴暗的角落和隐藏的深度,软件系统就是透明的。透明性是一种被动品质。
如果实际上能预测到程序行为的全部或大部分情况,并能建立简单的心理模型,这个程
序就是透明的,因为可以着透机器究竟在于什么。
    如果软件系统所包含的功能是为了帮助人们对软件建立正确的“做什么、怎样做,,
的心理模型而设计,这个软件系统就是可显的。因此,举例来说,对用户而言,良好的
UNIX编程艺-宋
  www.pdf365.com
1 34
第6章透明性:来点儿光
文档有助于提高可显性;对程序员而言,良好的变量和函数名有助于提高可显性。可显
性是一种主动品质。在软件中要达到这一点,仅仅做到不晦涩是不够的,还必须尽力做
到有帮助1。
    透明性和可显性对用户和软件开发人员都很重要。但是重要性体现在不同的方面。
用户喜欢UI中的这些特性,是因为这意味着学习曲线比较平缓。当人们说UI“直观’’
时,很大程度上是指UI的透明性和可显性;剩下一部分则来源于最小立异原则。我们将
在第1 1章更深入分析这些使用户界面舒适而有效的特性。
    软件开发者喜欢代码本身(用户不可见部分)的这些品质,因为他们经常需要对代
码有很好理解后才能进行修改和调试。同时,如果程序的设计使内部数据流程非常容易
理解,则这个程序更不可能因设计者没有注意到的不良交互而崩溃,更可能优雅地向前
发展(包括适应新维护者接手的变化)。
    透明性是本章引言David Gelemter所说的“美”的主要构成部分。Unix程序员借鉴
了数学家的说法,经常使用更明确的术语“优雅”来表达Gelemter所说的品质。优雅是
力量与简洁的结合。优雅的代码事半功倍;优雅的代码不仅正确,而且显然正确;优雅
的代码不仅将算法侍达给计算机,同时也把见解和信心传递给阅读代码的人。通过追求
代码的优雅,我们能够编写更好的代码。学习编写透明的代码是学习如何编写优雅代码
的第一关,很难的一关——而关注代码的可显性则帮助我们学习如何编写透明的代码。
优雅的代码既透明又可显。
    通过两个极端例子来辨别透明性和可显性之间的差别可能会更容易些。Linux的内核
源码相当透明(相对其行为的内在复杂性而言),但根本不具备可显性——要获得必要
的知识以便融入代码中并理解开发者的惯用语相当困难,而一旦做到,~切便豁然开朗2。
另一方面,Emacs Lisp库是可显的,但却不透明。要获得足够的知识来领会一件事很容
易,但要理解整个系统却相当困难。
1一位有经济头脑的朋友评论:  “可显性降低进入门槛;透明性则减少代码中的存在成本。”
2 Linux内核在可显性上做了很多努力,包括Linux内核源码tarball中的Documentation子目录和相
当多的教程网站和指导书籍。这些努力比起内核变化的速度来说远远不够,文档还将长期落后。
UNIX编程艺术
    www.pdf365.com
6.1研究实例
135


\chapter{附录}
缩写术语表如API之类的请自行搜索之,这里就省略了。
参考文献部分也省略了,如果有需要请查看原始pdf文档。

%这里空一行

\end{common-format}  
\end{document}



